 \documentclass[10pt,a4paper]{article}
\usepackage[latin2]{inputenc}
\usepackage[spanish]{babel}
\usepackage{amsmath}
\usepackage{amsfonts}
\usepackage{amssymb}
\usepackage{makeidx}
\usepackage{graphicx}
\usepackage[left=2cm,right=2cm,top=2cm,bottom=2cm]{geometry}

\begin{document}

\section*{7. Conclusiones:}\\
En general, al realizar la pr\'{a}ctica pudimos observar que la propiedad intr\'{i}nseca de dos materiales ls cuales fueron Aluminio y Cobre  que fu\'{e} su coeficiente de dilataci\'{o}n lineal, a ambas barras  las hicimos variar a cierta temperatura, se encontr\'{o} un comportamiento esperado, al menos hasta la temperatura que llegamos encontramos el comportamiento de una funci\'{o}n lineal estrictamente creciente de con los par\'{a}metros $\Delta T$ y $\Delta L$, y que sin embargo no alcanzamos los 100 grados celsius esto es para alcanzar en valor m\'{a}s preciso del coeficiente de dilataci\'{o}n lieal. 
\section*{8. Referencias:}\\
\\
\medskip
\\
\\1.- Bit\'{a}cora de laboratorio de Flores Rodr\'{i}guez Jaziel David.
\\
2.- Manual de pr\'{a}cticas auxilar. Autor: Fco. Havez Varela y las notas del profesor Salvador Tirado Guerra.
\\
3.- Fisica Universitaria - Sears - Zemansky - 12ava Edici\'{o}n - Cap\'{i}tulo l7 -2009.\\
4.- http://www.tainstruments.com/productos/dilatometros/?lang=es&gclid=CjwKEAjwpJ_JBRC3tYai4Ky09zQSJAC5r7ruRo-sNaN2pynD6aFmyTkhKCPjCUz-kK0ZBzrR2CdMtBoCuwcB \\
\end{document}