\documentclass[10pt,a4paper]{article}
\usepackage[latin2]{inputenc}
\usepackage[spanish]{babel}
\usepackage{amsmath}
\usepackage{amsfonts}
\usepackage{amssymb}
\usepackage{makeidx}
\usepackage{graphicx}
\usepackage[left=2cm,right=2cm,top=2cm,bottom=2cm]{geometry}
\author{Flores Rod\'{i}guez Jaziel David }
\title{¿C\'{o}mo y por qu\'{e} los Mexicanos tenemos tanta historia?}
\begin{document}
\begin{figure}
\centering
\includegraphics[width=6cm]{../../../../../Pictures/MNAH-Mexico-02.jpg} 
\includegraphics[width=8cm]{../../../../../Pictures/IMG_20170228_175017880_HDR.jpg} 
\end{figure}
Hace bastante tiempo que no me interesaba la historia ni mucho menos el contexto en el que mi pa\'{i}s se encuentra ahora. Hace bastante tiempo comenc\'{e} a involucrarme en conceptos m\'{a}s "generales", en los que en alg\'{u}n sue\~{n}o comenzaban a flotar, entes extra\~{n}os absueltos de cualquier interpretaci\'{o}n subjetiva, \'{e}psilons y deltas que recorren el \'{i}nfimo detalle en el supremo de cada esquina, era incre\'{i}ble apropiarse de una rama tan bonita y tan abstracta, que yo ciegamente cre\'{i} eran los \'{u}nicos que realmente trascendian la realidad y pod\'{i}an transformar el mundo. Hace poco escuch\'{e} que nuestra naci\'{o}n no crec\'{i}a porque su gente no ten\'{i}a identidad, una idetidad nacional, y adoptaba una ya hecha, como lo es la americana consumista de lujos con su vida bohemia, eso se ve m\'{a}s notorio e esta cultura  "juvenil" hiper conectada con el mundo que comienzan a anhear una forma de vivir como la de estadounidense promedio, de hamburguesas y centros comerciales para despilfarrar el dinero en cosas que no necesitamos. Basta de introducciones,  mi recorrido comieza con una agradable bienvenida, por el Dios Tl\'{a}loc, un monolito de pieda (de los m\'{a}s grandes del mundo) que se encuentra a las afueras del museo, de las cosas m\'{a}s impresionantes al entrara a las salas es el enorme paraguas con cascada el se compone de una columna de contreto que sostiente a enorme superficie c\'{o}ncava de una extensi\'{o}n de dimensiones colosales, comec\'{e} a recordar mis clases de F\'{i}sica 2, me imaginaba el peso que generaba esa superficie que genera la cascada, que actuaba perpedicularmente al piso, era impresionane siquiera pensar los esfuerzos axiales qe soportaba alg\'{u}n punto de la columna que loa saten\'{i}a, tambi\'{e} su m\'{o}dulo de compresi\'{o}n volum\'{e}trica de material, sin duda eso no se caer\'{i}a, tantos a\~{n}os y aguantar\'{i}a muchos m\'{a}s, me sent\'{i}a peqe\~{n}o frente ante na escultura a la que conoc\'{i}a sus impicaciones f\'{i}sicas, vaya que estaba frente a una de las "cubiertas colgantes" m\'{a}s grandes del mundo. \\
\begin{figure}[hbtp]
\centering
\includegraphics[width=6.5cm]{../../../../../Pictures/descarga.jpg} 
\includegraphics[width=7.7cm]{../../../../../Pictures/IMG_20170318_182828657.jpg} 
\caption{Restos de mamut encontrados en mi pueblito natal. }
\end{figure}
\\
Recorr\'{i} la historia de mi pa\'{i}s a traves de una caminanta progresiva, mi novia y yo parados enfrente de nuestro pasado, parec\'{i} no tener sentido seguir all\'{i}, ten\'{i}a muy arraigada la idea de que un f\'{i}sico o matem\'{a}tico se la deben pasar estudiando todo el d\'{i}a frente algu\'{u}n ibro en la biblioteca, sin embargo, ha cambiado \'{u}timamene mi percepci\'{o}n del mundo, la manera de como observar mi entorno debe ser m\'{a}s profunda, como lo ha dicho mi profesor actual de f\'{i}sica; un cient\'{i}fico debe preguntarse m\'{a}s y dedicarle mucho tiempo a observar los detalles de la naturaleza, pero creo yo que la naturaleza abarca ya los contextos historiocos del hombre, es decir, espectador-marco de referencia con fen\'{o}meno-suceso. Sin m\'{a}s por agregar, comenzamos ambos en la historia de las primeras civilizaciones en America. Lo que a mi novia y a mi nos tom\'{o} por sorpresa es ver al mamut encontrado en nuestra comunidad Santa Isabel Ixtapan localizada en el municipio de Atenco Estado de M\'{e}xico, aunque lo m\'{a}s extra\~{n}o es que en el vidrio que conten\'{i}a los restos tiene un error, colocaron el nombre como Iztapan. Como ambos ten\'{i}amos el tiempo com\'{i}endonos encima, continuamos nuestro recorrido por las salas en las que exhiben a los teotihuacanos, amos hab\'{i}amos ido a visitar las pir\'{a}ides de de teotihuacan, pero jam\'{a}s hab\'{i}a prestado atenci\'{o}n de verdad. Ser\'{i}a un buen lugar para vivir, pens\'{e} entre alg\'{u}n comentario y broma. La personita que me acompa\~{n}aba explicaba la cosmovisi\'{o}n que ten\'{i}an nuestras culturas prehisp\'{a}nicas, pero sobre todo me imprsion\'{o} el gran respeto hacia con su tierra.\\
\begin{figure}[hbtp]
\centering
\includegraphics[width=6cm]{../../../../../Pictures/IMG_20170228_183139381.jpg} 
\includegraphics[width=6cm]{../../../../../Pictures/IMG_20170228_183150715.jpg}  
\end{figure}
\\
Pero sobre todo que deb\'{i}an de adquirir gran conocimiento quien las trabajasen. Pude observar que los visitantes en su mayor\'{i}a, y m\'{a}s porque era entre semana, extranjeros de todas las nacionalidades.Despu\'{e}s pasamos a la sala de las civilizaciones del valle de M\'{e}xico, era poca mi bater\'{i}a y muchas fotos las que hab\'{i}a por tomar.\\
\begin{figure}[hbtp]
\centering 
\includegraphics[width=6cm]{../../../../../Pictures/IMG_20170228_181931939.jpg} 
\includegraphics[width=6cm]{../../../../../Pictures/IMG_20170228_183107822.jpg} 
\end{figure}
\\
Me impresion\'{o} ver semejantes monolitos y piedras tan grandes talladas a partir de una visi\'{o}n incluyente del mundo. La percepci\'{o}n de mi pa\'{i}s comenzaba a cambiar, sobre todo a recorrer su historia y su contexto en el que trabaja ahora, ?`c\'{o}mo puede un pa\'{i}s como M\'{e}xico tener tanta historia y tanta biodiversidad sin siquiera poder verlo apreciarlo? Me daba un poco de tristesa no prestarle atenci\'{o}n a mi tierra como nuestros antepasados se extaciaban con ello.\\
\begin{figure}[hbtp]
\centering 
\includegraphics[width=6cm]{../../../../../Pictures/IMG_20170228_181952805.jpg} 
\includegraphics[width=6cm]{../../../../../Pictures/IMG_20170228_181939615.jpg}    
\end{figure}
\\
\\
Cabe mencionar, que aunque no lo visit\'{e}, e el museo hay un arco falso o arco maya el cual es una construcci\'{o}n con forma de arco obtenida a base de colocar a ambos lados de un conjunto bloques de piedra escalonados de manera uniforme, hasta que se encuentran en un punto medio. Es curioso este arco porque a diferencia de un verdadero arco, un falso arco no funciona solamente por esfuerzos compresi\'{o}n, es decir, a pesar de podr\'{i} haber un avance frente al sistema horizontal que salva a un espacio entre dos apoyos, su eficiencia es limitada.\\
\begin{figure}[hbtp]
\centering
\includegraphics[width=6cm]{../../../../../Pictures/arco-falso-maya.jpg} 
\end{figure}
\\
\pagebreak
\\
Finalmente recorr\'{i} la \'{u}ltima sala a la que el tiempo me perit\'{i}a, era una exposici\'{o}n temporal llena de color, sobre una etnia de San Luis Potos\'{i}, los Huichloes,  era sobre universitarios de esta etnia que quisieron que el mundo coociera su identidad y sus costumbres, ciertamente me impresion\'{e} ver sus rituales donde caminaban horas y d\'{i}as para comer al dios peyotl, tambi\'{e}n hab\'{i}a muestras de su arte. Sab\'{i}a que era el peyote, por un ensayo de Aldous Huxley sobre los psicodelicos, conoc\'{i}a la experiencia de un estado de conciencia semejante\\
\begin{figure}[hbtp]
\centering
\includegraphics[width=10cm]{../../../../../Pictures/IMG_20170228_180406674.jpg}  
\end{figure}
\\
Internacionalmente los Huicholes son conocidos por eso, el peyote, los extranjeros vienen y lucran con el, y hasta los mismos mexicanos lo cambian por unas monedas, a tal punto de que esta cactacea se encuentre en peligro de extinci\'{o}n, me di\'{o} m\'{a}s tristeza, investigando m\'{a}s de esa etnia, saber que mineras extranjeras lucrar\'{i}an con sus tierras santas.
\\
\begin{figure}[hbtp]
\centering
\includegraphics[width=7cm]{../../../../../Pictures/045n1soc-1.jpg} 
\end{figure}
\\
Para algunos lo ver\'{i}an como algo ajeno, pero en mi visita y en su cosmovisi\'{o}n del mundo, profanar sus tierras sagradas por s\'{o}olo dinero es lo mismo que los espa\~{n}oles hicieron durante su conquista. La historia se construye todos los d\'{i}as y si uno no conoce su entorno, pronto el tiempo se lo comer\'{a} porque vivi\'{o} sin herramientas, ahora siento que las ciencias del hombre deben estar en beneficio del hombre y de su entorno, practicanete u beneficio colectivo. Y no s\'{o}lo en beneficio de una \'{e}lite. 
\end{document}