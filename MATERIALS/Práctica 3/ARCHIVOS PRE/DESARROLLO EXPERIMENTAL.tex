\documentclass[10pt,a4paper]{article}
\usepackage[latin1]{inputenc}
\usepackage[spanish]{babel}
\usepackage[utf8]{inputenc}
\usepackage{amsmath}
\usepackage{amsfonts}
\usepackage{amssymb}
\usepackage{graphicx}
\usepackage[left=2cm,right=2cm,top=2cm,bottom=2cm]{geometry}
\begin{document}
\section*{Desarrollo Experimental.}\\
\textbf{Lista de Materiales:} \\
Instrumento espec\'{i}fico de laboratorio que mide fuerzas muy peque\~{n}os.\\
Term\'{o}metro de mercurio. \\
Un cron\'{o}metro. \\
Un arillo met\'{a}lico con soporte en un s\'{o}lo punto para conectarlo a un gancho. \\
Aceite de motor SAE 40.\\
Parrilla el\'{e}ctrica. \\
Vaso de precipitados.\\
\\
\section*{Arreglo experimenal.}
\begin{figure}[hbtp]
\centering
\\
\includegraphics[width=14.5cm]{../../../../../Pictures/ARREGLOTENSION.jpg} 
\caption{Arreglo general}
\end{figure} 
\\
\begin{figure}[hbtp]
\centering
\\
\includegraphics[width=8cm]{../../../../../Pictures/IMG_20170221_130403399.jpg} 
\includegraphics[width=8cm]{../../../../../Pictures/IMG_20170221_123835908.jpg} 
\caption{Variaci\'{o}n de la temperaura del fluido con una parrilla el\'{e}ctrica y medida con term\'{o}metro} 
\end{figure} 
\\
\section*{Procedimiento.}
1.-Colocar aceite en un vaso de precipitados y calentar en un mechero.\\
2.-Dejar calentar a T=1000 grados cent\'{i}grados.\\
3.-Colocar el arillo sobre la superficie. \\
4.-Despegar el arillo midiendo la fuerza.\\
5.-Dejar que baje la T, y repetir los pasos 3) y 4) para 10 o 12 T's diferentes .\\
6.-LLenar la tabla y graficar $\gamma$ vs T, y ajustar para obtener $\gamma= \gamma (T)$ .
\textbf{M\'{e}todo para obtener F.}\\
1.-Medir el peso w del arillo.\\
2.-Calibrar el aparato y suspender el arillo.\\
3.-Re-equilibrar el aparato , medir $\alpha $, y calcular k. (Cada divisi\'{o}n en la escala es igual a 2 grados). \\
4.-Acerar el aparato para poder medir el $\theta$ de despegue. Y poder calcular la F de despegue.\\ 
\section*{Resultados.}
Comenzamos la pr\'{a}ctica con ell material proporcionado por el equipo de laboratorio, procedimos a medir algunas de nuestras constantes como lo son; el radio del anillo$r=8.5\times { 10 }^{ -3 } m $, masa del anillo $m=7.68\times { 10 }^{ -4 } kg$, y por el m\'{e}todo anterior se obtuvo la constante $w=113.6$ grad y en consecuencia se hall\'{o} la consante k del brazo que sostiene al anillo $k=6.62952\times { 10 }^{ -6 } N/m$, la distancia entre el centro del instrumento de medici\'{o}n hasta donde est\'{a} sujetado en anillo $b=1\times { 10 }^{ -1 }m$. Llenamos la Tabla 1 proporcionada para despu\'{e}s  graficar $\gamma$ vs T, y ajustar para calcular  $\gamma= \gamma (T)$, es decir encontrar una dependencia de la tensi\'{o}n superficial con respecto a la temperatura del fluido.
\\
\begin{figure 2}
\caption{\textbf{Tabla 1.}}
\\
\centering
\includegraphics[width=8cm]{../../../../../../Pictures/Tabla1tensionsup.png}    
\\
\end{figure 2}
\\
\medskip
De las cuales extragimos los siguientes datos de la siguiente tabla para poder hacer la gr\'{a}fica $\gamma vs T$.
\medskip
\\
\begin{figure 3}
\caption{\textbf{Tabla 2.}}
\\
\centering
\includegraphics[width=5cm]{../../../../../../Pictures/Tabla2tensionsup.jpg} 
\\
\caption{Datos tabulados para graficar.}
\end{figure 3}
\\
\section*{Gr\'{a}fica de dispersi\'{o}n}
\\
\\
\begin{figure 5}

\centering
\includegraphics[width=10cm]{../../../../../Pictures/GRAFICADEDISPER.jpg} 
\\
\caption{Gr\'{a}fico de dispersi\'o}n para los datos de la tensi\'{o}n superficial a trav\'{e}s del tiempo.}
\end{figure 5}
\\
\section*{Ajuste de datos.}\\
Por el Apr\'{e}ndice 1 podemos hacer el respectivo ajuste por el m\'{e}todo de m\'{i}nimos cuadrados para encontrar un modelo lineal ${Y}_{1}={a}_{1}x +{b}_{1}$ tales que $\left( { x }_{ i },{ y }_{ i } \right) \rightarrow \left( T( C ),{\gamma (N/m))$  de los datos del experimento, cuya tablas de entrada y modelo es el siguiente:
\medskip
\\
\textbf{Tabla de entrada.}
\\
\begin{figure 6}
\centering
\begin{tabular}{|c|c|c|c|c|}
\hline 
n&$\sum _{ i=1 }^{ n }{ { x }_{ i } } $( C) & $\sum _{ i=1 }^{ n }{ { y }_{ i } }$(N/m) & $ \sum _{ i=1 }^{ n }{ { y }_{ i } } { x }_{ i }$ (N C/m) & $\sum _{ i=1 }^{ n }{ { x }_{ i }^{ 2 } }({ C }^{ 2 })$ \\ 

\hline 
11&$6.5\times { 10 }^{ 2 }$& $3.21\times { 10 }^{ -1 }$ & $2.1901\times { 10 }^{ 2 }$ & $4.9225\times { 10 }^{ 4 }$\\ 
\hline 
\end{tabular}
\end{figure 6} 
\\
\\
De donde:
\[{a}_{1}=\frac { n\sum _{ i=1 }^{ n }{ { x }_{ i }{ y }_{ i } } -\sum _{ i=1 }^{ n }{ { x }_{ i } } \sum _{ i=1 }^{ n }{ { y }_{ i } }  }{ n\sum _{ i=1 }^{ n }{ { x }_{ i }^{ 2 } } -{ \left( \sum _{ i=1 }^{ n }{ { x }_{ i } }  \right)  }^{ 2 } } \quad y\quad {b}_{1}=\frac { \sum _{ i=1 }^{ n }{ { x }_{ i }^{ 2 } } \sum _{ i=1 }^{ n }{ { y }_{ i } } -\sum _{ i=1 }^{ n }{ { x }_{ i }{ y }_{ i } } \sum _{ i=1 }^{ n }{ { x }_{ i } }  }{ n\sum _{ i=1 }^{ n }{ { x }_{ i }^{ 2 } } -{ \left( \sum _{ i=1 }^{ n }{ { x }_{ i } }  \right)  }^{ 2 } }.\]
Sustituyendo los valores queda:

\[{a}_{1}=\frac { 11\times 2.1901\times { 10 }^{ 2 } -\left[6.5\times { 10 }^{ 2 } \times 3.21\times { 10 }^{ -1 } \right]  }{ 11\times 4.9225\times { 10 }^{ 4 }-{ \left( 6.5\times { 10 }^{ 2 } \right)  }^{ 2 } } ={-4\times 10 }^{ -5 } (N/m C).\]

\[{b}_{1}=\frac { 4.9225\times { 10 }^{ 4 } \times  3.21\times { 10 }^{ -1 } - \left[2.1901\times { 10 }^{ 2 } \times 6.5\times { 10 }^{ 2 } \right] }{ 11\times 4.9225\times { 10 }^{ 4 }-{ \left( 6.5\times { 10 }^{ 2 } \right)  }^{ 2 } } ={2.37\times 10 }^{ -2 } (N/m).\]
Finalmente queda el modelos propuesto:
\[{ Y }_{ 1 }= {-4\times 10 }^{ -5 }x + {2.37\times 10 }^{ -2 }  \quad \rightarrow(1) \]
De (1)  podemos graficar sus correspondientes lineas de tendencia, a continuaci\'{o}n vamos a graficar su modelo y a tratarlo con m\'{a}s detalle.
\pagebreak

\section*{Ajuste por excel.}\\
Notemos que mientras por medio de c\'{a}alculos pudimos enconrar un modoelo, el programa Excel pudo enontrar otro, es cual se muestra a continuaci\'{o}n.
\\
\begin{figure}[hbtp]
 \centering
\includegraphics[width=8cm]{../../../../../Pictures/ARREGLOEXELLELELE.jpg} 
\end{figure}
\\
Por la ecuaci\'{o}n anterior se tiene que $\gamma= \gamma (T) = {-4\times 10 }^{ -5 }(T) + {2.37\times 10 }^{ -2 }$ y as\'{i} se logra obtener una relacio\'{o}n entre la tensi\'{o}n superficial y la temperatura. Cabe resaltar que se ha obtenido lo que se esperaba, la tensi\'{o}n superficial baja mientras sube la temperatura, sin embargo hubo UN PUNTO en el que estaba completamente fuera del rango esperado as\'{i} que lo podemos omitir en nuestro an\'{a}lisis como un error de medici\'{o}n, tambi\'{e}n a\~{n}adimos que el aceite de motor es un SAE 40, es decir  que este aceite se usa para trabajos pesados y en tiempo de mucho calor y por lo tanto  no cambia mucho sus propiedades a altas temperaturas y como veremos m\'{a}s adelante su viscosidad. Recalcamos la ecuaci\'{o}n que relaciona su tensi\'{o}n superficial con su temperatura es:
\[ \gamma (T) = {-4\times 10 }^{ -5 }(T) + {2.37\times 10 }^{ -2 } \]
\\
\section*{Error Porcentual.}\\
Los valores verdaderos de su tensi\'{o}n superficial a 20 grados celsius es de 0.0275(N/m). Entonces, de nuestras mediciones y c\'{a}lculos podemos obtener el error porcentual:

\[{ E }rror\quad porcentual\quad { M }_{ c }=\frac { Error\quad verdadero }{ Valor\quad Verdadero } =\frac { Valor\quad verdadero - Valor\quad aproximado }{ Valor\quad verdadero } \times 100= 23\%. \]
\\
 \section*{Discusiones.}\\
Al llevar acabo este experimento pudimos notar que hubo ciertas fallas con la calibraci\'{o}n del instrumento, el cual nos cost\'{o} entender. Fuera de eso los puntos fueron un poco locos ya que algunas veces el valor de la tensi\'{o}n superficial oscilaba entre un valor y otro, esto puede ser desde que no nos percatabamos con suficiente sencibilidad, el momento en que el instrumento marcaba equilibrio. Finalmente se logr\'{o} encontrar una ecuaci\'{o}n de tendencia que era, no del todo, pero esperada al comportamiento de la temperatura. 



\end{document}