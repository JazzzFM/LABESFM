\documentclass[10pt,a4paper]{article}
\usepackage[latin1]{inputenc}
\usepackage[spanish]{babel}
\usepackage[utf8]{inputenc}
\usepackage{amsmath}
\usepackage{amsfonts}
\usepackage{amssymb}
\usepackage{graphicx}
\usepackage[left=2cm,right=2cm,top=2cm,bottom=2cm]{geometry}
\begin{document}
\section*{Desarrollo Experimental.}\\
\textbf{Lista de Materiales:} \\
Instrumento de laboratorio con el cual se introducir\'{a} eneg\'{i}a t\'{e}rmica.\\
Matr\'{a}z de vidrio para almacenar vapor. \\
Dos mangueras de caucho. \\
Dos barras met\'{a}licas de distintos materiales (Aluminio y Cobre). \\
Tornillo microm\'{e}trico. \\
Mult\'{i}metro digiital con escalas.\\
\section*{Arreglo experimenal.}
\begin{figure}[hbtp]
\centering
\\
\includegraphics[width= 12 cm]{../../../../../../Pictures/LLLLLLLLLL.jpg} 
\caption{Arreglo general. Notemos que esta vez, a diferencia de otras pr\'{a}cticas nos basaremos en el hecho de que la resistencia ele\'{e}ctrica cambia con la temperatura, esto es, desde que no podr\'{i}amos tomar la temperatura de la barra con un term\'{o}metro convencional de Mercurio o Alcohol. Esto justifica los valores mostrados de la tabla de temperatura y resistencia en la cual nos basaremos. }
\end{figure} 
\\
\begin{figure}[hbtp]
\centering
\\
\includegraphics[width=7 cm]{../../../../../../Pictures/IMG_20170404_120403996.jpg} 
\includegraphics[width=7 cm]{../../../../../../Pictures/IMG_20170404_120337161.jpg} 
\caption{Elevaci\'{o}n de la temperatura del agua y que el vapor caliente pasa por la manguera de caucho.} 
\end{figure} 
\\
\pagebreak

\section*{Procedimiento.}
1.- Medir la longitud ${L}_{0}$ de la muestra a la temperatura ambiente ${T}_{0}$.\\
2.-	Ajustar las puntas en contacto de manera que el tornillo microm\'{e}trico mida cero.\\
3.-	Correr el tornillo microm\'{e}trico para dejar dilatar libremente a la muestra. \\
4.-	Calentar hasta $100 \°C$ y hacer contacto para medir la elongaci\'{o}n.\\
5.-	Dejar bajar la temperatura, y cada $5 \°C$ ir midiendo la elongaci\'{o}n. \\
6.-	Llenar la tabla y hacer la gr\'{a}fica $\Delta  l$ vs $\Delta  T$, ajustar y calcular $\alpha$.\\
\medskip
\section*{Resultados.}
\\
Comenzamos la pr\'{a}ctica con ell material proporcionado por el equipo de laboratorio, procedimos a medir algunas de nuestras constantes como lo son; la longitud inicial de la barra de Aluminio $l_{0}=75.1 \times 10^{-2} m$ y la longitud inicial de la barra de cobre $L_{0} = 75.3 \times 10^{-2} m$, adem\'{a}s de las de la temperatura ambiente $T_{0}$, cabe resaltar que esta vez no fue una sola, porque mientras se realiz\'{o} con un material a otro cambi\'{o} la temperatura ambiente 2 grados celsius, es decir para la primer prueba $T_{0}= 23 \°C$ mientras para el otro fu\'{e} de $T_{0}=25 \°C$. Llenamos la Tabla 1 para ambos materiales proporcionada para despu\'{e}s  graficar $\Delta L$ vs $\Delta T$, y ajustar para calcular el coeficiente de expansi\'{o}n lineal $\alpha$.\\
\\
\medskip
\begin{figure 2}
\caption{\textbf{Tabla 1.}}
\\
\centering
\includegraphics[width=8cm]{../../../../../../Pictures/JEJEJEJEJE.jpg} 
\includegraphics[width=7.35cm]{../../../../../../Pictures/jajajajajajaa.jpg} 
\\
\end{figure 2}
\\
\medskip
De las cuales extragimos los siguientes datos de la siguiente tabla para poder hacer la gr\'{a}fica $\Delta L$ vs $\Delta T$.
\medskip
\\
\\
\begin{figure 3}
\caption{\textbf{Tabla 2.}}
\\
\centering
\includegraphics[width=5cm]{../../../../../../Pictures/pepepepe.jpg} 
\includegraphics[width=4.65cm]{../../../../../../Pictures/oeoeoeeooe.jpg} 
\\
\caption{Datos tabulados para graficar.}
\end{figure 3}
\\
\section*{Gr\'{a}fica de dispersi\'{o}n}
\\
\\
\begin{figure 5}
\centering
\includegraphics[width=8cm]{../../../../../../Pictures/CACACACA.jpg}
\includegraphics[width=8cm]{../../../../../../Pictures/CACACA2.jpg}  
\\
\caption{Gr\'{a}ficos de dispersi\'o}n del lugar geom\'{e}trico de la diferencias de temperatura y diferencias de longitudes .}
\end{figure 5}
\\ 
\section*{Ajuste de datos.}\\
Por el Apr\'{e}ndice 1 podemos hacer el respectivo ajuste por el m\'{e}todo de m\'{i}nimos cuadrados para encontrar un modelo lineal ${Y}_{1}={a}_{1}x +{b}_{1}$ tales que $\left( { x }_{ i },{ y }_{ i } \right) \rightarrow \left( \Delta T( \°C ),{\Delta L (m))$ y tambi\'{e}n un segundo modelo lineal ${Y}_{2}={a}_{2}x +{b}_{2}$ tales que $\left( { x }_{ j },{ y }_{ j } \right) \rightarrow \left( \Delta T( \°C ),{\Delta L (m))$de los datos del experimento, cuya tablas de entrada y modelo son las siguientes, de donde:
\[{a}_{1}=\frac { n\sum _{ i=1 }^{ n }{ { x }_{ i }{ y }_{ i } } -\sum _{ i=1 }^{ n }{ { x }_{ i } } \sum _{ i=1 }^{ n }{ { y }_{ i } }  }{ n\sum _{ i=1 }^{ n }{ { x }_{ i }^{ 2 } } -{ \left( \sum _{ i=1 }^{ n }{ { x }_{ i } }  \right)  }^{ 2 } } \quad y\quad {b}_{1}=\frac { \sum _{ i=1 }^{ n }{ { x }_{ i }^{ 2 } } \sum _{ i=1 }^{ n }{ { y }_{ i } } -\sum _{ i=1 }^{ n }{ { x }_{ i }{ y }_{ i } } \sum _{ i=1 }^{ n }{ { x }_{ i } }  }{ n\sum _{ i=1 }^{ n }{ { x }_{ i }^{ 2 } } -{ \left( \sum _{ i=1 }^{ n }{ { x }_{ i } }  \right)  }^{ 2 } }.\]
Y adem\'{a}s.
\[{a}_{2}=\frac { n\sum _{ j=1 }^{ n }{ { x }_{ j }{ y }_{ j } } -\sum _{ j=1 }^{ n }{ { x }_{ j } } \sum _{ j=1 }^{ n }{ { y }_{ j } }  }{ n\sum _{ j=1 }^{ n }{ { x }_{ j }^{ 2 } } -{ \left( \sum _{ j=1 }^{ n }{ { x }_{ j } }  \right)  }^{ 2 } } \quad y\quad {b}_{2}=\frac { \sum _{ j=1 }^{ n }{ { x }_{ j }^{ 2 } } \sum _{ j=1 }^{ n }{ { y }_{ j } } -\sum _{ j=1 }^{ n }{ { x }_{ j }{ y }_{ j } } \sum _{ j=1 }^{ n }{ { x }_{ j } }  }{ n\sum _{ j=1 }^{ n }{ { x }_{ j }^{ 2 } } -{ \left( \sum _{ j=1 }^{ n }{ { x }_{ j } }  \right)  }^{ 2 } }.\]
\medskip
\\
\textbf{Tabla de entrada 1.}
\\
\begin{figure 6}
\centering
\begin{tabular}{|c|c|c|c|c|}
\hline 
n&$\sum _{ i=1 }^{ n }{ { x }_{ i } } ( \°C )$ & $\sum _{ i=1 }^{ n }{ { y }_{ i } }(m)$ & $ \sum _{ i=1 }^{ n }{ { y }_{ i } } { x }_{ i }(m \°C)$ & $\sum _{ i=1 }^{ n }{ { x }_{ i }^{ 2 } }({ \°C }^{ 2 })$ \\ 
\hline 
9&$ 306 $& $ 4.47\times { 10 }^{ -3 }$ & $ 29.17137$ & $11904$\\ 
\hline 
\end{tabular}
\end{figure 6} 
\\
Sustituyendo los valores queda:

\[{ a }_{ 1 }=\frac { (9\times 29.17137\left[ m\cdot \°C \right] )-(306\times 4,47\times { 10 }^{ -3 }\left[ m\cdot \°C \right] ) }{ (11904\left[ {  \°C }^{ 2 } \right] )-{ \left( 306\left[ \°C \right]  \right)  }^{ 2 } } ={ 19.34626\times 10 }^{ -4 }\left[ { m }/{ \°C } \right] .\]

\[{ b }_{ 1 }=\frac { (11904\times 4.47\times { 10 }^{ -3 }\left[ m\cdot { \°C }^{ 2 } \right] )-(29.17137\times 306\left[ m\cdot { \°C }^{ 2 } \right] ) }{ (11904\left[ { \°C }^{ 2 } \right] )-{ \left( 306\left[ \°C \right]  \right)  }^{ 2 } } =-65.72\times { 10 }^{ -3 }\quad \left[ m \right] \]
Finalmente queda el modelos propuesto:
\[ Y_{1}= { 19.34626\times 10 }^{ -4 }x -65.72\times { 10 }^{ -3 }\quad \rightarrow(1) \]
Como la pendiente de la recta a tangente a la curva misma nos representa el coeficiente de expansi\'{o}n lineal, en el caso por su puesto para la gr\'{a}fica $\Delta T$t VS $\Delta L$, es decir:\\
$a_{1}= \frac { \Delta T }{ \Delta L } =L_{0} \alpha$, es decir $\frac{a_{1}}{L_{0}}= {\alpha}_{Al}$ (Coeficiente de expansi\'{o}n lineal). Y as\'{i}, podemos decir, que el coeficiente expansi\'{o}n lineal del material es: ${\alpha}_{Al}=2.5786\times 10 }^{ -5 }\left[{ \°C }^{ -1 } \right] .$\\
\\
\\
\pagebreak
\medskip
\medskip
\\
\textbf{Tabla de entrada 2.}
\\
\begin{figure 6}
\centering
\begin{tabular}{|c|c|c|c|c|}
\hline 
n&$\sum _{ j=1 }^{ n }{ { x }_{ j } } ( \°C )$ & $\sum _{ j=1 }^{ n }{ { y }_{ j } }(m)$ & $ \sum _{ j=1 }^{ n }{ { y }_{ j } } { x }_{ j }(m \cdot \°C)$ & $\sum _{ j=1 }^{ n }{ { x }_{ j }^{ 2 } }({ \°C }^{ 2 })$ \\ 

\hline 
10&$ 535 $& $3.54\times { 10 }^{ -3 }$ & $22.021\times { 10 }^{ -2 }$ & $ 30685 $\\ 
\hline 
\end{tabular}
\end{figure 6} 
\\
\medskip

Sustituyendo los valores queda:

\[{ { a }_{ 2 } }=\frac { (10\times 22.021\times { 10 }^{ -2 }\left[ m\cdot { \°C } \right] )-(535\times 3.53\times { 10 }^{ -3 }\left[ m\cdot { \°C } \right] ) }{ (30685\left[ { \°C }^{ 2 } \right] )-{ \left( 535\left[ \°C \right]  \right)  }^{ 2 } } ={ 1.5073\times 10 }^{ -5 }\left[ { m }/{ \°C } \right]. \]

\[{ { b }_{ 2 } }=\frac { (30685\times 3.54\times { 10 }^{ -3 }\left[ m\cdot { { \°C }^{ 2 } } \right] )-(535\times 22.02\times { 10 }^{ -2 }\left[ m\cdot { \°C } \right] ) }{ (30685\left[ { \°C }^{ 2 } \right] )-{ \left( 535\left[ \°C \right]  \right)  }^{ 2 } } =-{ 4.529\times 10 }^{ -4 }\left[ { m }/{ \°C } \right] .\]
Finalmente queda el modelos propuesto:
\[ y_{2}= { 1.5073\times 10 }^{ -5 }x -{ 4.529\times 10 }^{ -4 }  \quad \rightarrow(2) \]
Como la pendiente de la recta a tangente a la curva misma nos representa el coeficiente de expansi\'{o}n lineal, en el caso por su puesto para la gr\'{a}fica $\Delta T$t VS $\Delta L$, es decir:\\
$a_{1}= \frac { \Delta T }{ \Delta L } =L_{0} \alpha$, es decir $\frac{a_{2}}{L_{0}}= {\alpha}_{Cu}$ (Coeficiente de expansi\'{o}n lineal). Y as\'{i}, podemos decir, que el coeficiente expansi\'{o}n lineal del material es: ${\alpha}_{Cu}=1.9203\times 10 }^{ -5 }\left[{ \°C }^{ -1 } \right] .$.
\\
\section*{Ajuste por excel.}\\
Notemos que mientras por medio de c\'{a}alculos pudimos enconrar un modoelo, el programa Excel pudo enontrar otro, es cual se muestra a continuaci\'{o}n.
\\
\begin{figure}[hbtp]
 \centering
\includegraphics[width=8cm]{../../../../../../Pictures/equisteded1.jpg} 
\includegraphics[width=8cm]{../../../../../../Pictures/EXCEL2.jpg} 
\caption{Cuyas ecuaciones son $Yex_{1}={2.0767 \times { 10 }^{ -4 }}x-20.9400\times { 10 }^{ -5 }$ e $Yex_{2}={1.5073 \times { 10 }^{ -5 }}x-{45.2891 \times { 10 }^{ -5 }$}} 
\end{figure}
\\
Las cuales no difieren mucho de las calculadas anteriormente. Por la ecuaci\'{o}n (1) Y (2) anteriores se tienen que $\Delta {L}_{Al} ( T ) = { 19.34626\times 10 }^{ -4 }T -65.72\times { 10 }^{ -3 }$ y $\Delta {L}_{Cu} ( T ) = { 1.5073\times 10 }^{ -5 }T -{ 4.529\times 10 }^{ -4 }}$as\'{i} se logra obtener una relaciones entre las diferencias de temperatura y las diferencias de longitudes. Cabe resaltar que aunque el comportamiento de los puntos de dispersi\'{o}n era cmo se esperaba, es decir, claramente muestra un coportamiento lineal  y tambi\'{e}n de una funci\'{o}n estrictamente decreciente como se esperaba de acuerdo a nuestras hip\'{o}tesis, las diferencias de longitudes en una barra de Aluminio o Cobre crec\{i}an mientras aumentaba la temperatura, as\'{i} por medio de este an\'{a}lisis y nuestra teor\'{i}a mostrda podemos sacar los coeficientes de dilatac\'{o}n lineal de cada material los cuales quedan como:
\[{\alpha}_{Al}=2.5786\times 10 }^{ -5 }\left[{ \°C }^{ -1 } \right] \quad {\alpha}_{Cu}=1.9203\times 10 }^{ -5 }\left[{ \°C }^{ -1 } \right]\]
\\
\pagebreak

\section*{Error Porcentual.}\\
Los valores verdaderos de los coeficientes de dilataci\'{o}n lineal del Aluminio y del cobre  (los cuales se encuentran en la tabla 1 del marco te\'{o}rico) son de ${\alpha}_{Al}=2.24\times 10 }^{ -5 }\left[{ \°C }^{ -1 } \right]$ y ${\alpha}_{Cu}=1.67\times 10 }^{ -5 }\left[{ \°C }^{ -1 }  $ as\'{i} que podemos obtener el error porcentual, usando el rango de valores los cuales no cumple la condici\'{o}n:

\[{ E }rror\quad porcentual\quad { \alpha }_{ Al }=\frac { Error\quad verdadero }{ Valor\quad Verdadero } =\frac { Valor\quad verdadero - Valor\quad aproximado }{ Valor\quad verdadero } \times 100= 18.79\%. \]
\[{ E }rror\quad porcentual\quad { \alpha }_{ Cu }=\frac { Error\quad verdadero }{ Valor\quad Verdadero } =\frac { Valor\quad verdadero - Valor\quad aproximado }{ Valor\quad verdadero } \times 100= 24.43\%. \]
\\
 \section*{Discusiones.}\\
Al llevar a cabo el experimento, a pesar de su simplicidad encontramos con algunos inconvenientes con el arreglo, ya que en algunas ocasiones la manguera que conectaba el agua caliente con el dilat\'{o}metro se desconectaba y por ende derramaba aguan en todo nuestro exerimento, adem\'{a}s de que no pudimos llegar a los $100 \°C $ para poder alcanzar un valor m\'{a}s preciso al valor real, es decir encontrar un menor error porcentual. 


\end{document}