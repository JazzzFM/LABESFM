\documentclass[11pt,a4paper]{article}
\usepackage[latin1]{inputenc}
\usepackage[spanish]{babel}
\usepackage{amsmath}
\usepackage{amsfonts}
\usepackage{amssymb}
\usepackage{makeidx}
\usepackage{graphicx}
\usepackage{lmodern}
\usepackage[left=2cm,right=2cm,top=2cm,bottom=2cm]{geometry}
\author{Flores Rodguez Jaziel David }
\title{Práctica 1}
\begin{document}

\tableofcontents
%----------------------------------------------------------------------------------------
%	SECTION 1
%----------------------------------------------------------------------------------------
\\
\medskip
\medskip
\section{Resumen.}
\\
En esta pr\'{a}ctica lo que se busca es analizar el comportamiento de los conductores el\'{e}ctricos, a encontrar la relaci\'{o}n entre el voltaje aplicado al conductor, la corriente que circula por \'{e}l y la resistencia del mismo, a parte de determinar la resistividad de diferentes materiales. Tambi\'{e}n se observ\'{o} con el transcurso de la pr\'{a}ctica que la corriente es directamente proporcional al voltaje suministrado.
\\
\medskip
Palabras clave: conductores, resistencia, resistividad.

%----------------------------------------------------------------------------------------
%	SECTION 2
%----------------------------------------------------------------------------------------

\section{Objectivos.}\\

a) El alumno estudiar\'{a} el comportamiento de los conductores el\'{e}ctricos.\\
b) Encontrar\'{e} la relaci\'{o}n entre el voltaje aplicado al conductor, la corriente que circula por el y la resistencia del mismo.\\
c) Determinar\'{a} la resistividad de diferentes conductores. \\
\pagebreak


%----------------------------------------------------------------------------------------
%	SECTION 3
%----------------------------------------------------------------------------------------
\section{Metodolog\'{i}a.}
Un conductor el\'{e}ctrico es un cuerpo en el cual se puede establecer una corriente el\'{e}ctrica I, al colocar dos de sus puntos a una diferencia de potencial V.  
\begin{figure}[hbtp]
\centering
\includegraphics[width=9cm]{../../../../../../Pictures/JJDJDDHDJBJHa.jpg}
\end{figure}
\\
Existen conductores llamados \'{o}hmicos para los cuales se cumple que: 
\[V= IR\]
Donde R es una constante escalar. Adem\'{a}s se cumple que:
\[R= \rho \frac{L}{A}\]
Donde $\rho$ es una constante, cuyo valor depende de las propiedades materiales del conductor. Se denomina a $\rho$ la resistividad del material y a R la resistencia del conductor. 
%----------------------------------------------------------------------------------------
%	SECTION 4
%----------------------------------------------------------------------------------------

\section{Instrumentaci\'{o}n.}\\
1.-Fuente regulada 40 V. 10 A.\\
2.-Fuente regulada 25 V. 10 A.\\
3.-Mult\'{i}metro anal\'{o}gico.\\
4.-Mult\'{i}metro digital.\\
5.-Puente de impedancias.\\
6.-Tablero con un conductor.\\
7.-Tablero con tres conductores.\\
8.-Tablero con cinco conductores.\\
9.-Foco de 12 V.\\
10.-Conectores de diferente longitud.\\
\pagebreak

%----------------------------------------------------------------------------------------
%	SECTION 5
%----------------------------------------------------------------------------------------


\section{Datos y Resultados.}\\
\textbf{PARTE 1.}\\
Se conectaron los elementos como se indica en el siguiente diagrama. 
\\
\begin{figure}[hbtp]
\centering
\includegraphics[width=9cm]{../../../../../../Pictures/sasasasasasasasasasasasasasasasasa.jpg}
\end{figure}
\\
Por medio de la fuente se aplican voltajes de 0. 5 en 0. 5 V hasta 5.0 V a cada conductor y mide  la corriente correspondiente, se llen\'{o} la siguiente tabla de datos para cada conductor. Posteriormente se elebora una gr\'{a}fica de I vs V para cada conductor y se propuso una funci\'{o}n de ajuste I = I(V) que represente mejor los puntos experimentales, las cuales se muestran a continuaci\'{o}n.
\\
\begin{figure}[hbtp]
\centering
\includegraphics[width=7cm]{../../../../../../Pictures/ppppppppppppppppppppppp.jpg}
\end{figure}
\\
\textbf{Gr\'{a}fico de V VS ${I}_{1}$}
\\
\begin{figure}[hbtp]
\centering
\includegraphics[width=9cm]{../../../../../../Pictures/rururuurururururu.jpg}
\end{figure}
\\
\textbf{Ajueste de datos:} Con el ajuste obtenemos la siguiente ecuaci\'{o}n.
\[{I}_{1}(V)=3.949\times {10}^{-1} (1/ohm)V (Volt) + 0.022  \]
Con una desviaci\'{o}n est\'{a}ndar de $1.513\times {10}^{-1}$ (Amp) y un error E=+ 0.0183 (Amp). Donde 1/R = 0.0995 (1/$\Omega $) y R = 10.050 $\Omega$.
\\
\\
\textbf{Gr\'{a}fico de V VS ${I}_{2}$}
\\
\begin{figure}[hbtp]
\centering
\includegraphics[width=9cm]{../../../../../../Pictures/rirururururuururu.jpg} 
\end{figure}
\\
\textbf{Ajueste de datos:} Con el ajuste obtenemos la siguiente ecuaci\'{o}n.
\[{I}_{2}(V)=9.95\times {10}^{-2} (1/ohm)V (Volt) + 0.0183  \]
Con una desviaci\'{o}n est\'{a}ndar de $1.513 \times {10}^{-1}$ (Amp)  y un error E=+0.0183 (Amp). Donde 1/R = 0.0995 (1/$\Omega	$) y R = 10.050 $\Omega$
\\
\\
\textbf{Gr\'{a}fico de V VS ${I}_{3}$}
\\
\begin{figure}[hbtp]
\centering
\includegraphics[width=9cm]{../../../../../../Pictures/upupupupupupuppu.jpg} 
\end{figure}
\\
\textbf{Ajueste de datos:} Con el ajuste obtenemos la siguiente ecuaci\'{o}n.
\[{I}_{3}(V)=2.56\times {10}^{-1} (1/ohm)V (Volt) + 0.04  \]
Con una desviaci\'{o}n est\'{a}ndar de $3.88 \times {10}^{-1}$ (Amp)  y un error E=+0.04 (Amp). Donde 1/R = 0.256 (1/$\Omega	$) y R = 3.90625 $\Omega$
\\
\section*{Discusi\'{o}n.}\\
Podemos concluir que los 4 alambres primeramente analizados, cumple efectivamente con la ley de Ohm y les podemos llamar ohmicos, debido a que la relaci\'{o}n entre V vs I es lineal.
\\
\pagebreak


\textbf{PARTE 2.}
\\
Con el tornillo microm\'{e}trico midi\'{o} el di\'{a}metro de cada uno de los cinco conductores del tablero y posteriormente se determina sus \'{a}reas de secci\'{o}n transversal. Con el puente de impedancias se midi\'{o} la resistencia de cada conductor para las longitudes siguientes, 1.0, 0.75, 0.50, .0.25 m. Luego se llen\'{o} la siguiente tabla de datos. Con los datos de la tabla se calcula la resistividad de cada conductor y para despu\'{e}s consultar en un manual adecuado.
\\
\begin{figure}[hbtp]
\centering
\includegraphics[width=12cm]{../../../../../../Pictures/pssspsppspspspspspspsasasasasassssssss.jpg}
\end{figure}
\\
\section*{Discusi\'{o}n.}\\
Los coeficientes encontrados para los materiales anotados en la tabla dos se asemejan  algunos encontrados en tablas de libros, hay que especificar que no son los exactos nuestros resultados por el equipo que se tiene, teniendo errores del 0.5 al 10$\%$ aproximadamente de error.
\\
\pagebreak
\\
\textbf{PARTE 3.}
\\
Del tablero que contiene el conductor de 10m. de longitud, se midi\'{o} su di\'{a}metro y se obtuvo su \'{a}rea de secci\'{o}n transversal. Utilizando el puente de impedancias, se mide la resistencia del conductor para las diferentes longitudes que se indican en la tabla de acontinuaci\'{o}n. Y se llena la siguiente tabla de datos.
\\
\begin{figure}[hbtp]
 \centering
 \includegraphics[width=7cm]{../../../../../../Pictures/DDSDSDasdasfFASFASFAS.jpg}
 \end{figure}
\\
\textbf{Gr\'{a}fico de Resistencia vs Longitud.}
\begin{figure}[hbtp]
\centering
\includegraphics[width=9cm]{../../../../../../Pictures/papasasjdhaskjdashdakjsdhas.jpg}
\end{figure}
\\
\textbf{Ajuste de datos} Con la ecuación del ajuste:  
\[R (m) = 3.2585 (\Omega/m)L(m) + 0.012 (\Omega)\]
Con una desviaci\'{o}n estándar de 9.866 ($\Omega$) y un Error de E = +  0.012 (ohm). Donde $\rho$/A = 3.2585 ($\Omega$/m), A= 4.312$\times {10}^{-7}$ y $\rho$=  1.405$\times{10}^{-6}$($\Omega$m)
\section*{Discusi\'{o}n.}\\
El coeficiente de resistividad para el alambre de 10 m fue de $1.405\times {10}^{-6} (\Omega)$m, su relaci\'{o}n tambi\'{e}n fue lineal y se considera por lo tanto ohmico, con una variaci\'{o}n de resistencia constante.
\\
\pagebreak
\\
\textbf{PARTE 4.}
\\
Se conecta el arreglo experimental como se indica en la siguiente figura. Por medio de la fuente de c.d. se aplica los voltajes que se indican en la tabla siguiente y medimos la corriente para cada uno de ellos, completamos la tabla de datos.
\\
\begin{figure}[hbtp]
 \centering
 \includegraphics[width=4cm]{../../../../../../Pictures/VADADADVADVADVADSA.jpg}
 \includegraphics[width=9cm]{../../../../../../Pictures/figurerfdfs.jpg}
 \end{figure}
\\
\textbf{Gr\'{a}fico de I vs V en foco}
\begin{figure}[hbtp]
\centering
\includegraphics[width=9cm]{../../../../../../Pictures/xxxxxxxxxxxxxxxxxxxxxxxxxxxxxxxxxxxxx.jpg}
\end{figure}\\
\textbf{Ajuste de datos} Con la ecuaci\'{o}n del ajuste: 
\[I(V) = 0.0522(1/\Omega)V + 0.1998 (Ampere).\]
Con una desviaci\'{o}n est\'{a}ndar de : $1.76\times {10}^{-1}$ (Ampere) y un error de E = + 0.1998 (Ampere). Donde 1/R = 0.0522 (1/$\Omega$) y R = 19.157088812 $\Omega$
\section*{Discusi\'{o}n.}\\
Finalmente, el experimento del foco nos denot\'{o} otra relaci\'{o}n lineal a pesar de la energ\'{i}a perdida por el foco. Como en los casos anteriores, todas las relaciones fueron lineales y no hubo necesidad de tomar en cuenta los datos con un cambio de variable, pues en el \'{u}ltimo caso este era m\'{a}s impreciso.
%----------------------------------------------------------------------------------------
%	SECTION 6
%---------------------------------------------------------------------------------------

\section{Conclusiones.}
Se comprendi\'{o} el concepto de resistencia y resitividad, adem\'{a}s se encontr\'{o} de manera pr\'{a}ctica cual era la relaci\'{o}n de dependencia de ellas, encontramos tambi\'{e}n una relaci\'{o}n entre la corriente, diferencia de poencial y la resistencia. Despu\'{e}s encontramos una dependencia entre la variables f\'{i}sicas como el \'{a}rea transversal de un alambre y la longitud del mismo, y finalmente se entendi\'{o} el concepto de material conductor el\'{e}ctrico. 
%----------------------------------------------------------------------------------------
%	SECTION 9
%----------------------------------------------------------------------------------------
 \section{Bibliograf\'{i}a.}
1.-https://es.wikipedia.org/wiki/Conductor_elC3A9ctrico\\
2.-https://unicrom.com/codigo-de-colores-de-las-resistencias/.\\
3.-https://sites.google.com/site/labenriquesalgadoruiz/home/politecnico-1/fisica-iii .\\
4.-Resnick/Halliday/Krane. Fundamentos de F\'{i}sica. Volumen 2. Edici\'{o}n 6, extendida. CESA\\

%----------------------------------------------------------------------------------------
%	SECTION 10
%----------------------------------------------------------------------------------------

\end{document}