\documentclass[10pt,a4paper]{article}
\usepackage[latin1]{inputenc}
\usepackage[spanish]{babel}
\usepackage{amsmath}
\usepackage{amsfonts}
\usepackage{amssymb}
\usepackage{graphicx}
\usepackage[left=2cm,right=2cm,top=2cm,bottom=2cm]{geometry}
\begin{document}
\section*{Desarrollo Experimental.}\\

 \\
\textbf{Lista de Materiales:} \\
Soporte con brazo de palanca que se mantiene ene equilibrio con pesas.\\
Pesas (jinetitos)de distintos masas.\\
Aceite de motor.\\
Tubo de probeta graduada.\\
Cilindro met\'{a}lico con vol\'{u}men const\'{a}nte.\\
Medidor \'{o}ptico con visor inverso.  \\
\\
\\
\section*{Arreglo experimenal. Parte A).} \\

\begin{figure}[hbtp]
\centering
\includegraphics[width=15cm]{../../../../../../Pictures/Arreglogeneral4.jpg}  
\caption{Arreglo general}
\end{figure}
\\
\\
\begin{figure}[hbtp]
\centering
\includegraphics[width=12cm]{../../../../../../Pictures/UUu.jpg} 
\caption{Medidor \'{o}ptico de nivel y jinetitos.}
\end{figure}

\pagebreak

\textbf{Procedimiento.}\\
\\
1.-Introducir un poco el ciclinro en el fuido.\\
2.-Equilibrar con los jinetitos.\\
3.-Calcular B \\
4. Medir el desplazamiento h y calcular v.\\
5.-Repetr de 10 o 12 veces.\\
\\
\\
\textbf{Resultados.}\\
\\
Comenzamos la pr\'{a}ctica con ell material proporcionado por el equipo de laboratorio, procedimos a medir algunas de nuestras constantes como lo son; la masa del cilindro inicial, el radio interior de la probeta, con el rado anteiror encontrar el \'{a}rea interior de la probeta. \\
\medskip
\\
\medskip
\caption{Tabla 1.} 
\\ 
\begin{tabular}{|c|c|c|c|c|}
\hline 
Radio interior(m) & \'{A}rea (${ m }^{ 2 }$) & Mo (kg)\\ 
\hline 
${ 13.5\times 10 }^{ -3 }$ & $5.73\times 10 }^{ -4 }$ & ${ 36.066\times 10 }^{ -3 }$ \\
\hline 
\end{tabular}\\

\medskip
Llenamos la Tabla 1 proporcionada para despu\'{e}s graficar el vol\'{u}men contra B (torque) del instrumento para despu\'{e}s realizar el respectivo ajuste y encontrar la densidad..\\
\\
\textbf{Tabla 2.}\\
\\
\begin{figure 2}
\centering
\includegraphics[width=7cm]{../../../../../../Pictures/DENSIDAD2.jpg}  
\\
\end{figure 2}
\\
De la cual extragimos los datos de la siguiente tabla para poder hacer la gr\'{a}fica V vs B.
\\
\textbf{Tabla 3.}\\
\\
\begin{figure 3}
\centering
\includegraphics[width=4cm]{../../../../../../Pictures/densidad3.jpg} 
\\
\end{figure 3}
\\
Notemos que hay un valor (el cual se marca de amarillo) el cual se encuentra comletamente fuera del rango esperado, en la secci\'{o}n posterior veremos si este punto se encuentra muy fuera de un rango, si ese es el caso, tal punto se eliminar\'{a}.
\\
\section*{Gr\'{a}fica de dispersi\'{o}n}
\\
\\
\begin{figure 5}
\centering
\includegraphics[width=10cm]{../../../../../../Pictures/densidad6.jpg} 
\\
\end{figure 5}
\\
\\
Cabe reslatar que , como se muestra en la figura, el punto marcado en un c\'{i}rculo rojo est\'{a} completamente fuera de un rango que se esperar\'{i}a como se muestra en los dem\'{a}s puntos, esto justifica que a ese punto se omitir\'{a} y se tomar\'{a} como un error de medici\'{o}n.  
\\
\section*{Ajuste de datos.}\\
Por el Apr\'{e}ndice 1 podemos hacer el respectivo ajuste por el m\'{e}todo de m\'{i}nimos cuadrados para encontrar un modelo lineal $Y=ax +b$ tales que $\left( { x }_{ i },{ y }_{ i } \right) \rightarrow \left( V ({ m }^{ 3 }), B (Nm)\right) $ para cada uno de los datos de cada exprimento se tendr\'{a} su tabla de entrada y los siguientes coeficientes a y b, de donde:
\[a=\frac { n\sum _{ i=1 }^{ n }{ { x }_{ i }{ y }_{ i } } -\sum _{ i=1 }^{ n }{ { x }_{ i } } \sum _{ i=1 }^{ n }{ { y }_{ i } }  }{ n\sum _{ i=1 }^{ n }{ { x }_{ i }^{ 2 } } -{ \left( \sum _{ i=1 }^{ n }{ { x }_{ i } }  \right)  }^{ 2 } } \quad y\quad b=\frac { \sum _{ i=1 }^{ n }{ { x }_{ i }^{ 2 } } \sum _{ i=1 }^{ n }{ { y }_{ i } } -\sum _{ i=1 }^{ n }{ { x }_{ i }{ y }_{ i } } \sum _{ i=1 }^{ n }{ { x }_{ i } }  }{ n\sum _{ i=1 }^{ n }{ { x }_{ i }^{ 2 } } -{ \left( \sum _{ i=1 }^{ n }{ { x }_{ i } }  \right)  }^{ 2 } }.\]
\medskip
\\
\\Tabla de entrada 1.
\\
\\
\begin{figure 6}
\centering
\begin{tabular}{|c|c|c|c|c|}
\hline 
 n & $\sum _{ i=1 }^{ n }{ { x }_{ i } } ({ m }^{ 3 }) $ & $\sum _{ i=1 }^{ n }{ { y }_{ i } } (Nm)$ & $ \sum _{ i=1 }^{ n }{ { y }_{ i } } { x }_{ i } \left( Nm \right) \left({ m }^{ 3 }\right)$ & $\sum _{ i=1 }^{ n }{ { x }_{ i }^{ 2 } }{ \left( m  \right)  }^{ 6 }$ \\ 

\hline 
9 & $4.19026\times { 10 }^{ -4 }$& $3.20331472$ & $1.49156\times { 10 }^{ -4 }$ & $1.95226\times { 10 }^{ -8 }$ \\ 
\hline 
\end{tabular}
\end{figure 6} 
\\
\medskip
Sustituyendo los valores queda:
\[a=\frac { (9\times 1.49156\times { 10 }^{ -4 }\left[ Nm \right] \left[ { m }^{ 3 } \right] )-(4.19026\times { 10 }^{ -4 }\times 3.20331472\left[ Nm \right] \left[ { m }^{ 3 } \right] ) }{ (9\times 1.95226\times { 10 }^{ -8 }\left[ { m }^{ 6 } \right] )-{ \left( 4.19026\times { 10 }^{ -4 }{ m }^{ 3 } \right)  }^{ 2 } } =1116.827749\frac { \left[ Nm \right]  }{ \left[ { m }^{ 3 } \right]  }.\]

\[b=\frac { (1.95226\times { 10 }^{ -8 }\times 3.20331472\left[ { Nm } \right] \left[ { m }^{ 6 } \right] )-(1.49156\times { 10 }^{ -4 }\times 4.19026\times { 10 }^{ -4 }\left[ { Nm } \right] \left[ { m }^{ 6 } \right] ) }{ (9\times 1.95226\times { 10 }^{ -8 }\left[ { m }^{ 6 } \right] )-{ \left( 4.19026\times { 10 }^{ -4 }{ m }^{ 3 } \right)  }^{ 2 } } ={ 30.3926151\times 10 }^{ -2 }\left[ Nm \right] .\]
Finalmente queda el modelos propuesto:
\[ Y = 30.3926151\times 10 }^{ -2 } + 1116.827749x\quad \rightarrow(1) \]
\\
Como la pendiente de la recta a tangente a la curva misma nos representa el la densidad multiplicada por la aceleraci\'{o}n de la gravedad , es decir $B= {\rho}_{0}Vg$, por lo tanto se tendr\'{a} la siguiente relaci\'{o}n para la densidad:
\[{\rho}_{0} =\frac {B }{ Vg }.\]
Y as\'{i}, podemos decir, que la densidad del aceite es: ${\rho}_{0}=617.0319058 \left[ Kg/{ m }^{ 3 } \right] .$
\pagebreak
\section*{Error Porcentual.}\\
Los valores verdaderos (Tabla 2) de la densidad del aceite SAE 40 $\rho = 889 \left[ Kg/{ m }^{ 3 } \right] $. Entonces, de nuestras mediciones y c\'{a}lculos podemos obtener el error porcentual:

\[{ E }rror \quad porcentual \quad \rho =\frac { Error\quad verdadero }{ Valor\quad Verdadero } =\frac { Valor\quad verdadero - Valor\quad aproximado }{ Valor\quad verdadero } \times 100= 28.34 \%.\]
\\
\section*{Ajuste por excel.}\\
Notemos que mientras por medio de c\'{a}alculos pudimos enconrar un modoelo, el programa Excel pudo enontrar otro, es cual se muestra a continuaci\'{o}n.

\begin{figure}[hbtp]
 \centering
\includegraphics[width=11cm]{../../../../../../Pictures/DENSIDA0.jpg} 
 \caption{Modelo por medio de excel. }  
\end{figure}
\\
Y cuya ecuaci\'{o}n es $Y = 1116.8x + 0.3039$.
 \pagebreak


\section*{Parte B): Densidad en funci\'{o}n de la temperatura.} \\

\begin{figure}[hbtp]
\centering
\includegraphics[width=14cm]{../../../../../../Pictures/DENSIDADTEMPP.jpg}    
\caption{Arreglo general.}
\end{figure}
\\
\\
\pagebreak


\textbf{Procedimiento.}\\
\\
1.-Medir Lo a temperatura ambiente.\\
2.-Prender el mechero y esperar a $T=30\°C$.\\
3. Repetir el paso anterior, de 12 a 25 veces.\\
4.-Repetir 10 0 12 veces cada $5\°C$.\\

\textbf{Resultados.}\\
\\
Comenzamos la pr\'{a}ctica con ell material proporcionado por el equipo de laboratorio, procedimos a medir algunas de nuestras constantes como lo fueron el nivel inicial del fluido (lo=0.35m) y la densidad (${\rho}_{0}=617.0319058 \left[ Kg/{ m }^{ 3 } $)de la Parte A). \\
\medskip
Llenamos la Tabla 2.B proporcionada para despu\'{e}s graficar la densidad con respecto a la temperatura.\\
\textbf{Tabla 2.B}\\
\\
\begin{figure 2}
\centering
\includegraphics[width=12cm]{../../../../../../Pictures/PARTTEB.jpg} 
\\
\caption{De la cual extragimos los datos de la siguiente tabla para poder hacer la gr\'{a}fica $\rho$ VS T.}
\end{figure 2}

\begin{figure 3}
\centering
\caption{\textbf{Tabla 3.}}
\\
\includegraphics[width=5cm]{../../../../../../Pictures/PARTEBBB.jpg} 
\\
\caption{Datos extra\'{i}dos.}
\end{figure 3}


\\
\section*{Gr\'{a}fica de dispersi\'{o}n}
\\
\\
\begin{figure 5}
\centering
\includegraphics[width=9cm]{../../../../../../Pictures/PARTEBBBBBBBBB.jpg}  
\\ 
\caption{Gr\'{a}fico de dispersi\'o}n para el m\'{e}todo de nivel.}
\end{figure 5}  
\section*{Ajuste de datos.}\\
Por el Apr\'{e}ndice 1 podemos hacer el respectivo ajuste por el m\'{e}todo de m\'{i}nimos cuadrados para encontrar un modelo lineal $Y=ax +b$ tales que $\left( { x }_{ i },{ y }_{ i } \right) \rightarrow \left( T(\°C), \rho (kg/{m}^3) \right) $  para cada uno de los datos de cada exprimento se tendr\'{a} su tabla de entrada y los siguientes coeficientes a y b, de donde:
\\
\medskip
\[a=\frac { n\sum _{ i=1 }^{ n }{ { x }_{ i }{ y }_{ i } } -\sum _{ i=1 }^{ n }{ { x }_{ i } } \sum _{ i=1 }^{ n }{ { y }_{ i } }  }{ n\sum _{ i=1 }^{ n }{ { x }_{ i }^{ 2 } } -{ \left( \sum _{ i=1 }^{ n }{ { x }_{ i } }  \right)  }^{ 2 } } \quad y\quad b=\frac { \sum _{ i=1 }^{ n }{ { x }_{ i }^{ 2 } } \sum _{ i=1 }^{ n }{ { y }_{ i } } -\sum _{ i=1 }^{ n }{ { x }_{ i }{ y }_{ i } } \sum _{ i=1 }^{ n }{ { x }_{ i } }  }{ n\sum _{ i=1 }^{ n }{ { x }_{ i }^{ 2 } } -{ \left( \sum _{ i=1 }^{ n }{ { x }_{ i } }  \right)  }^{ 2 } }.\]
\medskip
\\
\caption{Tabla de entrada 2.}
\\
\begin{figure 6}
\\
\medskip  
\medskip 
\centering
\begin{tabular}{|c|c|c|c|c|}
\hline 
 n & $\sum _{ i=1 }^{ n }{ { x }_{ i } } (\°C)$ & $\sum _{ i=1 }^{ n }{ { y }_{ i } } (kg/{m}^3) $ & $ \sum _{ i=1 }^{ n }{ { y }_{ i } } { x }_{ i }(\°C)(kg/{m}^3) $  & $\sum _{ i=1 }^{ n }{ { x }_{ i }^{ 2 } }\left( \°C \right)  }^{ 2 }$ \\ 
\hline 
11 & $522$& $6347.581563$ & 300779.4378 & 2332832\\ 
\hline 
\end{tabular} 
\end{figure 6}
\\

\end{figure 6} 
\\
\medskip
Sustituyendo los valores queda:

\[a=\frac { (11\times 300779.4378\left[ \°C \right] \left[ { kg }/{ { m }^{ 3 } } \right] )-(522\times 6347.581563\left[ \°C \right] \left[ { kg }/{ { m }^{ 3 } } \right] ) }{ (11\times 2332832\left[ \°C^{ 2 } \right] )-{ \left( 522\left[ \°C \right]  \right)  }^{ 2 } } =-0.6274\frac { \left[ { kg }/{ { m }^{ 3 } } \right]  }{ \left[ \°C \right]  }.\]

\[b=\frac { (2332832\times 6347.581563\left[ \°C^{ 2 } \right] \left[ { kg }/{ { m }^{ 3 } } \right] )-(300779.4378\times 522\left[ \°C^{ 2 } \right] \left[ { kg }/{ { m }^{ 3 } } \right] ) }{ (11\times 2332832\left[ \°C^{ 2 } \right] )-{ \left( 522\left[ \°C \right]  \right)  }^{ 2 } } =606.83\left[ { kg }/{ { m }^{ 3 } } \right].\]
Finalmente queda el modelos propuesto:
\[Y = 577.063 + -0.6274x\quad \rightarrow(2) \]
Es decir encontramos una relaci\'{o}n entre la temperatura y la densidad de un l\'{i}quido, m\'{a}s expl\'{i}citamente se encontr\'{o} $\rho=\rho(T)=606.83 + -0.6274(T) $
\\
\section*{Error Porcentual.}\\
El valor verdadero de la densidad de un aceite SAE 40 a $20\°C$ es de $\rho=889 \left[ Kg/{ m }^{ 3 }$. Entonces, de nuestras mediciones y c\'{a}lculos podemos obtener el error porcentual:

\[{ E }rror-porcentual-{ Y }_{ 1 }=\frac { Error\quad verdadero }{ Valor\quad Verdadero } =\frac { Valor\quad verdadero - Valor\quad aproximado }{ Valor\quad verdadero } \times 100=36.74 \% \]
\\
\pagebreak

\section*{Ajuste por excel.}\\
Notemos que mientras por medio de c\'{a}alculos pudimos enconrar un modelo, el programa Excel pudo enontrar otro, es cual se muestra a continuaci\'{o}n.
\\
\begin{figure}[hbtp]
 \centering
\includegraphics[width=10cm]{../../../../../../Pictures/PARTEB.jpg} 
 \caption{Modelo por medio de excel. }
\end{figure} 
 Cuya ecuaci\'{o}n es:
 \[{ Y }_{ ex}= -0.191572x +577.0619603. \quad \rightarrow(2.1) \]
 \\
 \section*{Discusiones.}\\
 N\'{o}tese que aunque los resultados no fueron muy precisos (de hecho muy imprecisos), se encontr\'{o} un comportamiento acorde a la temperatura deseada, es decir se encontr\'{o} que la funci\'{o}n $\rho(T)$ es decreciente, as\'{i} que mientras la temperatura sube la densidad baja. Adem\'{a}s que elapartao (Balanza) ten\'{i}a bastante sensibiidad, cosa que no tomamos en cuenta al hacer las mediciones.
 

\end{document}