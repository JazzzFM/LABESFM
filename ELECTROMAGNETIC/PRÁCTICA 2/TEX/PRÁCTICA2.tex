\documentclass[11pt,a4paper]{article}
\usepackage[latin1]{inputenc}
\usepackage[spanish]{babel}
\usepackage{amsmath}
\usepackage{amsfonts}
\usepackage{amssymb}
\usepackage{makeidx}
\usepackage{graphicx}
\usepackage{lmodern}
\usepackage[left=2cm,right=2cm,top=2cm,bottom=2cm]{geometry}
\author{Flores Rodguez Jaziel David }
\title{Práctica 1}
\begin{document}

\tableofcontents
%----------------------------------------------------------------------------------------
%	SECTION 1
%----------------------------------------------------------------------------------------
\\
\medskip
\medskip
\section{Resumen.}
\
Identificar los tipos de electrizaci\'{o}n de cuerpos es de suma importancia en el estudio de la electrost\'{a}tica y los cuales el hombre ten\'{i}a conciencia desde la aniguedad, si bien la palabra electr\'{o}n es de origen griego que significa \'{a}mbar y esta es una piedra las cuales produc\'{i}an ciertos fen\'{o}menos que ahora consideramos de ende electrost\'{a}tico, es decir existen objetos los cuales producen ciertos fen\'{o}menos f\'{i}sicos los cuales son medibles. Es por ello que en esta pr\'{a}ctica hemos cuantificado la fuerza entre cargas para las cuales existen de dos tipos y Benjamin Franklin adopt\'{o} el convenio de llamarlas como positivas y negativas, a esta fuerza entre dos cargas le llamaremos Ley de Coulomb. Adem\'{a}s clasificar las diferentes formas cualitativas de que cuerpo aquieran carga.
\\
\\
\medskip
Palabras clave: Electrizaci\'{o}n, Carga.

%----------------------------------------------------------------------------------------
%	SECTION 2
%----------------------------------------------------------------------------------------

\section{Objectivo.}
a)Los diferetes tipos de elcriaci\'{o}n de un cuerpo y podr\'{a} proponer la hip\'{o}tesis que justifique la interacci\'{o}n observada entre sus cuerpos cargados.
\\
\\
b)La ley de Coulomb, utilizando para esto, en su atividad la balanza de torsi\'{o}n y el amplificador lineal.

\pagebreak


%----------------------------------------------------------------------------------------
%	SECTION 3
%----------------------------------------------------------------------------------------
\section{Metodolog\'{i}a.}

En f\'{i}sica, se denomina electrizaci\'{o}n al efecto de ganar o perder cargas el\'{e}ctricas, normalmente electrones, producido por un cuerpo el\'{e}ctricamente neutro. Existen tres formas de electrizar un cuerpo: electrizaci\'{o}n por frotaci\'{o}n, contacto e inducci\'{o}n. En estos procedimientos siempre est\'{a} presente el principio de conservaci\'{o}n de la carga y la regla fundamental de la electrost\'{a}tica.
\\
\\
\textbf{Frotaci\'{o}n}:\\
En la electrizaci\'{o}n por frotaci\'{o}n, el cuerpo menos conductor retira electrones de las capas exteriores de los \'{a}tomos del otro cuerpo, quedando cargado de forma negativa, y el que libera electrones queda cargado de forma positiva. Al frotar dos cuerpos el\'{e}ctricamente neutros (n\'{u}mero de electrones igual al n\'{u}mero de protones), ambos se cargan, uno con carga positiva y el otro con carga negativa. Si se frota una barra de vidrio con un pa\~{n}o de seda, hay un traspaso de electrones del vidrio a la seda. Si se frota un l\'{a}piz de pasta con un pa\~{n}o de lana, hay un traspaso de electrones del pa\~{n}o al l\'{a}piz. El vidrio adquiere una carga el\'{e}ctrica positiva al perder un determinado n\'{u}mero de cargas negativas (electrones); estas cargas negativas son atra\'{i}das por la seda, con lo cual se satura de cargas negativas. Al quedar cargados el\'{e}ctricamente ambos cuerpos, ejercen una influencia el\'{e}ctrica en una zona determinada, que depende de la cantidad de carga ganada o perdida, dicha zona se llama campo el\'{e}ctrico, una explicaci\'{o}n sobre los materiales y c\'{o}mo se cargan puede hallarse en el efecto triboel\'{e}ctrico.
\\
\\
\textbf{Contacto}:\\
En la electrizaci\'{o}n por contacto el cuerpo conductor es puesto en contacto con otro cuya carga no es nula. Aquel cuerpo que presente un exceso relativo de electrones los transferir\'{a} al otro. Al finalizar la transferencia los dos cuerpos quedan con carga de igual signo.
\\
\\
\textbf{Inducci\'{o}n}:\\
Cuando un cuerpo cargado se acerca a uno descargado sin llegar a tocarlo, las cargas en este \'{u}ltimo se reagrupan en dos regiones distintas del mismo, debido a que los electrones del cuerpo descargado son atra\'{i}dos o repelidos a uno de los extremos seg\'{u}n sea el caso, al alejarse nuevamente el cuerpo cargado desaparece ese reagrupamiento de cargas. A esa separaci\'{o}n de cargas dentro de un objeto el\'{e}ctricamente neutro se le denomina polarizaci\'{o}n.
\\
\\
\textbf{Ley de Coulomb.}\\
Charles-Augustin de Coulomb desarroll\'{o} la balanza de torsi\'{o}n con la que determin\'{o} las propiedades de la fuerza electrost\'{a}tica. Este instrumento consiste en una barra que cuelga de una fibra capaz de torcerse. Si la barra gira, la fibra tiende a hacerla regresar a su posici\'{o}n original, con lo que conociendo la fuerza de torsi\'{o}n que la fibra ejerce sobre la barra, se puede determinar la fuerza ejercida en un punto de la barra. La ley de Coulomb tambi\'{e}n conocida como ley de cargas tiene que ver con las cargas el\'{e}ctricas de un material, es decir, depende de si sus cargas son negativas o positivas. En la barra de la balanza, Coulomb coloc\'{o} una peque\~{n}a esfera cargada y a continuaci\'{o}n, a diferentes distancias, posicion\'{o} otra esfera tambi\'{e}n cargada. Luego midi\'{o} la fuerza entre ellas observando el \'{a}ngulo que giraba la barra. Dichas mediciones permitieron determinar que:
\\
\\
La fuerza de interacci\'{o}n entre dos cargas $q_{1}$ y $q_{2}$ duplica su magnitud si alguna de las cargas dobla su valor, la triplica si alguna de las cargas aumenta su valor en un factor de tres, y as\'{i} sucesivamente. Concluy\'{o} entonces que el valor de la fuerza era proporcional al producto de las cargas:
\[F\propto { q }_{ 1 }\quad y\quad F\propto { q }_{ 2 }\]
En consecuencia: $F\propto { q }_{ 1 }{ q }_{ 2 }$.\\
Si la distancia entre las cargas es \textbf{r}, al duplicarla, la fuerza de interacci\'{o}n disminuye en un factor de $4\quad ({ 2 }^{ 2 })$; al triplicarla, disminuye en un factor de $9\quad ({ 3 }^{ 2 })$ y al cuadriplicar \textbf{r}, la fuerza entre cargas disminuye en un factor de $16\quad ({ 4 }^{ 2 })$. En consecuencia, la fuerza de interacci\'{o}n entre dos cargas puntuales, es inversamente proporcional al cuadrado de la distancia:
\[F\propto \frac { 1 }{ { r }^{ 2 } } \]
Asociando ambas relaciones:
\[F\propto \frac { { q }_{ 1 }{ q }_{ 2 } }{ { r }^{ 2 } } \]
Finalmente, se introduce una constante de proporcionalidad para transformar la relaci\'{o}n anterior en una igualdad:
\[F=\frac { 1 }{ 4\pi { \epsilon  }_{ 0 } } \left( \frac { { q }_{ 1 }{ q }_{ 2 } }{ { r }^{ 2 } }  \right) \]
\\
\begin{figure}
\centering
\includegraphics[width=8cm]{../../../../../../Pictures/Cargas_atrac-rep.jpg} 
\end{figure}
\\
Donde para el sistema internacional de unidades:\\
1.- ${ \epsilon  }_{ 0 } }= 8,8541878176\times 1{ 0 }^{ -12 }\quad \left[ { { C }^{ 2 } }/{ N{ m }^{ 2 } } \right]$\\
2.- ${ q }_{ 1 }$ y ${ q }_{ 2 } }$ son ls cargas en Coulombs.\\
3.- \textbf{r} es la distancia que separa a las cargas en metros.\\
4.- \textbf{F} es la fuerza de atracci\'{o}n o repulsi\'{o}n entre las cargas, en Newtons.\\ 
\pagebreak

%----------------------------------------------------------------------------------------
%	SECTION 4
%----------------------------------------------------------------------------------------

\section{Instrumentaci\'{o}n.}\\
1- Balanza de torsi\'{o}ncon alambres de un material en espec\'{i}fico y una barilla uniforme.\\
2- Amplificador lineal.\\
3- Generador electrost\'{a}tico. \\
4- Mult\'{i}metro. \\
5. Portadores de carga (esferas met\'{a}licas).\\
6. Flex\'{o}metro.\\
7. Nivel de burbuja.\\
8. L\'{a}mpara de iluminaci\'{o}n.\\
9. Tripie.\\
10. Barra de pl\'{a}stico.\\
11. Barra de vidrio.\\
12. Electroscopio de wulf.\\
13. Jaula de Faraday.\\
14. Suspensi\'{o}n para barras.\\
%----------------------------------------------------------------------------------------
%	SECTION 5
%----------------------------------------------------------------------------------------


\section{Datos y Resultados.}\\

\textbf{(a)Electrizaci\'{o}n:}\\
Al comenzar la pr\'{a}ctica nos proporcionaron el material no conductor como una barra de vidrio y otra de un material rojo, adem\'{a}s de una " piel de gato'' y franela. Primero vimos electrizaci\'{o}n por frotamiento (Observe las figuras):

\begin{figure}[hbtp]
\centering 
\includegraphics[width=5.5cm]{../../../../../../Pictures/IMG_20170905_124036704.jpg}
\caption{Aqu\'{i} se observa que al frotar la barra del mismo tipo con la franela \'{e}stas se repelen, sin embargo cuando la frotaba con la piel de gato no pasaba nada, esto quiere decir que existe un tipo de caracter\'{i}stica en el medio que hace que dos barras del mismo material se repelan.}
\end{figure}
\\
\begin{figure}[hbtp]
\centering
\includegraphics[width=5.5cm]{../../../../../../Pictures/IMG_20170905_125911929.jpg}
\caption{Se observa ahora que a frotar la barrra de vidrio con la piel de gato se observ\'{o} un fen\'{o}meno de atracci\'{o}n entre las barras, sin embargo cuando franela no pasaba nada, esto quiere decir que existe algun tipo de caracter\'{i}tica distinta a la del experimento anterior que hace que las dos barras distintas se atraigan. Ahora bien, usado el electroscopio y cargando la barra de vidrio la colocamos en el platillo y se observa lo siguiente:
}
\end{figure}
\\
\begin{figure}[hbtp]
\centering
\includegraphics[width=5.5cm]{../../../../../../Pictures/IMG_20170905_124512652.jpg}
\caption{Se observa que la lamina dentro del electroscopio se separa de como estaba al inicio, es decir las laminas y su origen tienen la misma caracter\'{i}stica el\'{e}ctrica y por ello se repelen. Esto ocurre porqe la barra en el platillo induce una caracter\'{i}stica el\'{e}ctrica a todo el material met\'{a}lico, cabe resaltar que nunca tocamos el platillo y \'{e}ste sin embargo sufr\'{i}a de un fen\'{o}meno ele\'{e}ctrico a distancia, y que al tocarlo completamente se volv\'{i}a a poner como en sus condiciones iniciales.}
\end{figure}
\\
\pagebreak

\section*{(a)LEY DE COULOMB:}\\
\textbf{Procedimiento 1.}Con ayuda de la balanzade torsi\'{o}n determine la fuerza que se ejercen entre si dos cargas de valor constante, para diferentes distancias de separaci\'{o}n entre ellas, obtener los siguientes datos:\\
1) Constante de torsi\'{o}n de la balanza, $k=3.04\times { 10 }^{ -4 }\left[ Nm \right] $\\
2) Distancia del espejo al pizarr\'{o}n. R=3.25 m.\\
3) Brazo de palanca. r= 6cm.\\
4) Distancia de separaci\'{o}n entre las cargas.(variable)\\
Conecte el equipo como se indica en la figura 1 y separe los portadores de carga 10 cm de centro a centro de las esferas, con ayuda de la m\'{a}quina electrost\'{a}tica cargue cada una de las esferas y mida la desviaci\'{o}n del haz luminoso x, con la cual se puede calcular la fuerza F (ecuaci\'{o}n de la balanza de torsi\'{o}n).
\begin{figure}[hbtp]
\centering
\includegraphics[width=6.5cm]{../../../../../../Pictures/IMG_20170905_121446726.jpg}
\caption{Areglo experimental.}
\end{figure}
Cambie la distancia de separaci\'{o}n d de las esferas cargadas (manteniendo constante la carga inicial) en intervalos de 1cm cada vez y calcule la fuerza correspondiente. Y as\'{i} obtuvimos los siguientes datos:
\begin{figure}[hbtp]
\centering
\includegraphics[width=4.5cm]{../../../../../../Pictures/jejeje.jpg} 
\caption{Tabla1. Carga pernamece constante pero la distancia de separaci\'{o}n entre las cargas es variable.}
\end{figure}
\\
Y as\'{i} obtenemos la siguiente gr\'{a}fica de dispersi\'{o}n se obtine como sigue:
\begin{figure}[hbtp]
\centering
\includegraphics[width=10cm]{../../../../../../Pictures/graficochido.jpg}
\caption{Gr\'{a}fico de dispersi\'{o}n.}
\end{figure}
\\
\textbf{Ajuste de datos.} Se observa que los puntos experimentales siguen un patr\'{o}n alineal, entonces para ajustarlo a la mejor ecuaci\'{o}n ocuparemos un modelo exponencial ajustado por m\'{i}nimos cuadrados, esto es:

\[y=b{ x }^{ a }\]

Donde b y a son los par\'{a}metros buscados dados por:
\[a=\frac { n\sum _{ i=1 }^{ n }{ { x }_{ i }ln{ y }_{ i } } -\sum _{ i=1 }^{ n }{ ln{ y }_{ i } } \sum _{ i=1 }^{ n }{ { x }_{ i } }  }{ n\sum _{ i=1 }^{ n }{ { x }_{ i }^{ 2 } } (n)-\sum _{ i=1 }^{ n }{ { \left( { x }_{ i } \right)  }^{ 2 } }  } \]
\[ln(b)=\frac { \sum _{ i=1 }^{ n }{ { x }_{ i }ln({ y }_{ i })- } \sum _{ i=1 }^{ n }{ ln({ y }_{ i })\sum _{ i=1 }^{ n }{ { x }_{ i } }  }  }{ \sum _{ i=1 }^{ n }{ { x }_{ i }^{ 2 }(n)\quad -\sum _{ i=1 }^{ n }{ { \left( { x }_{ i } \right)  }^{ 2 } }  }  } . \]
Donde la tabla 1 queda de la siguiente manera:
\begin{figure}[hbtp]
\caption{Tabla 2.}
\centering
\includegraphics[width=5.5cm]{../../../../../../Pictures/gfgfgfgfgfgf.jpg}
\end{figure}
\\
\medskip
\medskip
\\
\textbf{Tabla de entrada 1.}
\\
\begin{figure 6}
\centering
\begin{tabular}{|c|c|c|c|c|c|}
\hline 
n&$\sum _{ i=1 }^{ n }{ { x }_{ i } } (m)$ & $\sum _{ i=1 }^{ n }{ { ln y }_{ i } }(1/s)$ & $ \sum _{ i=1 }^{ n }{ {ln y }_{ i } } { x }_{ i }(m/s)$ & $\sum _{ i=1 }^{ n }{ { x }_{ i }^{ 2 } }({ m }^{ 2 })$& $\sum _{ i=1 }^{ n }{ { \left( { x }_{ i } \right)  }^{ 2 }{ (m) }^{ 2 } }$ \\ 
\hline 
11& 1.43 & -123.43 & -16.6370766 & 0.1969 & 2.04  \\ 
\hline 
\end{tabular}
\end{figure 6} 
\\
Sustituyendo los valores queda:
\[ a=\frac { (11\times -16.6370766\left[ Nm \right] )-(-123.43\left[ N \right] \times (1.43)\left[ { m } \right] ) }{ (0.1969)\left[ { m }^{ 2 } \right] -(2.04)\left[ { m }^{ 2 } \right]  } =-6.373 \left[ { N }/m \right]  .\]

\[ ln(b)=\frac { \left( -16.6370766 \right) \left[ Nm \right] -\left( (-123.43)(1.43) \right) \left[ Nm \right]  }{ (0.1969)\left[ m^{ 2 } \right] -(2.04)\left[ m^{ 2 } \right]  } =-24.635288\left[ { N }/m \right] .\]
Y entonces 
\[ b=2\times { 10 }^{ -11 } \left[ { N }/m \right].\]
Finalmente queda el modelos propuesto:
\[ F(d)=2\times { 10 }^{ -11 }d^{ -6.373 }  \quad \rightarrow(1) \]
\medskip
\\
\section*{Ajuste por excel.}\\
Notemos que mientras por medio de c\'{a}alculos pudimos enconrar un modoelo, el programa Excel pudo enontrar otro, es cual se muestra a continuaci\'{o}n.
\\
\begin{figure}[hbtp]
 \centering
\includegraphics[width=10cm]{../../../../../../Pictures/EXCEL.jpg} 
\end{figure}
\\
\textbf{Procedimiento 2.}Con la balanza de torsi\'{o}n y amplificador lineal, obtener la fuerza que se ejercen entre si dos cuerpos cargados que se encuentran a una distancia fija, para diferentes valores de carga, obtenga los siguientes datos:\\
1) Constante de torsi\'{o}n de la balanza.\\
2) Distancia del espejo al pizarr\'{o}n.\\
3) Brazo de palanca.\\
4)Distancia de separaci\'{o}n entre las cargas.\\
Haga su arreglo experimental como se indica en la figura 1, coloque las esferas a 10 cm una de otra (mantenga esta distancia fija); con la m\'{a}quina electrost\'{a}tica y el portador de carga, cargue cada una de las esferas (con cargas de igual o distinto signo), y anote la desviaci\'{o}n del punto de equilibrio del haz luminoso x, con el amplificador lineal mida la carga de cada esfera y an\'{o}tela.
\begin{figure}[hbtp]
\caption{Arreglo experimental.}
\centering
\includegraphics[width=9cm]{../../../../../../Pictures/IMG_20170905_121619836.jpg}
\end{figure}
\\
As\'{i} pues obtuvimos los siguientes datos:\\
\begin{figure}[hbtp]
\caption{Tabla 2. Cargas variables pero distancia entre ellas constante.}
\centering
\includegraphics[width=8cm]{../../../../../../Pictures/hahahahaha.jpg} 
\end{figure}
\\
\begin{figure}[hbtp]
\centering
\includegraphics[width=8cm]{../../../../../../Pictures/geomememe.jpg}
\caption{Gr\'{a}fico de dispersi\'{o}n.}
\end{figure}
Notemos que nuestra gr\'{a}fica de dispersi\'{o}n no toma ninguna forma ni funci\'{o}n conocidad, as\'{i} que por el momento con la herrramientas con las que contamos hasta ahora solo podremos darle una forma, a la dispersi\'{o}n de puntos, a la de una func\'{o}n lineal.
\\
\textbf{Ajuste de datos.} 
Por el Apr\'{e}ndice 1 podemos hacer el respectivo ajuste por el m\'{e}todo de m\'{i}nimos cuadrados para encontrarun modelo lineal Y = ax + b tales tales que $\left( { x }_{ i },{ y }_{ i } \right) \rightarrow \left( Q1Q2(  C ^{ 2 } ),{F (N))$  de los datos del experimento, cuya tablas de entrada y modelo es el siguiente:
\\
\textbf{Tabla de entrada.}
\\
\begin{figure 6}
\centering
\begin{tabular}{|c|c|c|c|c|}
\hline 
n&$\sum _{ i=1 }^{ n }{ { x }_{ i } } (C ^{ 2 } )$ & $\sum _{ i=1 }^{ n }{ { y }_{ i } }(N)$ & $ \sum _{ i=1 }^{ n }{ { y }_{ i } } { x }_{ i }(N \cdot C ^{ 2 })$ & $\sum _{ i=1 }^{ n }{ { x }_{ i }^{ 2 } }(C ^{ 4 })$ \\ 

\hline 
9&$ 2.60908\times { 10 }^{ -15 } $& $5.07\times { 10 }^{ -4 }$ & $1.57\times { 10 }^{ -19 }$ & $5.84\times { 10 }^{ -2 }$\\ 
\hline 
\end{tabular}
\end{figure 6} 
\\
\\
De donde:
\[a=\frac { n\sum _{ i=1 }^{ n }{ { x }_{ i }{ y }_{ i } } -\sum _{ i=1 }^{ n }{ { x }_{ i } } \sum _{ i=1 }^{ n }{ { y }_{ i } }  }{ n\sum _{ i=1 }^{ n }{ { x }_{ i }^{ 2 } } -{ \left( \sum _{ i=1 }^{ n }{ { x }_{ i } }  \right)  }^{ 2 } } \quad y\quad b=\frac { \sum _{ i=1 }^{ n }{ { x }_{ i }^{ 2 } } \sum _{ i=1 }^{ n }{ { y }_{ i } } -\sum _{ i=1 }^{ n }{ { x }_{ i }{ y }_{ i } } \sum _{ i=1 }^{ n }{ { x }_{ i } }  }{ n\sum _{ i=1 }^{ n }{ { x }_{ i }^{ 2 } } -{ \left( \sum _{ i=1 }^{ n }{ { x }_{ i } }  \right)  }^{ 2 } }.\]
Sustituyendo los valores queda:

\[{ a }=\frac { (9\times 1.57\times { 10 }^{ -19 }\left[ N{ C }^{ 2 } \right] )-(2.61\times { 10 }^{ -15 }\times 5.07\times { 10 }^{ -4 }\left[ N{ C }^{ 2 } \right] ) }{ (9\times 5.84\times { 10 }^{ -2 }\left[ { C }^{ 4 } \right] )-{ \left( 2.1\times { 10 }^{ -15 }\left[ C \right]  \right)  }^{ 4 } } ={ 6\times 10 }^{ 11 }\left[ { N }/{ { C }^{ 2 } } \right] .\]

\[{ b }=\frac { (5.84\times { 10 }^{ -2 }\times 5.07{ \times 10 }^{ -4 }\left[ { NC }^{ 4 } \right] )-(1.57\times { 10 }^{ -19 }\times 2.61{ \times 10 }^{ -15 }\left[ { NC }^{ 4 } \right] ) }{ (9\times 5.84\times { 10 }^{ -2 }\left[ { C }^{ 4 } \right] )-{ \left( 2.61\times { 10 }^{ -15 }\left[ { C }^{ 2 } \right]  \right)  }^{ 2 } } =5.63\times { 10 }^{ -5 }\left[ N \right] .\]
Finalmente queda el modelos propuesto:
\[ F(Q1Q2)= { 6\times 10 }^{ 11 }(Q1Q2) + 5.63\times { 10 }^{ -5 } \quad \rightarrow(2) \]
De (2)  podemos graficar sus correspondientes lineas de tendencia, a continuaci\'{o}n vamos a graficar su modelo y a tratarlo con m\'{a}s detalle
\section*{Ajuste por excel.}\\
Notemos que mientras por medio de c\'{a}alculos pudimos enconrar un modoelo, el programa Excel pudo enontrar otro, es cual se muestra a continuaci\'{o}n.
\\
\begin{figure}[hbtp]
 \centering
\includegraphics[width=10cm]{../../../../../../Pictures/EXELITO.jpg} 
\end{figure}
\\
\pagebreak
%----------------------------------------------------------------------------------------
%	SECTION 6
%----------------------------------------------------------------------------------------

\section{Discusi\'{o}n.}
En el procedimiento 1 obtuvimos una curva  muy parecia a la esperada, sin embargo, al considerar el ajuste para la funci\'{o}n que depende de la distancia para la fuerza se encontr\'{o} que la misma no depende del cuadrado de la distancia, sin de la potencia sexta, pero a su vez multplicada por una cosntante, eso pudo ser porque fueron pocos datos o inclusive aproximando por polinomio de Taylor a primer orden lo obtendr\'{i}amos, sin embargo tambi\'{e}n pudo ser por la humedad en el medio. Para el procedimiento 2 se encontr\'{o} una tendencia lineal como se esperaba, adem\'{a}s de la funci\'{o}n de la fuerza variando con la carga parece no tan presciso, sin embargo se ha hayado una expresi\'{o}n aceptable para ella. 

%----------------------------------------------------------------------------------------
%	SECTION 7
%----------------------------------------------------------------------------------------

\section{Conclusiones.}
En esta pr\'{a}ctica aprendimos a utilizar e identificar bien la balanza de torsi\'{o}n utilizando dos esferas cargadas que si bien variamos las distancias entre ellas o no, encontramos en cualquiera de ellas una expresi\'{o}n en t\'{e}rminos de las variables. Se concluye que si la fuerza var\'{i}a cuando lo hace su distancia entre ellas su curva de tendencia era una para\'{a}bola o bien en forma exponencial  (si bien no del todo como cuadrado de la distancia) mientras que la fuerza en funci\'{o}n de la variaci\'{o}n de las cargas se muestra una tendencia lineal, es decir se encontr\'{o} el comportamiento cualitativo y se acerc\'{o} bastante al aproximado al comportamiento marcado por Coulomb.

%----------------------------------------------------------------------------------------
%	SECTION 9
%----------------------------------------------------------------------------------------
 \section{Bibliograf\'{i}a.}
1.- https://es.wikipedia.org/wiki/Ley_de_Coulomb.\\
2.-https://es.wikipedia.org/wiki/Balanza_de_torsiC3B3n.\\
3.-https://es.wikipedia.org/wiki/MultC3ADmetro.\\
4.-https://sites.google.com/site/labenriquesalgadoruiz/home/politecnico-1/fisica-iii.\\
5.-Resnick/Halliday/Krane. Fundamentos de F\'{i}sica. Volumen 2. Edici\'{o}n 6, extendida. CESA\\

%----------------------------------------------------------------------------------------
%	SECTION 10
%----------------------------------------------------------------------------------------

\end{document}