\documentclass[10pt,a4paper]{article}
\usepackage[latin1]{inputenc}
\usepackage[spanish]{babel}
\usepackage[utf8]{inputenc}
\usepackage{amsmath}
\usepackage{amsfonts}
\usepackage{amssymb}
\usepackage{graphicx}
\usepackage[left=2cm,right=2cm,top=2cm,bottom=2cm]{geometry}
\begin{document}
\section*{Desarrollo Experimental.}\\
\textbf{Lista de Materiales:} \\
1.- Soporte del calefactor (Alambre de cobre de 25 cm)\\
2.- Agitador (Varilla de pl\'{a}stico ( de ap\'{o}ximadamente 3m de di\'{a}metro y 15 cm de largo). \\
3.- Soporte con mariposas. \\
4.- Termo para l\'{i}quidos. (Hecho de polietileno)\\
5.- Parrila para calentar l\'{i}quidos.\\
6.- Term\'{o}metro.\\
\medskip
\section*{Arreglo experimental.}
\begin{figure}[hbtp]
\centering
\\
\includegraphics[width=9cm]{../../../../../../Pictures/Arreglogeneral.jpg} 
\caption{Arreglo general}
\end{figure} 
\\
\pagebreak

\section*{Procedimiento.}
\\
Primera fase: Determinaci\'{o}n del calor espec\'{i}fico del calor\'{i}metro.\\
1.- Mida la masa del vaso interior del calor\'{i}metro (${m}_{c}$).\\
2.- Vierta en el calor\'{i}metro 20 ml de agua (${m}_{1}$) a temperatura ambiente.\\
3.- Cierre el calor\'{i}metro, espere aproximadamente 1 minuto y mida la temperatura del equiibrio que alcanza el vaso interior del calor\'{i}metro y el agua que verti\'{o} en el paso 2 (${T}_{1} $). \\
4.- Vierta en el calor\'{i}metro 30 ml de agua (${m}_{2}$) previamente calentada a una temperatura aproximada de 70$\ºC$ (${T}_{2}$), y cierre el calor\'{i}metro lo m\'{a}s r\'{a}pido que pueda una vez vaciada el agua caliente.\\
5.- Espere a que se alcance la temperatura de equilibrio de la mezcla de agua que contiene el calor\'{i}metro, para ello observe cuidadosamente el term\'{o}metro hasta que la medici\'{o}n se estabilice (eso deber\'{a} ocurrir aproximadamente 1 minuto despu\'{e}s de completado el punto 4), cuando eso ocurra anote la temperatura de equilibrio (${T}_{f}$) .\\
6.- El calor\'{i}metro se llena con un aslante t\'{e}rmico que pueda ser lana de v\'{i}drio, corcho o icopor.\\
7.- Accesorio: Term\'{o}metro gradudado en grados celsuis.\\
8.- Se cartan unos 8 cm de un  alambre de resistencia de estufa del cula se hace el eemento calefactor.\\
\medskip
\\
\section*{Resultados.}
Este procedimiento permite ahora aplicar la \'{u}ltima ecuaci\'{o}n, la cual nos permite determinar
\[\frac { { C }_{ calir'{i}metro\quad  }\times { m }_{ c }({ T }_{ f }-{ T }_{ 1 }) }{ { m }_{ 2 }({ T }_{ f }-{ T }_{ 2 })-{ m }_{ 1 }(T_{ f }-{ T }_{ 1 }) } ={ c }_{ AGUA }\]
\\
\textbf{Tabla 1.}
\\
\begin{figure 6}
\centering
\begin{tabular}{|c|c|c|c|c|c|c|}
\hline 
T f ( C)& T1 ( C)& T2 ( C)& mc (Kg) & m1 (Kg)& m2 (kg) & C-calo (Cal/g C).\\ 
\hline 
38 & 25 & 70 & $3.47\times { 10 }^{ -3 }$ & $3.67\times { 10 }^{ -3 }$& $4.18\times { 10 }^{ -3 }$&1.9\\ 
\hline 
\end{tabular}
\end{figure 6} 
\\
Sustituyrndo nos queda que:
\[{ c }_{ AGUA }=\frac { 1.9\times 377\times { 10 }^{ -3 }(70-38) }{ 367\times { 10 }^{ -3 }(38-25)-4.18\times { 10 }^{ -3 }(70-38_{ 1 }) } ={ c }_{ AGUA }\]
Por lo tanto ${ c }_{ AGUA }=0.731$ calor\'{i}a/gramo $\°C$
\section*{Error porcentual.}
Tenemos que el agua tiene un calor espec\'{i}fico de 1 calor\'{i}a/gramo $\°C$=4,186 jules/g $\°C$,, entonces el error porcentual ser\'{a} de:
\[{ E }rror\quad porcentual\quad { M }_{ c }=\frac { Error\quad verdadero }{ Valor\quad Verdadero } =\frac { Valor\quad verdadero - Valor\quad aproximado }{ Valor\quad verdadero } \times 100= 27\%. \]
\section*{Discusiones.}
Conjeturar que el material con el que est\'{a} completamente hecho nuestro calor\'{i}metro, es decir, uq nuestro calor\'{i}metro est\'{e} hecho solamnte de polietileno es equivocado, sin ebargo pudo acercarse al color espec\'{i}fico de agua, para datos m\'{a}s precisos necesitariamos tomar otra parte del material (como podr\'{i}a ser Espuma de poliuretano). 

\end{document}