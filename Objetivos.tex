\documentclass[11pt,a4paper]{article}
\usepackage[latin1]{inputenc}
\usepackage[spanish]{babel}
\usepackage{amsmath}
\usepackage{amsfonts}
\usepackage{amssymb}
\usepackage{makeidx}
\usepackage{graphicx}
\usepackage[left=2cm,right=2cm,top=2cm,bottom=2cm]{geometry}
\author{Flores Rodguez Jaziel David }
\begin{document}
\section*{OBJETIVOS:}
\medskip
\medskip
\\
\textbf{Objetivos particulares de la unidad:}\\
1.Entender los principios de la elasticidad lineal(tensi\'{o}n) y elasicidad tangencial (corte), as\'{i} como las popiedades de los materiales involucrados en esos principios (m\'{o}dulo de Young, constante de reposici\'{o}n y m\'{o}dulo de rigidez.\\
2.Realizar emperimentos en donde se involucra el c\'{a}lculo de esas propiedades.\\
\medskip
\\
\textbf{Objetivos del experimento:} 
\\Exposici\'{o}n te\'{o}rica del concepto de elasticidad (m\'{o}dulo de Young). Realizaci\'{o}n del experimento correspondiente mediante el cual se determinar\'{a} la elongaci\'{o}n de ua muestra para relacionarla con un peso suspendido y poder calcular el m\'{o}dulo de Young de la muestra. 
\end{document}