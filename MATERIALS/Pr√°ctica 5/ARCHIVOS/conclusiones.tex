 \documentclass[10pt,a4paper]{article}
\usepackage[latin2]{inputenc}
\usepackage[spanish]{babel}
\usepackage{amsmath}
\usepackage{amsfonts}
\usepackage{amssymb}
\usepackage{makeidx}
\usepackage{graphicx}
\usepackage[left=2cm,right=2cm,top=2cm,bottom=2cm]{geometry}

\begin{document}

\section*{7. Conclusiones:}\\
En general, al realizar la pr\'{a}ctica pudimos observar que la propiedad intr\'{i}nseca de \'{e}ste fluido (aceite) que fu\'{e} la viscosidad la cual hicimos variar a cierta temperatura, se encontr\'{o} un comportamiento esperado, se concluye que en caso de \'{e}ste aceite de motor SAE 40 el cual se mencion\'{o} que se usa para trabajos pesados y en tiempos de altas teperaturas, y que sin embargo su viscosidad si cambia considerablemente al aumentar la temperaura aunque en un rango muy bajo, casi 40 grados. Recalcamos,la dispersi\'{o}n de puntos no muestra un comportamiento de una funci\'{o}n lineal pudimos hacer una aprximaci\'{o}n muy buena con un comportamento de una funci\'{o}n estrictamente decreciente. 
\section*{8. Referencias:}\\
\\
\medskip
\\
\\1.- Bit\'{a}cora de laboratorio de Flores Rodr\'{i}guez Jaziel David.
\\
2.- Manual de pr\'{a}cticas auxilar. Autor: Fco. Havez Varela y las notas del profesor Salvador Tirado Guerra.
\\
3.- Fisica Universitaria - Sears - Zemansky - 12ava Edici\'{o}n - Cap\'{i}tulo l4 -2009.\\
4.- http://noria.mx/lublearn/entendiendo-los-grados-de-viscosidad-sae-para-lubricantes-de-motor/ \\
\end{document}