\documentclass[10pt,a4paper]{article}
\usepackage[latin1]{inputenc}
\usepackage[spanish]{babel}
\usepackage{amsmath}
\usepackage{amsfonts}
\usepackage{amssymb}
\usepackage{graphicx}
\usepackage[left=2cm,right=2cm,top=2cm,bottom=2cm]{geometry}
\begin{document}
\section*{2.  Desarrollo Experimental.}\\

 \\
\textbf{Lista de Materiales:} \\
Soporte con material \'{o}ptico reflexivo (espejo y rayo de luz).\\
Material a estudiar (lat\'{o}n o cobre). \\
Hoja de papel milim\'{e}trico. \\
Regla y l\'{a}piz. \\
Fuente de alimentaci\'{o}n.\\
Pesas de distintas masas. \\
Metro para medir. \\
Nivel de agua. \\
Medidor de \'{a}ngulos de inclinaci\'{o}n \\
\\
\\
\section*{2.1 Arreglo experimenal. Parte 1: M\'{e}todo \'{O}ptico.} \\

\begin{figure}[hbtp]
\centering
\includegraphics[width=10cm]{../../../../../Pictures/dddddd.jpg}  
\caption{Arreglo general}
\end{figure}

\begin{figure}[hbtp]
\centering
\includegraphics[width=11cm]{../../../../../Pictures/xxxxxxxxxxxxxxxxxx.jpg}  
\caption{a)Horizontalidad del rayo inicial.  b)Marca de origen. }
\end{figure}

\begin{figure}[hbtp]
\centering
\includegraphics[width=7cm]{../../../../../Pictures/16651396_1391761460896440_996823809_o.jpg} 
\caption{Agrupamiento de pesas en el arreglo.}
\end{figure}

\begin{figure}[hbtp]
\centering
\includegraphics[width=17cm]{../../../../../Pictures/ssssssss.jpg} 
\caption{Material para medir a H correspondiente.}
\end{figure}
\pagebreak 

\textbf{Procedimiento.}\\
\\
1.-Medir las const\'{a}ntes X, lo, x, y, calcular A.\\
2.-Asegurar la horizontalidad del rayo reflejado inicial y marcar el origen en el papel. (Figura 1)\\
3.-Colocar un peso y marcar en el papella nueva posici\'{o}n del \'{i}ndice. \\
4. Repetir el paso 3, de 12 a 25 veces.\\
5.-Medir en cada caso la H corespondiente, y calcular $\beta$, $\alpha$ y $\Delta l$ respectiva.\\
6.- Llenar la tabla.\\
7.- Graficar $\varepsilon$ VS (DU)t y ajustar para calcular la Y de la muestra.\\

\textbf{Resultados.}\\
\\
Comenzamos la pr\'{a}ctica con ell material proporcionado por el equipo de laboratorio, procedimos a medir algunas de nuestras constantes como lo son; la distancia del soporte medida desde el espejo hasta la hoja de papel milim\'{e}trico (X), la elongaci\'{o}n inicial del material ya sea lat\'{o}n o cobre, la distancia del soporte para en espejo hasta el hilo, y procedimos a medir el \'{a}rea transversal del material por medio de un tornillo microm\'{e}trico. \\
\medskip
\\
\medskip
\caption{Tabla 1.} 
\\ 
\begin{tabular}{|c|c|c|c|c|}
\hline 
Di\'{a}metro(m) & X (m) & A(${ m }^{ 2 }$) & lo(m)& x (m)\\ 
\hline 
${ 3.54\times 10 }^{ -3 }$ & 2.32 & ${ 3.937\times 10 }^{ -3 }$ & 1.185 &${ 8.5\times 10 }^{ -2 }$ \\
\hline 
\end{tabular}\\

\medskip
Llenamos la Tabla 2 proporcionada para despu\'{e}s graficar el esfuerzo vs deformaci\'{o}n unitaria y ajustar para calcular el Y de cada muestra.\\
\\
\textbf{Tabla 2.}\\
\\
\begin{figure 2}
\centering
\includegraphics[width=14cm]{../../../../../Pictures/momomom1.jpg} 
\\
\end{figure 2}
\\
De la cual extragimos los datos de la siguiente tabla para poder hacer la gr\'{a}fica $\varepsilon$ VS (DU).
\\
\textbf{Tabla 3.}\\
\\
\begin{figure 3}
\centering
\includegraphics[width=4cm]{../../../../../Pictures/momomomeo2.jpg} 
\\
\end{figure 3}
\\
\section*{Gr\'{a}fica de dispersi\'{o}n}
\\
\\
\begin{figure 5}
\centering
\includegraphics[width=10cm]{../../../../../Pictures/MOMOMO3.jpg} 
\\
\caption{Gr\'{a}fico de dispersi\'o}n para el m\'{e}todo \'{o}ptico.}
\end{figure 5}
\\
\section*{5. Ajuste de datos.}\\
Por el Apr\'{e}ndice 1 podemos hacer el respectivo ajuste por el m\'{e}todo de m\'{i}nimos cuadrados para encontrar un modelo lineal $Y=ax +b$ tales que $\left( { x }_{ i },{ y }_{ i } \right) \rightarrow \left( Du(Ad),{ \sigma  }_{ t }(Pa) \right) $ para cada uno de los datos de cada experimento y cuya tabla de entrada es:
\medskip
\\
\\Tabla de entrada 1.
\\
\\
\begin{figure 6}
\centering
\begin{tabular}{|c|c|c|c|c|}
\hline 
$\sum _{ i=1 }^{ n }{ { x }_{ i } } $(Ad) & $\sum _{ i=1 }^{ n }{ { y }_{ i } } $ (Pa) & $ \sum _{ i=1 }^{ n }{ { y }_{ i } } { x }_{ i } \left( Pa \right) \left( Ad \right)$ & $\sum _{ i=1 }^{ n }{ { x }_{ i }^{ 2 } }{ \left( Ad \right)  }^{ 2 }$ & n \\ 

\hline 
$1.541\times { 10 }^{ -2 }$& $1.837\times { 10 }^{ 9 }$ & $4.193\times { 10 }^{ 8 }$ & $2.449\times { 10 }^{ -5 }$& 13 \\ 
\hline 
\end{tabular}
\end{figure 6} 
\\
\\
De donde:
\[a=\frac { n\sum _{ i=1 }^{ n }{ { x }_{ i }{ y }_{ i } } -\sum _{ i=1 }^{ n }{ { x }_{ i } } \sum _{ i=1 }^{ n }{ { y }_{ i } }  }{ n\sum _{ i=1 }^{ n }{ { x }_{ i }^{ 2 } } -{ \left( \sum _{ i=1 }^{ n }{ { x }_{ i } }  \right)  }^{ 2 } } \quad y\quad b=\frac { \sum _{ i=1 }^{ n }{ { x }_{ i }^{ 2 } } \sum _{ i=1 }^{ n }{ { y }_{ i } } -\sum _{ i=1 }^{ n }{ { x }_{ i }{ y }_{ i } } \sum _{ i=1 }^{ n }{ { x }_{ i } }  }{ n\sum _{ i=1 }^{ n }{ { x }_{ i }^{ 2 } } -{ \left( \sum _{ i=1 }^{ n }{ { x }_{ i } }  \right)  }^{ 2 } }.\]
Sustituyendo los valores queda:

\[a=\frac { 13\times 4.193\times { 10 }^{ 8 }-\left[ 1.541\times { 10 }^{ -2 }\times 1.837\times { 10 }^{ 9 } \right]  }{ 13\times 2.449\times { 10 }^{ -5 }-{ \left( 1.541\times { 10 }^{ -2 } \right)  }^{ 2 } } ={ 6.67\times 10 }^{ 10 }(Pa).\]

\[b=\frac { 2.449\times { 10 }^{ -5 }\times 1.837\times { 10 }^{ 9 }-\left[ 4.193\times { 10 }^{ 8 }\times 1.541\times { 10 }^{ -2 } \right]  }{ 13\times 2.449\times { 10 }^{ -5 }-{ \left( 1.541\times { 10 }^{ -2 } \right)  }^{ 2 } } ={ -7.88\times 10 }^{ 10 }(Ad).\]
Finalmente queda el modelos propuesto:
\[{ Y }_{ 1 }= -(788\times { 10 }^{ 10 })(Ad) + (6.67\times { 10 }^{ 10 } (Pa))x\quad \rightarrow(1) \]

\\
Como la pendiente de la recta a tangente a la curva misma nos representa el m\'{o}dulo de Young , en el cas por su puesto para la gr\'{a}fica $\varepsilon$t VS (DU)t, es decir:
\[m=b=tan(\theta)= \frac { \varepsilon t }{ (Du)t } =\quad Y\quad (M\'{o}dulo\quad de\quad Young)\]
Y as\'{i}, podemos decir, por definici\'{o}n, que el m\'{o}duo de Young del material es: $Y=6.67\times 10 }^{ 10 }(Pa).$
\\
\section*{Error Porcentual.}\\
Los valores verdaderos (Taba 1) de los m\'{o}dulos de Young del cobre es ${ Y }_{ 1 }=11\times { 10 }^{ 10 }Pa$. Entonces, de nuestras mediciones y c\'{a}lculos podemos obtener el error porcentual:

\[{ E }rror-porcentual-{ Y }_{ 1 }=\frac { Error\quad verdadero }{ Valor\quad Verdadero } =\frac { Valor\quad verdadero - Valor\quad aproximado }{ Valor\quad verdadero } \times 100= 36 %)\]
\\
\section*{Ajuste por excel.}\\
Notemos que mientras por medio de c\'{a}alculos pudimos enconrar un modoelo, el programa Excel pudo enontrar otro, es cual se muestra a continuaci\'{o}n.

\begin{figure}[hbtp]
 \centering
 \includegraphics[width=11cm]{../../../../../Pictures/KKKKKKKKKKK.jpg} 
 \caption{Modelo por medio de excel. }
\end{figure}
 Cuya ecuaci\'{o}n es:
 \[{ Y }_{ ex}= (2\times { 10 }^{ 7 } (Ad))+(1\times { 10 }^{ 11 } (Pa))x\quad \rightarrow(1) \]
 
 \section*{Discusiones.}\\
Al llevar acabo este experimento pudimos notar que hubo ciertas fallas, como por ejemplo: perdimos la continuaci\'{o}n de las pesas, es decir, quitabamos unas para poner m\'{a}s pesadas y as\'{i} sin darnos cuenta perdimos nuestro r\'{e}gimen el\'{a}stico y se comenzaba a deformar nuestro alambre. Los modelos que propusimos tieen un amplio rango de error, adem\'{a}s uno de de un orden mayor, pero as\'{i} falla por un r\'{e}gimen aceptable a comparaci\'{o} de el valor verdadero.



\end{document}