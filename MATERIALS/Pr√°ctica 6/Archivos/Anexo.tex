\documentclass[10pt,a4paper]{article}
\usepackage[latin1]{inputenc}
\usepackage[spanish]{babel}
\usepackage[utf8]{inputenc}
\usepackage{amsmath}
\usepackage{amsfonts}
\usepackage{amssymb}
\usepackage{graphicx}
\usepackage[left=2cm,right=2cm,top=2cm,bottom=2cm]{geometry}
\begin{document}
\section*{Anexo.}
Veamos que, en el gr\'{a}fico de dispersi\'{o}n se pueden observar que despu\'{e}s de los tr\'{e}s primeros puntos existe un salto enorme sobre el patr\'{o}n de los puntos. As\'{i} que, como mostraremos en la sigiente Figura 1.1 discrimiaremos a estos y s\'{o}lo trabajaremos con los puntos aparentemente uniformes. 
\begin{figure}[hbtp]
\centering
\\
\includegraphics[width=10cm]{../../../../../../Pictures/ddddddddddddddddddddd.jpg} 
\caption{Los puntos encerrados se apartar\'{a}n}
\end{figure} 
\\
\section*{Arreglo de datos.}
El modelo que se requierir\'{a} entonces ser\'{a} de la siguiente forma:
\\
\begin{figure}[hbtp]
\centering
\\
\includegraphics[width=10cm]{../../../../../../Pictures/operads.jpg} 
\end{figure} 
\\
Cuya ecuaci\'{o}n es la siguiente $y = -{3\times 10^{-4}}x + 3.08\times 10^{-2}$.
\section*{Error porentual.}\\
Los valores verdaderos de la viscosidad del aceite SAE 40 (los cuales se encuentran en la tabla 1 del marco te\'{o}rico) a una temperaturamenor que $100 \°C$ debe ser menor que 16.3 cp (cp es una unidad de viscosidad denominada centi Poise, donde 1 centipoise = $1\times 10 }^{ -3 } Pa\cdots$. Entonces, de nuestras mediciones y c\'{a}lculos con un valor en nuestra ecuaci\'{o}n hallada de $49\°C$ {que claramente cumple con la condici\'{o}n de que sea menor a $100\°C$ y esta nos arroja un valor de 16.06 cp, un valor muy cercano al verdadero, y as\'{i} que podemos obtener el error porcentual, usando el rango de valores los cuales no cumple la condici\'{o}n:

\[{ E }rror\quad porcentual\quad { M }_{ c }=\frac { Error\quad verdadero }{ Valor\quad Verdadero } =\frac { Valor\quad verdadero - Valor\quad aproximado }{ Valor\quad verdadero } \times 100= 48.65\%. \]
\\
 \section*{Discusiones.}\\
En esta ocasi\'{o}n se observa que en apesar de que hicimos un an\'{a}lisis con los datos que aparentemente ten\'{i}an un patr\'{o}n m\'{a}s uniforme y que discriminamos 3 puntos, la ecuaci\'{o}n que constru\'{i}mos nos arroja valores con un mayor porcentaje de error, es decir, que tal vez los valores en de la imagen (viscosidad) de la funci\'{o}n deben ser un poco m\'{a}s altos y as\'{i} poder tener un menor porcentaje de error. 


\end{document}
