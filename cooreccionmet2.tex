\documentclass[10pt,a4paper]{article}
\usepackage[latin1]{inputenc}
\usepackage[spanish]{babel}
\usepackage{amsmath}
\usepackage{amsfonts}
\usepackage{amssymb}
\usepackage{graphicx}
\usepackage[left=2cm,right=2cm,top=2cm,bottom=2cm]{geometry}
\begin{document}

\section*{2.2 Arreglo experimenal. Parte 2: M\'{e}todo de Nivel.} \\

\begin{figure}[hbtp]
\centering
\includegraphics[width=6cm]{../../../../../Pictures/general.jpg}  
\caption{Arreglo general.}
\end{figure}

\begin{figure}[hbtp]
\centering
\includegraphics[width=5cm]{../../../../../Pictures/jhhhfajsfhf.jpg}  
\caption{Vista lateral del arreglo}
\end{figure}

\pagebreak 

\textbf{Procedimiento.}\\
\\
1.-Medir las const\'{a}ntes lo, x.\\
2.-Colocar una masa en el alambre y registrar el alcance del tornillo, nivelando de nuevo barra por medio de la burbuja.\\
3. Repetir el paso anterior, de 12 a 25 veces.\\
4.- Llenar la tabla.\\
5.- Graficar $\varepsilon$ VS (DU)t y ajustar para calcular la Y de la muestra.\\

\textbf{Resultados.}\\
\\
Comenzamos la pr\'{a}ctica con ell material proporcionado por el equipo de laboratorio, procedimos a medir algunas de nuestras constantes como lo son; la elongaci\'{o}n inicial del material ll\'{a}mese lat\'{o}n, procedimos a medir el \'{a}rea transversal del material por medio de un tornillo microm\'{e}trico y tambi\'{e}n la medida inicial del instrumento con medida angular. \\
\medskip
\\
\medskip
\medskip
\caption{Tabla 1.} 
\\
\begin{tabular}{|c|c|c|c|c|}
\hline 
Di\'{a}metro(Ad) & A(${ m }^{ 2 }$) & lo(m)& $\varphi$ (m)\\ 
\hline 
${ 1.34\times 10 }^{ -3 }$ & ${ 3.84\times 10 }^{ -5 }$ & 1.438 & ${ 3.5\times 10 }^{ -4 }$ \\
\hline 
\end{tabular}\\


\medskip

Llenamos la Tabla 2 proporcionada para despu\'{e}s graficar el esfuerzo vs deformaci\'{o}n unitaria y ajustar para calcular el Y de cada muestra.\\
\textbf{Tabla 2.}\\
\\
\begin{figure 2}
\centering
\includegraphics[width=12cm]{../../../../../Pictures/popopop.jpg}
\\
\caption{De la cual extragimos los datos de la siguiente tabla para poder hacer la gr\'{a}fica $\varepsilon$ VS (DU)t.}
\end{figure 2}

\begin{figure 3}
\centering
\caption{\textbf{Tabla 3.}}
\\
\includegraphics[width=6cm]{../../../../../Pictures/fgqhwregnetw.jpg}
\\
\caption{Datos extra\'{i}dos.}
\end{figure 3}


\\
\section*{Gr\'{a}fica de dispersi\'{o}n}
\\
\\
\begin{figure 5}
\centering
\includegraphics[width=9cm]{../../../../../Pictures/rrrrrrrrrrrrr.jpg} 
\\ 
\caption{Gr\'{a}fico de dispersi\'o}n para el m\'{e}todo de nivel.}
\end{figure 5}
\\
\section*{5. Ajuste de datos.}\\
Por el Apr\'{e}ndice 1 podemos hacer el respectivo ajuste por el m\'{e}todo de m\'{i}nimos cuadrados para encontrar un modelo lineal $Y=ax +b$ tales que $\left( { x }_{ i },{ y }_{ i } \right) \rightarrow \left( Du(Ad),{ \sigma  }_{ t }(Pa) \right) $ para cada uno de los datos de cada experimento y cuya tabla de entrada es:
\medskip
\\
\caption{Tabla de entrada 2.}
\\
\begin{figure 6}
\\
\medskip  
\medskip 
\centering
\begin{tabular}{|c|c|c|c|c|}
\hline 
$\sum _{ i=1 }^{ n }{ { x }_{ i } } $(Ad) & $\sum _{ i=1 }^{ n }{ { y }_{ i } } $ (Pa) & $ \sum _{ i=1 }^{ n }{ { y }_{ i } } { x }_{ i }$ (Pa) & $\sum _{ i=1 }^{ n }{ { x }_{ i }^{ 2 } }(Ad)$ & n \\ 
\hline 
$1.445\times { 10 }^{ -2 }$& 4,590,874.98 & 6,752,629 & $2.166\times { 10 }^{ -5 }u$& 11 \\ 
\hline 
\end{tabular} 
\end{figure 6}
\\

\end{figure 6} 
\\
\\
De donde:
\[a=\frac { n\sum _{ i=1 }^{ n }{ { x }_{ i }{ y }_{ i } } -\sum _{ i=1 }^{ n }{ { x }_{ i } } \sum _{ i=1 }^{ n }{ { y }_{ i } }  }{ n\sum _{ i=1 }^{ n }{ { x }_{ i }^{ 2 } } -{ \left( \sum _{ i=1 }^{ n }{ { x }_{ i } }  \right)  }^{ 2 } } \quad y\quad b=\frac { \sum _{ i=1 }^{ n }{ { x }_{ i }^{ 2 } } \sum _{ i=1 }^{ n }{ { y }_{ i } } -\sum _{ i=1 }^{ n }{ { x }_{ i }{ y }_{ i } } \sum _{ i=1 }^{ n }{ { x }_{ i } }  }{ n\sum _{ i=1 }^{ n }{ { x }_{ i }^{ 2 } } -{ \left( \sum _{ i=1 }^{ n }{ { x }_{ i } }  \right)  }^{ 2 } }.\]
Sustituyendo los valores queda:

\[a=\frac { 11\times 6,752,629-\left[ 1.445\times { 10 }^{ -2 }\times 4,59,874.98\right]  }{ 11\times 2.166\times { 10 }^{ -5 }-{ \left( 1.445\times { 10 }^{ -2 } \right)  }^{ 2 } } ={ 26.935\times 10 }^{ 10 }(Pa).\]

\[b=\frac { 2.166\times { 10 }^{ -5 }\times 4.590,874.98-\left[ 6,752.629\times 1.44\times { 10 }^{ -2 } \right]  }{ 11\times 2.116\times { 10 }^{ -5 }-{ \left( 1.445\times { 10 }^{ -2 } \right)  }^{ 2 } } = 63,523.440 (Ad) \]
Finalmente queda el modelos propuesto:
\[{ Y }_{ 1 }= (63,523.440 (Ad))+(26.935\times 10 }^{ 9 } (Pa))x\quad \rightarrow(1) \]

Como la pendiente de la recta a tangente a la curva misma nos representa el m\'{o}dulo de Young , en el cas por su puesto para la gr\'{a}fica $\varepsilon$t VS (DU)t, es decir:
\[m=b=tan(\theta)= \frac { \varepsilon t }{ (Du)t } =\quad Y\quad (M\'{o}dulo\quad de\quad Young)\]
Y as\'{i}, podemos decir, por definici\'{o}n, que el m\'{o}duo de Young del material es: $Y=26.935\times 10 }^{ 10 }(Pa).$
\\
\section*{Error Porcentual.}\\
Los valores verdaderos de los m\'{o}dulos de Young del lat\'{o}n es ${ Y }_{ 1 }=90\times { 10 }^{ 9 }Pa$. Entonces, de nuestras mediciones y c\'{a}lculos podemos obtener el error porcentual:

\[{ E }rror-porcentual-{ Y }_{ 1 }=\frac { Error\quad verdadero }{ Valor\quad Verdadero } =\frac { Valor\quad verdadero - Valor\quad aproximado }{ Valor\quad verdadero } \times 100=67 % \]
\\
\section*{Ajuste por excel.}\\
Notemos que mientras por medio de c\'{a}alculos pudimos enconrar un modelo, el programa Excel pudo enontrar otro, es cual se muestra a continuaci\'{o}n.

\begin{figure}[hbtp]
 \centering
\includegraphics[width=8cm]{../../../../../Pictures/GGGGGGGGG.jpg} 
 \caption{Modelo por medio de excel. }
\end{figure}
 Cuya ecuaci\'{o}n es:
 \[{ Y }_{ ex}= (3\times { 10 }^{ 9 }(Pa))x +(66685(Ad))\quad \rightarrow(1) \]
 
 \section*{Discusiones.}\\
 Al llevar acabo este experimento pudimos notar que hubo ciertas fallas, como por ejemplo: perdimos la continuaci\'{o}n de las pesas, es decir, quitabamos unas para poner m\'{a}s pesadas y as\'{i} sin darnos cuenta perdimos nuestro r\'{e}gimen el\'{a}stico y se comenzaba a deformar nuestro alambre. Los modelos que propusimos tienen un amplio rango de error, adem\'{a}s uno de de un orden mayor, pero as\'{i} falla por un r\'{e}gimen aceptable a comparaci\'{o} de el valor verdadero. Desde la linea vertical hasta el final del conjunsto de puntos discretos en la Figura anterior, el esfuerzo y la deformaci\'{o}n ya no son proporcionales, y no se obedece la ley de Hooke, la deformaci\'{o}n sigue aumentando, el material sufri\'{o} una deformaci\'{o}n irreversible y adquiri\'{o} un ajuste permanente. El comportamiento del material entre la regi\'{o}n se denomina flujo pl\'{a}stico o deformaci\'{o}n pl\'{a}stica. Una deformaci\'{o}n pl\'{a}stica es irreversible; si se elimina el esfuerzo, el material no vuelve a su estado original. 
 

\end{document}