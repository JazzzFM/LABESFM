\documentclass[10pt,a4paper]{article}
\usepackage[latin2]{inputenc}
\usepackage[spanish]{babel}
\usepackage{amsmath}
\usepackage{amsfonts}
\usepackage{amssymb}
\usepackage{makeidx}
\usepackage{graphicx}
\usepackage[left=2cm,right=2cm,top=2cm,bottom=2cm]{geometry}

\begin{document}

\section*{7. Conclusiones:}\\
En general, al realizar la pr\'{a}ctica pudimos observar que la propiedad intr\'{i}nseca de \'{e}ste fluido (aceite) que fu\'{e} la tensi\'{o}n superficial la cual hicimos variar a cierta temperatura, se encontr\'{o} un comportamiento esperado, se concluye que en caso de \'{e}ste aceite de motor SAE 40 el cual se mencion\'{o} que se usa para trabajos pesados y en tiempo de mucho calor y por lo tanto  no cambia mucho sus propiedades a altas temperaturas, y eso fue lo que se encontr\'{o}, a pesar de que la escuaci\'{o}n de tendencia marca un comportamiento decreciente el cual es muy lento y esto mismo se debe a la naturalea del fluido. 
\section*{8. Referencias:}\\
\\
\medskip
\\
\\1.- Bit\'{a}cora de laboratorio de Flores Rodr\'{i}guez Jaziel David.
\\
2.- Manual de pr\'{a}cticas auxilar. Autor: Fco. Havez Varela y las notas del profesor Salvador Tirado Guerra.
\\
3.- Fisica Universitaria - Sears - Zemansky - 12ava Edici\'{o}n - Cap\'{i}tulo l4 -2009.
\end{document}