\documentclass[11pt,a4paper]{article}
\usepackage[latin1]{inputenc}
\usepackage[spanish]{babel}
\usepackage{amsmath}
\usepackage{amsfonts}
\usepackage{amssymb}
\usepackage{makeidx}
\usepackage{graphicx}
\usepackage{lmodern}
\usepackage[left=2cm,right=2cm,top=2cm,bottom=2cm]{geometry}
\author{Flores Rodguez Jaziel David }
\title{Práctica 1}
\begin{document}

\tableofcontents
%----------------------------------------------------------------------------------------
%	SECTION 1
%----------------------------------------------------------------------------------------
\\
\medskip
\medskip
\section{Resumen.}
\
Se maneja de manera experimental el efecto Joule debido a una reistencia ele\'{e}ctrica sumergida en el seno de un calor\'{i}metro y usando el principio de conservaci\'{o}n de la energ\'{i}a se encuentra emp\'{i}ricamente el equivalente el\'{e}ctrico del calor, se tomean datos hasta que la temperatura est\'{e} adecuada por sobre la temperatura ambiente. Estas condiciones minimizan efectos del medio, ya que el agua gana energ\'{i}a de \'{e}ste durante la mitad del experimento y le cede durante la otra mitad. Se resalta que en el Calor\'{i}metro en la tapa con sus cables bien conectados se evita as\'{i} p\'{e}rdidas por radiaci\'{o}n y convecci\'{o}n.\\
\\

\medskip
Palabras clave: Calor\'{i}metro, principio de conservaci\'{o}n de la energ\'{i}a, radiaci\'{o}n. convenci\'{o}n, efecto Joule.

%----------------------------------------------------------------------------------------
%	SECTION 2
%----------------------------------------------------------------------------------------

\section{Objectivo.}
Estudiar la transferencia de energ\'{i}a entre una resistencia el\'{e}ctrica energizada y el medio ambiente que est\'{a} sumergida (agua) obteniendo, a partir de los resultados, una medici\'{o}n emp\'{i}rica del equivalente el\'{e}ctrico del calor.
\pagebreak


%----------------------------------------------------------------------------------------
%	SECTION 3
%----------------------------------------------------------------------------------------
\section{Metodolog\'{i}a.}
El agua es calentada por una resistencia el\'{e}ctrica sumergida en ella, por la que circula una corriente. El calor disipado por efecto Joule en la resistencia es transferido al agua. Usamos el Principio de Conservaci\'{o}n de la Energ\'{i}a, suponiendo que no hay p\'{e}rdidas de calor, lo que equivale a que toda la energ\'{i}a entregada por la resistencia es absorbida por el agua. La energ\'{i}a disipada en la resistencia es:
\[E= \~{P}t \quad (1) \]
Donde t es el tiempo que circula la corriente, $\~{P}$ es la potencia media dada por $\~{P}=\~{V}{\~{I}}^{2} \quad (2) }$ donde $\~{V}$ es el voltaje promedio y $\~{I}$ la corriente promedio. La energ\'{i}a absorbida por el agua est\'{a} dada por:
\[Q = m c \Delta T \° \quad(3)\]
Donde m es la masa de agua, c es el calor espec\'{i}fico del agua (1 cal /g $\°C$) y $\Delta T \°$ es el cambio en la temperatura del agua. Para obtener el equivalente el\'{e}ctrico del calor, se iguala la energ\'{i}a disipada por la resistencia (en Joule) a la energ\'{i}a ganada por el agua (en calor\'{i}as). Por lo tanto
\[(\~{V}{\~{I}}^{2})t = m c \Delta T \° \quad(4)\]

%----------------------------------------------------------------------------------------
%	SECTION 4
%----------------------------------------------------------------------------------------

\section{Instrumentaci\'{o}n.}\\
1.- Amplificador Lineal.\\
2.- Mult\'{i}metro.\\
3.- Resistencia de filamento 0.5$\Omega $. \\
4.- Balanza Digital.\\
5.- Calor\'{i}metro de aluminio.\\
6. -Conectores.\\
7.- 120 ml de agua destilada.\\
8.- Term\'{o}metro graduado en Celsius.\\
\pagebreak

%----------------------------------------------------------------------------------------
%	SECTION 5
%----------------------------------------------------------------------------------------


\section{Datos y Resultados.}\\

\textbf{PARTE 1.}\\
\\
Mase el recipiente interior del calor\'{i}metro. Introduzca la resistencia de filamento de 0,5$ \Omega$  en el interior del Calor\'{i}metro, uniendo a ella los conectores necesarios. Conecte los cables de la resistencia a la fuente el\'{e}ctrica CC ( 2 a 3 volts) se evita que los cables hagan contacto con agua o se topen entre ellos.
\\
\\
Aseg\'{u}rese de que no se produzcan corrientes de aire en el interior del laboratorio, para no alterar las mediciones de la temperatura.
\\
\\
Conecte los Instrumentos Amper\'{i}metro y Volt\'{i}metro. No active la Fuente CC. Si la conecta con la asesor\'{i}a de un asistente cerci\'{o}rese previamente que el Volt\'{i}metro est\'{e} en paralelo a la resistencia y el Amper\'{i}metro en serie a la misma.
\\
\\
CUIDADO: Aseg\'{u}rese que la resistencia est\'{e} sumergida en agua cuando conecte el circuito. En caso contrario, \'{e}sta se quemar\'{a} al aplicar el voltaje.\\
Ponga 120 ml de agua en el vaso y mida la masa neta de agua dentro del Calor\'{i}metro. Use agua que est\'{e} a unos tres grados por debajo de la temperatura ambiente al iniciar la recolecci\'{o}n de datos (solo en \'{e}pocas c\'{a}lidas). Este procedimiento se realiza si el laboratorio est\'{a} a 18$ \° C  $o superior. Tome datos hasta que la temperatura est\'{e} a unos tres grados por sobre la temperatura ambiente. Estas condiciones minimizan efectos del medio, ya que el agua gana energ\'{i}a de \'{e}ste durante la mitad del experimento y le cede durante la otra mitad. Reste la masa del vaso, de la masa total, para obtener la masa del agua neta o puede usar el TARE de la balanza. Sumerja la resistencia en el agua. Cerci\'{o}rese si el Calor\'{i}metro tiene la tapa con sus cables bien conectados se evita as\'{i} p\'{e}rdidas por radiaci\'{o}n y convecci\'{o}n.
\\
\\
\textbf{Ejecuci\'{o}n del Experimento.}
\\IMPORTANTE: mientras se realice la medici\'{o}n, agite suavemente el agua, para asegurar calentamiento uniforme. Al menos registre datos durante 600 segundos. El term\'{o}metro registra en $ \°C $y su precisi\'{o}n es de $ 1 \°C  $ por lo tanto puede estimarse valores de cada medio grado. Cuando la temperatura alcance un valor tres grados por encima de la ambiente, abra el circuito el\'{e}ctrico. Contin\'{u}e agitando el agua y tomando datos. La temperatura subir\'{a} hasta un valor m\'{a}ximo, cuando todo el calor de la resistencia se haya disipado, y luego empezar\'{a} a descender, por disipaci\'{o}n al medio imprima su tabla de datos. Anote las temperaturas m\'{a}xima y m\'{i}nima de la tabla, en el rango v\'{a}lido de mediciones. Grafique Temperatura vs tiempo y anote los valores medios de voltaje y corriente que figuran al final de la tabla. Calcule la pendiente del gr\'{a}fico y relaci\'{o}nela con la constante equivalente J.\\



\section{Conclusiones.}

%----------------------------------------------------------------------------------------
%	SECTION 9
%----------------------------------------------------------------------------------------
 \section{Bibliograf\'{i}a.}
1.-http://guasa.ya.com/elektron/electropedia.html\\
2.-https://es.wikipedia.org/wiki/MultC3ADmetro.\\
3.-https://sites.google.com/site/labenriquesalgadoruiz/home/politecnico-1/fisica-iii .\\
4.-Resnick/Halliday/Krane. Fundamentos de F\'{i}sica. Volumen 2. Edici\'{o}n 6, extendida. CESA\\
5.- https://es.wikipedia.org/wiki/Campo_electri78co.\\

%----------------------------------------------------------------------------------------
%	SECTION 10
%----------------------------------------------------------------------------------------

\end{document}