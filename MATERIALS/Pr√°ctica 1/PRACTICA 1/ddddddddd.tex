\documentclass[10pt,a4paper]{article}
\usepackage[latin1]{inputenc}
\usepackage[spanish]{babel}
\usepackage{amsmath}
\usepackage{amsfonts}
\usepackage{amssymb}
\usepackage{graphicx}
\usepackage[left=2cm,right=2cm,top=2cm,bottom=2cm]{geometry}
\begin{document}
\section*{2.  Desarrollo Experimental.}\\
\\
\\
\textbf{Materiales:} \\
Soporte con material \'{o}ptico reflexivo (espejo y rayo de luz).\\
Material a estudiar (lat\'{o}n o cobre). \\
Hoja de papel milim\'{e}trico. \\
Regla y l\'{a}piz. \\
Fuente de alimentaci\'{o}n.\\
Pesas de distintas masas. \\
Metro para medir. \\
Nivel de agua. \\
Medidor de \'{a}ngulos de inclinaci\'{o}n \\
\\
\\
\section*{Arreglo experimenal. Parte 1: M\'{e}todo \'{O}ptico:} \\

\begin{figure}[hbtp]
\centering
\includegraphics[width=15cm]{../../../../../Pictures/dddddd.jpg}
\caption{Arreglo general. }
\end{figure}

\begin{figure}[hbtp]
\centering
\includegraphics[width=15cm]{../../../../../Pictures/xxxxxxxxxxxxxxxxxx.jpg}
\caption{a)Horizontalidad del rayo inicial.  b)Marca de origen.  }
\end{figure}

\begin{figure}[hbtp]
\centering
\includegraphics[width=15cm]{../../../../../Pictures/16651396_1391761460896440_996823809_o.jpg}
\caption{Agrupamiento de pesas en el arreglo. }

\begin{figure}[hbtp]
\centering
\includegraphics[width=15cm]{../../../../../Pictures/ssssssss.jpg}
\caption{Material para medir a H correspondiente, y calcular $\beta$, $\alpha$ y $\Delta l$ respectiva.}
\end{figure}
\\

\pagebreak 
Comenzamos la pr\'{a}ctica con el material proporcionado por el equipo de laboratorio, procedimos a medir algunas de nuestras constantes como lo son; la distancia del soporte medida desde el espejo hasta la hoja de papel milim\'{e}trico (X), la elongaci\'{o}n inicial del material ya sea lat\'{o}n o cobre, la distancia del soporte para en espejo hasta el hilo, y procedimos a medir el \'{a}rea transversal del material por medio de un tornillo microm\'{e}trico. \\
Aseguramos la horizontalidad del rayo inicial reflejado y marcar el origen del papel. (Figura 2 (a) y (b))
Colocamos un peso con las masas proporcionadas y marcamos en el papel la nueva posici\'{o}n del \'{i}ndice.\\
Y repetimos este mismo paso 13 veces (Figura 3).
\end{figure}\\
Medimos en cada caso la H correspondiente, y calcular $\beta$, $\alpha$ y $\Delta l$ respectiva (Figura 4). Llenamos la Tabla 1 proporcionada para despu\'{e}s graficar el esfuerzo vs deformaci\'{o}n unitaria y ajustar para calcular el Y de cada muestra.

\section*{3. Tablas.}
\\
Para cada experimento se extrajeron las siguientes tablas de datos, cuyos valores derivados fueron calculados y se encuentran en la bit\'{a}cora de trabajo como referencia.
\\
\begin{figure 2}
\centering
\includegraphics[width=10cm]{../../../../../Pictures/fffffffffff.jpg} 
\\
\caption{Tabla 1. Datos del M\'{e}todo \'{O}ptico.}
\\
\centering
\end{figure 2}
\\
\begin{figure 3}
\centering
\includegraphics[width=10cm]{../../../../../Pictures/tttttttttthhhhhh.jpg} 
\\
\caption{Tabla 2. Datos del M\'{e}todo de nivel.}\\
\\
\\
\end{figure 3}

\medskip
\section*{4. G\'{a}ficas de dispersi\'{o}n.}
\\
Una vez obtenidos los valores completos de las tabalas de datos anteriores,porocdimos a graficas los puntos obtenidos en un plano de Esfuerzo contra Deformaci\'{o}n.
\\
\begin{figure 4}
\centering
\includegraphics[width=8cm]{../../../../../Pictures/SDDDFSFASF.jpg} 
\\
\caption{Gr\'{a}fico de dispersi\'{o}n para el M\'{e}todo \'{O}ptico.}
\\
\end{figure 4}
\\
\begin{figure 5}
\centering
\includegraphics[width=8cm]{../../../../../Pictures/rrrrrrrrrrrrr.jpg} 
\\
\caption{Gr\'{a}fico de dispersi\'{o}n para el M\'{e}todo de nivel.}
\\
\end{figure 5}

\pagebreak

\section*{5. Ajuste de datos.}\\
\\
Por el Apr\'{e}ndice 1 que se encuentra en la bit\'{a}cora de trabajo (Referencia 1) podemos hacer el respectivo ajuste por el m\'{e}todo de m\'{i}nimos cuadrados para cada uno de los datos de cada experimento y cuyas tablas de entrada son:
\\
\medskip 
\medskip 
\begin{figure 6}
\centering
\\
\begin{tabular}{|c|c|c|c|c|}
\hline 
$\sum _{ i=1 }^{ n }{ { x }_{ i } } $(u) & $\sum _{ i=1 }^{ n }{ { y }_{ i } } $ (Pa) & $ \sum _{ i=1 }^{ n }{ { y }_{ i } } { x }_{ i }$ (Pa) & $\sum _{ i=1 }^{ n }{ { x }_{ i }^{ 2 } }({ u }^{ 2 })$ & n \\ 

\hline 
$1.541\times { 10 }^{ -2 }$& $1.837\times { 10 }^{ 9 }$ & $4.193\times { 10 }^{ 8 }$ & $2.449\times { 10 }^{ -5 }$& 13 \\ 
\hline 
\end{tabular} 
\\
\caption{Tabla de entrada 1.}
\\
\medskip  
\medskip 
\centering
\begin{tabular}{|c|c|c|c|c|}
\hline 
$\sum _{ i=1 }^{ n }{ { x }_{ i } } $(u) & $\sum _{ i=1 }^{ n }{ { y }_{ i } } $ (Pa) & $ \sum _{ i=1 }^{ n }{ { y }_{ i } } { x }_{ i }$ (Pa) & $\sum _{ i=1 }^{ n }{ { x }_{ i }^{ 2 } }({ u }^{ 2 })$ & n \\ 
\hline 
$1.445\times { 10 }^{ -2 }$& 4,590,874.98 & 6,752,629 & $2.166\times { 10 }^{ -5 }u$& 11 \\ 
\hline 
\end{tabular} 
\\
\caption{Tabla de entrada 2.}
\\
\end{figure 6}
\\

\medskip
\textbf{Modelos Propuestos (calculados):}
\[{ Y }_{ 1 }= -(788\times { 10 }^{ 10 })+(6.67\times { 10 }^{ 13 })x\quad \rightarrow(1) \]
\[{ Y }_{ 2 }= (63,523.440)+(2.6\times { 10 }^{ 8 })x\rightarrow(2)\]
\\
Por lo tanto, de (1) y (2) podemos afirmar que los m\'{o}dulos de Young son de ${ Y }_{ 1 }=6.67\times { 10 }^{ 13 }Pa$ y de ${ Y }_{ 2 }=2.6\times { 10 }^{ 8 }Pa$ para respectivo material y para el respectivo m\'{e}todo, ya sea el \'{o}ptico o de nivel. Cabe restaltar que estos valores fueron calculados por el m\'{e}todo de los m\'{i}nimos cuadrados.\\
\\
\medskip
\textbf{Gr\'{a}ficos de los modelos propuestos(calculados):}
\\
\begin{figure 7}
\centering
\includegraphics[width=7cm]{../../../../../Pictures/ggrsfa.jpg} 
\\
\caption{Gr\'{a}fico del modelo propuesto para el experimento con m\'{e}todo \'{o}ptico (calculado).}
\\
\centering
\includegraphics[width=7cm]{../../../../../Pictures/hhhhhhhhhhh.jpg} 
\\
\caption{Gr\'{a}fico del modelo propuesto para el experimento con m\'{e}todo de nivel (calculado).}
\\
\end{figure 7}\\
\medskip
\textbf{Modelos Propuestos (Por medio de Excel):}
\\
\[{ y }_{ 1 }=(2\times { 10 }^{ 7 })+(1\times { 10 }^{ 11 })x\rightarrow(3)\]
\[{ y }_{ 2 }=(66685)+(3\times { 10 }^{ 8 })x\rightarrow(4)\]
\\
Por lo tanto, de (3) y (4) podemos afirmar que los m\'{o}dulos de Young son de ${ Y }_{ 1 }=1\times { 10 }^{ 11 }Pa$ y de ${ Y }_{ 2 }=3\times { 10 }^{ 8 }Pa$ para respectivo material y para el respectivo m\'{e}todo, ya sea el \'{o}ptico o el de nivel. Cabe restaltar que estos valores fueron obtenidos por medio de un ajuste lineal de Excel.\\
\pagebreak
\\
\textbf{Gr\'{a}ficos de los modelos propuestos (Por medio de Excel):}
\\
\begin{figure 8}
\centering
\includegraphics[width=8cm]{../../../../../Pictures/KKKKKKKKKKK.jpg} 
\\
\caption{Gr\'{a}fico del modelo propuesto para el experimento con m\'{e}todo \'{o}ptico (Por medio de Excel).}
\\
\centering
\includegraphics[width=8cm]{../../../../../Pictures/GGGGGGGGG.jpg} 
\\
\caption{Gr\'{a}fico del modelo propuesto para el experimento con m\'{e}todo de nivel (Por medio de Excel).}
\\
\end{figure 8}\\
\medskip
\textbf{M\'{e}todo de Nivel. Procedimiento:}\\ 
Comenzamos el segundo m\'{e}todo para encontrar el m\'{o}dulo de Young con el material proporcionado por el equipo de laboratorio, el cual, en otro soporte se super pon\'{i}a un medidor de con una burbuja de nivel y as\'{i} medir la inclinaci\'{o}n, el cual primero medimos las constates.  (Figura 5).\\
\begin{figure}[hbtp]
\centering
\includegraphics[width=4cm]{../../../../../Pictures/IMG_20170207_132919315.jpg}
\caption{ M\'{e}todo por medida de nivel.}
\end{figure}
\\
Colocamos una masa en el alambre y registramos el avance del tornillo nivelandolo de nuevo. Repetimos el punto anterior de 12 a 15 veces. Llenamos la Tabla 2 y graficamos $\varepsilon$ vs DU y ajustamos por el m\'{e}todo de los minimos cuadrados. \\
\begin{figure}[hbtp]
\centering
\includegraphics[width=7cm]{../../../../../Pictures/IMG_20170207_132923819.jpg}
\caption{Vista lateral del principio de funcionamiento para el m\'{e}todo de nivel.}
\end{figure}

\end{document}