\documentclass[10pt,a4paper]{article}
\usepackage[latin1]{inputenc}
\usepackage[spanish]{babel}
\usepackage[utf8]{inputenc}
\usepackage{amsmath}
\usepackage{amsfonts}
\usepackage{amssymb}
\usepackage{graphicx}
\usepackage[left=2cm,right=2cm,top=2cm,bottom=2cm]{geometry}
\begin{document}
\section*{Desarrollo Experimental.}\\
\textbf{Lista de Materiales:} \\
Cuerda.\\
Generador de ondas (vibrador).\\
Portapesas.\\
Polea.\\
Pesas peque\~{n}as de 50, 100 o 200 gramos.\\

\section*{Arreglo experimenal.}
\begin{figure}[hbtp]
\centering
\\
\includegraphics[width=10cm]{../../../../../../Pictures/IMG_20170321_130926494.jpg} \\

\includegraphics[width=7.5cm]{../../../../../../Pictures/IMG_20170321_121220571_HDR.jpg} 
\includegraphics[width=7.5cm]{../../../../../../Pictures/IMG_20170321_124131096.jpg} 
\caption{Arreglo general. }
\end{figure} 
\\

\section*{Procedimiento y Resultados.}
1.- Se midieron las constantes l la cual da valor a la longitud inicial de la cuerda, de esto, l = 186.5 cm, se procedi\'{o} a calcular la masa m = 0.825 g de esta misma cuerda y resulta que con estos dos resultados se calcula la densidad lineal $\mu$ (=m/l).\\
2.-	Se conect\'{o} la cuerda al generador de ondas y en el portapesas se coloc\'{o} la menor cantidad de peso para generar las ondas y formar l\'{o}bulos, esto, de manera que se apreciara lo m\'{a}s posible, y as\'{i}, subir gradualmente el peso hasta llegar al mayor n\'{u}mero de l\'{o}bulos posibles. M\'{a}s expl\'{i}citamente, mientras se hac\'{i}a cada medici\'{o}n se aumentar\'{i}a el peso para incrementar la tensi\'{o}n para obtener un n\'{u}mero de l\'{o}bulo igual al anterior menos uno.\\
3.- Mientras se llevaba a cabo el experimento se llenaron las tablas para graficar la tensi\'{o}n (T) y la longitud de onda ($ \lambda$), obtenidas experimentalmente para ajustar la frecuencia de onda (f) y comparar con la frecuencia de la toma.\\
\medskip
\section*{Resultados.}
\\
\medskip
\begin{figure 2}
\caption{\textbf{Tabla 1.}}
\\
\centering
\includegraphics[width=7cm]{../../../../../../Pictures/1111111111111111111111111111.jpg} 
\\
\end{figure 2}
\\
\medskip
La tensi\'{o}n se obtiene del producto de la masa de las pesas y la gravedad, los nodos son los que formaba cada medici\'{o}n que depend\'{i}a de la tensi\'{o}n.\\
\medskip
\\
\\
\begin{figure 3}
\caption{\textbf{Tabla 2.}}
\\
\centering
\includegraphics[width=7cm]{../../../../../../Pictures/2222222222222222222222222222222222.jpg} 
\\
\caption{Se agreg\'{o} la constante $\mu$ que describe la densidad lineal de la cuerda, de este dato se obtendr\'{a} la frecuencia experimental mediante la ecuaci\'{o}n: $v\quad =\frac { 1 }{ \lambda  } \sqrt { \frac { T }{ \mu  }  } $. La cual describe la frecuencia de las ondas que en cuyo caso debe ser igual que la que se hab\'{i}a establecido en el generador, esto es, 65 Hz.}
\end{figure 3}
\\
\\
\medskip
\begin{figure 2}
\caption{\textbf{Tabla 3.}}
\\
\centering
\includegraphics[width=7cm]{../../../../../../Pictures/33333333333333333.jpg} 
\\
\caption{Una vez ya hechos los c\'{a}lculos se obtuvo la tabla 3 que en el quinto y sexto valor se acerca m\'{a}s al valor que se hab\'{a}a establecido en el generador. A continuaci\'{o}n se presenta un gr\'{a}fico que representa los datos en esta tabla.}
\end{figure 2}
\\
\medskip
\section*{Gr\'{a}fica de dispersi\'{o}n}
\\
\\
\begin{figure 5}
\centering
\includegraphics[width=8cm]{../../../../../../Pictures/Graficadedeisp.jpg} 
\\
\caption{Gr\'{a}ficos de dispersi\'o}n del lugar geom\'{e}trico de la tensi\'{o}n y la longitud de onda.}
\end{figure 5}
\\ 
\medskip
\section*{Ajuste de datos.}\\

Se observa que los puntos experimentales siguen un patr\'{o}n “lineal”, entonces para ajustarlo a la mejor ecuaci\'{o}n ocuparemos un modelo exponencial ajustado por m\'{i}nimos cuadrados, esto es:
\[ y=b{ x }^{ a }.\]
Donde b y a son los par\'{a}metros buscados dados por:
\[a=\frac { n\sum _{ i=1 }^{ n }{ { x }_{ i }ln{ y }_{ I } } -\sum _{ i=1 }^{ n }{ ln{ y }_{ i } } \sum _{ i=1 }^{ n }{ { x }_{ i } }  }{ n\sum _{ i=1 }^{ n }{ { x }_{ i }^{ 2 } } (n)-\sum _{ i=1 }^{ n }{ { \left( { x }_{ i } \right)  }^{ 2 } }  } \]
\[ln(b)=\frac { \sum _{ i=1 }^{ n }{ { x }_{ i }ln({ y }_{ i })- } \sum _{ i=1 }^{ n }{ ln({ y }_{ i })\sum _{ i=1 }^{ n }{ { x }_{ i } }  }  }{ \sum _{ i=1 }^{ n }{ { x }_{ i }^{ 2 }(n)\quad -\sum _{ i=1 }^{ n }{ { \left( { x }_{ i } \right)  }^{ 2 } }  }  } . \]
\\
Donde la tabla 3 queda de la siguiente manera:
\\
\medskip
\begin{figure 2}
\caption{\textbf{Tabla 4.}}
\\
\centering
\includegraphics[width=6cm]{../../../../../../Pictures/44444444444444444444444444444.jpg} 
\\
\end{figure 2}
\\
\medskip
\medskip
\\
\textbf{Tabla de entrada 1.}
\\
\begin{figure 6}
\centering
\begin{tabular}{|c|c|c|c|c|c|}
\hline 
n&$\sum _{ i=1 }^{ n }{ { x }_{ i } } (m)$ & $\sum _{ i=1 }^{ n }{ { ln y }_{ i } }(1/s)$ & $ \sum _{ i=1 }^{ n }{ {ln y }_{ i } } { x }_{ i }(m/s)$ & $\sum _{ i=1 }^{ n }{ { x }_{ i }^{ 2 } }({ m }^{ 2 })$& $\sum _{ i=1 }^{ n }{ { \left( { x }_{ i } \right)  }^{ 2 }{ (m) }^{ 2 } }$ \\ 
\hline 
6&$ -6.2671\times { 10 }^{ -1 } $& $2.5890\times { 10 }^{ 1 }$ & -3.0015 & 1.1489 & $3.9277\times { 10 }^{ -1 }$ \\ 
\hline 
\end{tabular}
\end{figure 6} 
\\
Sustituyendo los valores queda:

\[ a=\frac { (6\times (-3.0015)\left[ { m }/{ s } \right] )-(2.589\times { 10 }^{ 1 }\times (-6.2671\times { 10 }^{ -1 })\left[ { m }/{ s } \right] ) }{ (6\times 1.1489)\left[ { m }^{ 2 } \right] -(3.9277\times { 10 }^{ -1 })\left[ { m }^{ 2 } \right]  } =-2.7437\times { 10 }^{ -1 }\left[ { 1 }/{ ms } \right] .\]

\[ ln(b)=\frac { \left( 1.1489\times 2.589\times { 10 }^{ 1 } \right) \left[ m^{ 2 }/s \right] -\left( (-6.2671\times { 10 }^{ -1 })(-3.0015) \right) \left[ m^{ 2 }/s \right]  }{ (6\times 1.1489)\left[ m^{ 2 } \right] -(3.9277\times { 10 }^{ -1 })\left[ m^{ 2 } \right]  } =4.2863\left[ { s }^{ -1 } \right] .\]
Y entonces 
\[ b=72.6969\left[ { s }^{ -1 } \right].\]
Finalmente queda el modelos propuesto:
\[ Y=72.6969x^{ -0.27437 }\left[ { s }^{ -1 } \right]  \quad \rightarrow(1) \]
\medskip
\\
\section*{Ajuste por excel.}\\
Notemos que mientras por medio de c\'{a}alculos pudimos enconrar un modoelo, el programa Excel pudo enontrar otro, es cual se muestra a continuaci\'{o}n.
\\
\begin{figure}[hbtp]
 \centering
\includegraphics[width=8cm]{../../../../../../Pictures/exel.jpg}  
\end{figure}
\\


\section*{Error Porcentual.}\\

\[{ E }rror\quad porcentual =\frac { Valor\quad verdadero - Valor\quad aproximado }{ Valor\quad verdadero } \times 100= |(65-72.6969)/65|=0.1184*100 \% = 11.84\%. \]

\\
 \section*{Discusiones.}\\
Con \'{e}ste laboratorio se logr\'{o} entender el comportamiento de una onda estacionaria sobre en un medio, como una cuerda con ciertas caracter\'{i}sticas, se pudo determinar la longitud de la onda, la densidad lineal y la velocidad de probaci\'{o}n de la onda, por medio del amplificador de potencia se registraron los datos necesarios que permitieron desde la teor\'{i}a afirmarlos Es evidente que hubo errores en la medici\'{o}n los cuales escribir\'{e} en base que se propagaba diferente las ondas en cada tensi\'{o}n cuando se agregaban las pesas pues variaba en funci\'{o}n a la tensi\'{o}n que se aumentara, quiere decir que, era complicado comparar los datos y tratar de ajustar la medici\'{o}n a que se viera mejor. En cada una de las gr\'{a}ficas se comprob\'{o} la disminuci\'{o}n de la longitud de onda a medida que la frecuencia aumentaba.


\end{document}