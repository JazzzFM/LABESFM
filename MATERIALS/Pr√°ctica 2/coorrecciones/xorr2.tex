\documentclass[10pt,a4paper]{article}
\usepackage[latin1]{inputenc}
\usepackage[spanish]{babel}
\usepackage{amsmath}
\usepackage{amsfonts}
\usepackage{amssymb}
\usepackage{graphicx}
\usepackage[left=2cm,right=2cm,top=2cm,bottom=2cm]{geometry}
\begin{document}

\section*{2.1 Arreglo experimenal. Parte 1: M\'{e}todo de Torca o Est\'{a}tico.} \\

\begin{figure}[hbtp]
\centering
\includegraphics[width=10cm]{../../../../../../Pictures/arreglogeneralfwfsdf.jpg} 
\caption{Arreglo general.}
\end{figure}

\begin{figure}[hbtp]
\centering
\includegraphics[width=6cm]{../../../../../../Pictures/IMG_20170214_125326885.jpg}  
\caption{Vista lateral del arreglo}
\end{figure}

\pagebreak 

\textbf{Procedimiento.}\\
\\
1.-Calibramos el intrumento de medici\'{o}n.\\
2.-Medimos las constantes (X1,X2, X3, X4, rb, ra).\\
3. Colocar masas en el gancho.\\
4.-Medir el desplazamiento angular.\\
5.-LLenar la tabla con 10 a 12 mediciones.\\

\textbf{Resultados.}\\
\\
Comenzamos la pr\'{a}ctica con ell material proporcionado por el equipo de laboratorio, procedimos a medir algunas de nuestras constantes como lo son; la longitud de la barra, el radio de la polea.Llenamos la Tabla 1proporcionada para despu\'{e}s graficar la torca vs desplazamiento angular de cada mismo \'{a}ngulo de la barra y ajustar para calcular el G de cada muestra.\\
\\
\medskip
\textbf{Tabla 1.}
\\
\begin{figure 2}
\centering
\includegraphics[width=18cm]{../../../../../../Pictures/tatatatatatata.jpg} 
\end{figure 2}
\\
\\
De la cual extragimos los datos, del primer \'{a}nguo de la siguiente tabla para poder hacer la gr\'{a}ficatorca vs desplazamiento angular.
\\
\begin{figure 3}
\centering
\includegraphics[width=4.5cm]{../../../../../../Pictures/WW11.jpg}  
\includegraphics[width=4.5cm]{../../../../../../Pictures/WW21.jpg} 
\\
\caption{\textbf{Tabla 2.}  Datos tabulados para graficar.}
\end{figure 3}
\\
\section*{Gr\'{a}fica de dispersi\'{o}n}
\\
\\
\begin{figure 5}
\centering
\includegraphics[width=7cm]{../../../../../../Pictures/DESPEWW1.jpg} 
\includegraphics[width=7cm]{../../../../../../Pictures/DESPEWW3.jpg} 
\\
\caption{Gr\'{a}fico de dispersi\'o}n para el m\'{e}todo de la torca.}
\end{figure 5}
\\

\section*{Ajuste de datos.}\\
Por el Apr\'{e}ndice 1 podemos hacer el respectivo ajuste por el m\'{e}todo de m\'{i}nimos cuadrados para encontrar un modelo lineal ${Y}_{1}={a}_{1}x +{b}_{1}$ tales que $\left( { x }_{ r },{ y }_{ r } \right) \rightarrow \left( \theta(grad),{k}_{1} (Nm/grad))$  tambi\'{e}n un modelo lineal ${Y}_{2}={a}_{2}x +{b}_{2}$ tales que $\left( { x }_{ s },{ y }_{ s } \right) \rightarrow \left( \theta(grad), {M}_{1}(Pa))$y para cada uno de los datos de cada experimento, cuyas tablas de entrada y modelos son los siguientes:
\medskip
\\
\textbf{Tabla de entrada 1.}
\\
\begin{figure 6}
\centering
\begin{tabular}{|c|c|c|c|c|}
\hline 
n&$\sum _{ r=1 }^{ n }{ { x }_{ r } } $(grad) & $\sum _{ r=1 }^{ n }{ { y }_{ r } }$(Nm/grad) & $ \sum _{ r=1 }^{ n }{ { y }_{ r } } { x }_{ r }$ (Nm) & $\sum _{ r=1 }^{ n }{ { x }_{ r }^{ 2 } }({ grad }^{ 2 })$ \\ 

\hline 
12&$5.6\times { 10 }^{ 1 }$& $1.1\times { 10 }^{ 0 }$ & $5.36\times { 10 }^{ 0 }$ & $3.08\times { 10 }^{ 2 }$\\ 
\hline  
\end{tabular}
\end{figure 6} 
\\
\\
De donde:
\[{a}_{1}=\frac { n\sum _{ r=1 }^{ n }{ { x }_{ r }{ y }_{ r } } -\sum _{ r=1 }^{ n }{ { x }_{ r } } \sum _{ r=1 }^{ n }{ { y }_{ r } }  }{ n\sum _{ r=1 }^{ n }{ { x }_{ r }^{ 2 } } -{ \left( \sum _{ r=1 }^{ n }{ { x }_{ r } }  \right)  }^{ 2 } } \quad y\quad {b}_{1}=\frac { \sum _{ r=1 }^{ n }{ { x }_{ r }^{ 2 } } \sum _{ r=1 }^{ n }{ { y }_{ r } } -\sum _{ r=1 }^{ n }{ { x }_{ r }{ y }_{ r } } \sum _{ r=1 }^{ n }{ { x }_{ r } }  }{ n\sum _{ r=1 }^{ n }{ { x }_{ r }^{ 2 } } -{ \left( \sum _{ r=1 }^{ n }{ { x }_{ r } }  \right)  }^{ 2 } }.\]
Sustituyendo los valores queda:

\[{a}_{1}=\frac { 12\times5.36\times { 10 }^{ 0 }-\left[5.6\times { 10 }^{ 1 }\times 1.1\times { 10 }^{ 0 }  \right]  }{ 12\times 3.08\times { 10 }^{ 2 }-{ \left( 5.6\times { 10 }^{ 1 } \right)  }^{ 2 } } = { 4.1\times 10 }^{ -3 } (Nm/{grad}^{ 2 }).\]

\[{b}_{2}=\frac { 3.08\times { 10 }^{ 2 } \times 1.1\times { 10 }^{ 0 } -\left[ 5.36\times { 10 }^{ 0 } \times 5.6\times { 10 }^{ 1 } \right]  }{ 12\times 3.08\times { 10 }^{ 2 }-{ \left( 5.6\times { 10 }^{ 1 } \right)  }^{ 2 } } = { 7.24\times { 10 }^{ -2 } (Nm/grad).\]
Finalmente queda el modelos propuesto:
\[{ Y }_{ 1 }= { 4.1\times 10 }^{ -3 }x + 7.24\times { 10 }^{ -2 }  \quad \rightarrow(1) \]
\\
\textbf{Tabla de entrada 2.}
\\
\begin{figure 7}
\centering
\begin{tabular}{|c|c|c|c|c|}
\hline 
n&$\sum _{ s=1 }^{ n }{ { x }_{ s } } $(grad) & $\sum _{ s=1 }^{ n }{ { y }_{ s } }$(Pa) & $ \sum _{ s=1 }^{ n }{ { y }_{ s } } { x }_{ s }\quad(Pa\cdot grad)$ & $\sum _{ s=1 }^{ n }{ { x }_{ s }^{ 2 } }({ grad }^{ 2 })$ \\ 

\hline 
12&$5.6\times { 10 }^{ 1 }$& $5.39\times { 10 }^{ 11 }$ & $2.61\times { 10 }^{ 12 }$ & $3.08\times { 10 }^{ 2 }$\\ 
\hline 
\end{tabular}
\end{figure 7} 
\\
\\
De donde:
\[{a}_{2}=\frac { n\sum _{ s=1 }^{ n }{ { x }_{ s }{ y }_{ s } } -\sum _{ s=1 }^{ n }{ { x }_{ s } } \sum _{ s=1 }^{ n }{ { y }_{ s } }  }{ n\sum _{ s=1 }^{ n }{ { x }_{ s }^{ 2 } } -{ \left( \sum _{ s=1 }^{ n }{ { x }_{ s } }  \right)  }^{ 2 } } \quad y\quad {b}_{2}=\frac { \sum _{ s=1 }^{ n }{ { x }_{ s }^{ 2 } } \sum _{ s=1 }^{ n }{ { y }_{ s } } -\sum _{ s=1 }^{ n }{ { x }_{ s }{ y }_{ s } } \sum _{ s=1 }^{ n }{ { x }_{ s } }  }{ n\sum _{ s=1 }^{ n }{ { x }_{ s }^{ 2 } } -{ \left( \sum _{ s=1 }^{ n }{ { x }_{ s } }  \right)  }^{ 2 } }.\]
Sustituyendo los valores queda:

\[{a}_{2}=\frac { 12\times 2.61\times { 10 }^{ 12 } -\left[ 5.6\times { 10 }^{ 1 }\times 5.39\times { 10 }^{ 11 } \right]  }{{ 12\times 3.08\times { 10 }^{ 2 }-{ \left( 5.6\times { 10 }^{ 1 } \right)  }^{ 2 } } ={ 2\times 10 }^{ 9 } (Pa/grad).\]

\[{b}_{2}=\frac { 3.08\times { 10 }^{ 2 } \times5.39\times { 10 }^{ 11 }-\left[ 2.61\times { 10 }^{ 12 } \times 5.6\times { 10 }^{ 1 } \right]  }{ 11\times 7.66\times { 10 }^{ 3 }-{ \left( 1.952\times { 10 }^{ 2 } \right)  }^{ 2 } } = { 4.03\times  10 }^{ 10 } (Pa).\]
Finalmente queda el modelos propuesto:
\[{ Y }_{ 2 }= { 2\times 10 }^{ 9 }x + { 4.03\times  10 }^{ 10 }  \quad \rightarrow(2 ). \]
\\
De (1) y (2) podemos graficar sus correspondientes lineas de tendencia, a continuaci\'{o}n vamos a graficar su modelo y a tratarlo con m\'{a}s detalle.
\\
\section*{Ajuste por excel.}\\
Notemos que mientras por medio de c\'{a}alculos pudimos enconrar un modoelo, el programa Excel pudo enontrar otro, es cual se muestra a continuaci\'{o}n.
\\
\begin{figure}[hbtp]
 \centering
\includegraphics[width=8cm]{../../../../../../Pictures/EXCLE111.jpg} 
\includegraphics[width=8.2cm]{../../../../../../Pictures/EXCLE22222.jpg}  
\end{figure}
\\
Las ecuaciones para la linea punteada de color azul de la figura de la izquierda y derecha respectivamente son ${ Y }_{ 1 }= { 8.43\times 10 }^{ -4 }x+ 1.0367$ y ${ Y }_{ 2 }= { 3.067\times 10 }^{ 7 }x+ { 4.035\times  10 }^{ 10 }$, claramente se puede observar que en las g\'{a}ficas la linea de tendencia no es muy inclinada, es decir, su funci\'{o}n no da un cambio muy grande a lo largo de su dominio, as\'{i} que por simplificaci\'{o}n tomamos su ecuacion como una funci\'{o}n constante, proporcionada por el mismo modelo linal, que ser\'{a} la linea de rojo cuyo valor est\'{a} dado por la intersecciones con el eje Y y que est\'{a} marcado en un circulo rojo. Y as\'{i} tenemos que ${ Y }'_{ 1 }= 1.02$ y ${ Y }'_{ 2 }=4.07\times { 10 }^{ 10 }$. Al aplicar un momento torsional M en el extremo inferior de la barra, \'{e}ste experimenta una deformaci\'{o}n de torsi\'{o}n. Dentro de los l\'{i}mites de validez de la ley de Hooke, el \'{a}ngulo de torsi\'{o}n $\theta$ es directamente proporcional al momento torsional M aplicado, de modo que:
\[{ k }_{ 1 }= 1.0(Nm/grad)\left( { grad }/{ \left( { 1 }/{ 2\pi rad } \right)  } \right)=2.04(Nm/rad)\quad y \quad { M }_{ 1 }=4.07\times { 10 }^{ 10 }(Pa).\]
\\
\section*{Error Porcentual.}\\
Los valores verdaderos (Taba 1 del marco te\'{o}rico) de los m\'{o}dulos de corte o cizalladura del cobre es ${ M }_{ 1 }=4.4\times { 10 }^{ 10 }Pa$ y su  coeficiente de torsi\'{o}n o m\'{o}dulo el\'{a}stico de torsi\'{o}n es de ${ k }_{ 1 }=1.8(Nm/rad)$. Entonces, de nuestras mediciones y c\'{a}lculos podemos obtener el error porcentual:

\[{ E }rror\quad porcentual\quad { M }_{ 1 }=\frac { Error\quad verdadero }{ Valor\quad Verdadero } =\frac { Valor\quad verdadero - Valor\quad aproximado }{ Valor\quad verdadero } \times 100= 34\%. \]

\[{ E }rror\quad porcentual\quad { k }_{ 1 }=\frac { Error\quad verdadero }{ Valor\quad Verdadero } =\frac { Valor\quad verdadero - Valor\quad aproximado }{ Valor\quad verdadero } \times 100= 46\%. \]
\\
Para encontrar un modelo lineal $Y=ax +b$ tales que $\left( { x }_{ i },{ y }_{ i } \right) \rightarrow \left( {\theta}_{ 1 }(grad), M (Nm) \right) $ para cada uno de los datos de cada experimento y cuya tabla de entrada es:
\medski\\
\caption{Tabla de entrada 2.}
\\
\begin{figure 6}
\\
\medskip  
\medskip 
\centering
\begin{tabular}{|c|c|c|c|c|}
\hline 
$\sum _{ i=1 }^{ n }{ { x }_{ i } } $(grad) & $\sum _{ i=1 }^{ n }{ { y }_{ i } } $ (Nm) & $ \sum _{ i=1 }^{ n }{ { y }_{ i } } { x }_{ i }$ (Nm grad) & $\sum _{ i=1 }^{ n }{ { x }_{ i }^{ 2 } }(grad)$ & n \\ 
\hline 
$3.5\times { 10 }^{ 1 }$& 5.45 & 19.7 & $1.29\times { 10 }^{ 2 }u$& 12 \\ 
\hline 
\end{tabular} 
\end{figure 6}
\\

\end{figure 6} 
\\
\\
De donde:
\[a=\frac { n\sum _{ i=1 }^{ n }{ { x }_{ i }{ y }_{ i } } -\sum _{ i=1 }^{ n }{ { x }_{ i } } \sum _{ i=1 }^{ n }{ { y }_{ i } }  }{ n\sum _{ i=1 }^{ n }{ { x }_{ i }^{ 2 } } -{ \left( \sum _{ i=1 }^{ n }{ { x }_{ i } }  \right)  }^{ 2 } } \quad y\quad b=\frac { \sum _{ i=1 }^{ n }{ { x }_{ i }^{ 2 } } \sum _{ i=1 }^{ n }{ { y }_{ i } } -\sum _{ i=1 }^{ n }{ { x }_{ i }{ y }_{ i } } \sum _{ i=1 }^{ n }{ { x }_{ i } }  }{ n\sum _{ i=1 }^{ n }{ { x }_{ i }^{ 2 } } -{ \left( \sum _{ i=1 }^{ n }{ { x }_{ i } }  \right)  }^{ 2 } }.\]
Sustituyendo los valores queda:

\[a=\frac { 12\times 19.7-\left[ \3.5\times { 10 }^{ 1 }\times  5.45\right]  }{ 12\times 1.29\times { 10 }^{ 2 }-{ \left( 3.5\times { 10 } \right)  }^{ 2 } } ={ 0.08715 (grad/Nm).\]

\[b=\frac { 1.29\times { 10 }^{ 2 }\times 54.5-\left[ 19.7\times 35 \right]  }{ 12\times 1.29\times { 10 }^{ 2 }-{ \left( 3.5\times { 10 } \right)  }^{ 2 } } ={ 0.04195 (Nm).\]
Finalmente queda el modelos propuesto:
\[{ M }_{ 1 }=  0.08715x (grad/Nm) + 0.04195 (Nm) \rightarrow(2) \]

Como la pendiente de la recta a tangente a la curva misma nos representa, por la f\'{o}rmula de torsi\'{o}n:
\[ \frac { M c }{ I } =  { T }_{ max }\quad y \quad { T }_{ max }= G\theta\]
Y as\'{i}, podemos decir, por definici\'{o}n, que el m\'{o}duo de cizalladura  es: $G=7.69\times 10 }^{ 10 }(Pa).$
\\
Por lo anto el promedio de ambos queda { G }_{ prom } = $7.89\times 10 }^{ 10 }(Pa).$

\section*{Ajuste por excel.}\\
Notemos que mientras por medio de c\'{a}alculos pudimos enconrar un modelo, el programa Excel pudo enontrar otro, es cual se muestra a continuaci\'{o}n.

\begin{figure}[hbtp]
 \centering
\includegraphics[width=8.5cm]{../../../../../Pictures/EXCLE33333333.jpg} 
 \caption{Modelo por medio de excel. }
\end{figure}

\section*{Error Porcentual.}\\
Los valores verdaderos de los m\'{o}dulos de cizalladura del acero es ${ G }_{ 1 }=7.5\times { 10 }^{ 9 }Pa$. Entonces, de nuestras mediciones y c\'{a}lculos podemos obtener el error porcentual:

\[{ E }rror-porcentual-{ Y }_{ 1 }=\frac { Error\quad verdadero }{ Valor\quad Verdadero } =\frac { Valor\quad verdadero - Valor\quad aproximado }{ Valor\quad verdadero } \times 100=34 \% \]
\\
 
 \section*{Discusiones.}\\
 Al llevar acabo este experimento pudimos notar que las condiciones enlas que se hac\'{i}a no eran muy precisas, ya que nuestro medidor de desplazamiento \'{a}ngular d\'{i}gase comp\'{a}s, estaba colocado de una manera muy superficial y sin mayor soporte. A\'{u}n as\'{i} pudimos notar el fen\'{o}meno de torsi\'{o}n angular y cuando se involucra el esfuerzo de cizalladura, los \'{a}ngulos o desplazamiento angular variaba de manera casi constante con respecto a la torca, es decir, con respecto ala pocisi\'{o}n de la barra.
\end{document}