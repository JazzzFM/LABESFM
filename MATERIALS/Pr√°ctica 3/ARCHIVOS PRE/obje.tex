\documentclass[11pt,a4paper]{article}
\usepackage[latin1]{inputenc}
\usepackage[spanish]{babel}
\usepackage{amsmath}
\usepackage{amsfonts}
\usepackage{amssymb}
\usepackage{makeidx}
\usepackage{graphicx}
\usepackage[left=2cm,right=2cm,top=2cm,bottom=2cm]{geometry}
\author{Flores Rodguez Jaziel David }
\title{objetivos}
\begin{document}
\section*{OBJETIVOS.}
\\
\textbf{Objetivos particulares de la unidad.}\\
i)Entender algunas caracter\'{i}sticas de los fluidos (en especial l\'{i}quidos, tanto hidrost\'{a}ticas com din\'{a}minas.\\
ii)Efectuar algunos experimentos para el c\'{a}lculo de \'{e}stas propiedades, as\'{i} como variaci\'{o}n con la temperatura.\\
\\
\textbf{Objetivo de la pr\'a{}ctica.}\\
Explicaci\'{o}n te\'{o}rica del fen\'{o}meno de tensi\'{o}n superficial y manejo del tensi\'{o}metro de D'unouy. En la pr\'{a}ctica se utiliza dicho tensiometro para calcular el coeficiente de tensi\'{o}n superficial a diferentes temperaturas para diversos l\'{i}quidos.\\
\section{Marco Te\'{o}rico.}
\textbf{Tensi\'{o}n superficial} 
\\
Un objeto menos denso que el agua, como una pelota de playa inflada con aire, flota con una parte de su volumen bajo la superficie. Por otra parte, un clip puede descansar sobre una superficie de agua aunque su densidad es varias veces mayor que la del agua. Esto es un ejemplo de \textbf{tensi\'{o}n superficial}: la superficie del l\'{i}quido se comporta como una membrana en tensi\'{o}n (Figura 1).
\\
\begin{figure}[hbtp]
\centering
\includegraphics[width=4.5cm]{../../../../../Pictures/LALALAALALALLALA.jpg}
\caption{La superficie del agua act\'{u}a como membrana sometida a tensi\'{o}n, y permite a este insecto tejedor o zapatero de agua caminar literalmente sobre el agua}
\end{figure}
\\

La tensi\'{o}n superficial se debe a que las mol\'{e}culas del l\'{i}quido ejercen fuerzas de atracci\'{o}n entre s\'{i}. La fuerza neta sobre una mol\'{e}cula dentro del volumen del l\'{i}quido es cero, pero una mol\'{e}cula en la superficie es atra\'{i}da hacia el volumen (Figura 2).
\\
\begin{figure}[hbtp]
\centering
\includegraphics[width=4.5cm]{../../../../../Pictures/LALALALALALALALALALLALALA2.jpg}
\caption{Una mol\'{e}cula en la superficie es atra\'{i}da hacia el volumen del l\'{i}quido,y esto tiende a reducir el \'{a}rea superficial del l\'{i}quido.}
\end{figure}
\\
Por esa raz\'{o}n, el l\'{i}quido tiende a reducir al m\'{i}nimo su \'{a}rea superficial, tal como lo hace una membrana estirada. La tensi\'{o}n superficial explica por qu\'{e} las gotas de lluvia en ca\'{i}da libre son esf\'{e}ricas (no con forma de l\'{a}grima): una esfera tiene menor \'{a}rea superficial para un volumen dado que cualquier otra forma. Tambi\'{e}n explica por qu\'{e} se usa agua jabonosa caliente en el lavado de la ropa. Para lavarla bien, se debe hacer pasar el agua por los diminutos espacios entre las fibras (Figura 3).
\\
\begin{figure}[hbtp]
\centering
\includegraphics[width=5cm]{../../../../../Pictures/sskskskskskksks3.jpg}
\caption{La tensi\'{o}n superficial dificulta el paso del agua por aberturas peque\~{n}as. La presi\'{o}n requerida p del agua puede reducirse usando agua caliente con jab\'{o}n,lo que reduce la tensi\'{o}n superficial.}
\end{figure}
\\
\\
Esto implica aumentar el \'{a}rea superficial del agua, lo que es dif\'{i}cil por la tensi\'{o}n superficial. La tarea se facilita aumentando la temperatura del agua y a\~{n}adiendo jab\'{o}n, pues ambas cosas reducen la tensi\'{o}n superficial. La tensi\'{o}n superficial es importante para una gota de agua de 1 mm de di\'{a}metro, que tiene un \'{a}rea relativamente grande en comparaci\'{o}n con su volumen. (Una esfera de radio r tiene \'{a}rea $4\pi { r }^{ 2 }$y volumen ${ (4\pi  }/3)^{ 3 }$. La raz\'{o}n entre la superficie y el \'{a}rea es $3/r$, y aumenta al disminuir el radio.) En cambio, si la cantidad de l\'{i}quido es grande, la raz\'{o}n entre superficie y volumen es relativamente peque\~{n}a y la tensi\'{o}n superficial es insignificante en comparaci\'{o}n con las fuerzas de presi\'{o}n. En el resto del cap\'{i}tulo, s\'{o}lo consideraremos vol\'{u}menes grandes de fluidos, as\'{i} que ignoraremos los efectos de la tensi\'{o}n superficial. Ahora para veamos el caso de nuestro experimento, para incrementar la superficie de un l\'{i}quido se necesita efectuar cierta cantidad de trabajo, el cual resulta ser proporcional a este incremento:
\[dW=\gamma dA \longleftarrow (1)\]
\[\gamma= Coeficiente\quad de \quad tensi\'{o}n \quad superficial.\]
\\
\begin{figure}[hbtp]
\centering
\includegraphics[width=7cm]{../../../../../Pictures/TENSION.jpg} 
\end{figure}
\\
$dw=Fdx$ y $dA= 2ldx,$ por lo tanto desde que la pel\'{i}cula tiene dos caras $Fdx=\gamma(2ldx)$ o bien $F=2 \gamma l$. Al despegarse el arillo de la superficie de un l\'{i}quido, se forma un pequeño cilindro de radio \textbf{r} y altura \textbf{dh}, form\'{a}ndose as\'{i} una superficie: 
\[dA = 2(2 \pi r dh).\]
El cilindro teiene 2 caras, interna y externa. Y el trabajo efectuado en el despegue es:
\[dW=Fdh= \gamma 4 \pi r dh.\]
Y fialmente se tiene:
\[\gamma=\frac{F}{4 \pi r}.\]

\end{document}