\documentclass[10pt,a4paper]{article}
\usepackage[latin2]{inputenc}
\usepackage[spanish]{babel}
\usepackage{amsmath}
\usepackage{amsfonts}
\usepackage{amssymb}
\usepackage{makeidx}
\usepackage{graphicx}
\usepackage[left=2cm,right=2cm,top=2cm,bottom=2cm]{geometry}

\begin{document}

\section*{7. Conclusiones:}\\
En general, al realizar la pr\'{a}ctica pudimos observar que la propiedad intr\'{i}nseca de un fluido ll\'{a}mese densidad es una magniud en demas\'{i}a sensible en proporci\'{o}n a los jinetitos de la balanza, es decir pod\'{i}a tardar demasiado en lograr el equilibrio, cabe resaltar que esta balanza se basa en el principio de Arqu\'{i}mides, sin embargo existen dispositivos como el hridr\'{o}metro que se basanen el mismo fen\'{o}meno. Lo mencionamos porque los resultados variaron bastante como para ser tomados como oficiales, sin embargo se encontr\'{o} que el comportamiento esperado de la densidad es decir para su ecuaci\'{o}n de tendencia se econtr\'{o} una funci\'{o}n estrictamente decreciente .
\section*{8. Referencias:}\\
\\
\medskip
\\
\\1.- Bit\'{a}cora de laboratorio de Flores Rodr\'{i}guez Jaziel David.
\\
2.- Manual de pr\'{a}cticas auxilar. Autor: Fco. Havez Varela y las notas del profesor Salvador Tirado Guerra.
\\
3.- Fisica Universitaria - Sears - Zemansky - 12ava Edicion - Cap\'{i}tulo l4.
\end{document}