\documentclass[10pt,a4paper]{article}
\usepackage[latin2]{inputenc}
\usepackage[spanish]{babel}
\usepackage{amsmath}
\usepackage{amsfonts}
\usepackage{amssymb}
\usepackage{makeidx}
\usepackage{graphicx}
\usepackage[left=2cm,right=2cm,top=2cm,bottom=2cm]{geometry}

\begin{document}

\section*{7. Conclusiones:}\\
En general, al realizar la pr\'{a}ctica pudimos observar que la propiedad intr\'{i}nseca de cada material, ll\'{a}mese  M\'{o}dulo de Young, tambi\'{e}n pudimos observar que la maner cambiabar las pesas no era la adecuada por lo que generamos una interferencia en la deformaci\'{o}n unitaria modificando as\'{i} un fen\'{o}meno de naturalea m\'{a}s pl\'{a}stica. Cabe resaltar que estamos consientes que usamos cantidades o normas de ingenirer\'{i}a  ya que consideramos una \'{a}rea transversal constante pero esto no es del todo exacto, es decir, la elongaci\'{o}n conforme el tiempo avanza, y mientras se continue jalando, \'{e}sta aumenta, pero eso si el volumen es siempre const\'{a}nte. De qu\'{e} nos sirve esto? Pues yo creo que afecta porque el esfuerzo depende directamente del \'{a}rea transversal donde se aplica la fuerza y la fuera misma. Adem\'{a}s que el error porcentual es enorme, es decir que algo no se efectu\'{o} de manera correcta, sin embargo para el material de lat\'{o}n se aproximaba mucho al valor real, mientras que el de cobre quedaba muy lejos del valor real, esto puede ser por una mala medici\'{o}n, un c\'{a}lculo err\'{o}neo o un mal empleo del material de laboratorio.
 
\section*{8. Referencias:}\\
\\
\medskip
\\
\\1.- Bit\'{a}cora de laboratorio de Flores Rodr\'{i}guez Jaziel David.
\\
2.- Manual de pr\'{a}cticas auxilar. Autor: Fco. Havez Varela y las notas del profesor Salvador Tirado Guerra.
\\
3.- Fisica Universitaria - Sears - Zemansky - 12ava Edicion - Cap\'{i}tulo l1
\end{document}