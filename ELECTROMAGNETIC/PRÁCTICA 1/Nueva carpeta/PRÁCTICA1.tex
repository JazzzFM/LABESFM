\documentclass[11pt,a4paper]{article}
\usepackage[latin1]{inputenc}
\usepackage[spanish]{babel}
\usepackage{amsmath}
\usepackage{amsfonts}
\usepackage{amssymb}
\usepackage{makeidx}
\usepackage{graphicx}
\usepackage{lmodern}
\usepackage[left=2cm,right=2cm,top=2cm,bottom=2cm]{geometry}
\author{Flores Rodguez Jaziel David }
\title{Práctica 1}
\begin{document}

\tableofcontents
%----------------------------------------------------------------------------------------
%	SECTION 1
%----------------------------------------------------------------------------------------
\\
\medskip
\medskip
\section{Resumen.}
La balanza de torsi\'{o}n fue de suma importacnia en el estudio de la electrost\'{a}tica en la que Charles-Augustin de Coulomb se bas\'{o} para cuantificar, por primera vez en la historia, la fuerza entre cargas. Adem\'{a}s sabemos que actualmente es m\'{a}s dif\'{i}cil medir la carga directamente, as\'{i} que utilizamos el mult\'{i}metro, un aparato dise\~{n}ado para medir carga por unidad de tiempo que pasa por un cable conductor, estos disposiivos son muy importantes tanto en la industria como en la vidad diaria por el uso actual de las tecnolog\'{i}as de electr\'{o}nica, mientras que el amplificadr lineal sirve para poder sensar cargas cuya magnitud es muy peque\~{n}a y es dif\'{i}cil medir. Es por ello que en esta pr\'{a}ctica hemos calibrado estos instrumentos de medici\'{o}n, apartir de conceptos previamente estudiados, principalmente el \textbf{m\'{o}dulo de torsi\'{o}n} de un alambre o material en general, el cual lo hayamos de dos modos distintos y nos servir\'{a} para nuestra balanza de torsi\'{o}n. \\
\\
\medskip
Palabras clave: carga, cable conductor, m\'{o}dulo de torsi\'{o}n.

%----------------------------------------------------------------------------------------
%	SECTION 2
%----------------------------------------------------------------------------------------

\section{Objectivo.}
En esta actividad el alumno aprender\'{a} a manejar y calibrar, el amplificador lineal, el mult{i}metro y la balanza de torsi\'{o}n.

\pagebreak

%----------------------------------------------------------------------------------------
%	SECTION 3
%----------------------------------------------------------------------------------------
\section{Metodolog\'{i}a.}
La balanza de torsi\'{o}n es un dispositivo creado por el f\'{i}sico Charles-Augustin de Coulomb en el a\~{n}o 1777, con el objeto de medir fuerzas d\'{e}biles. Coulomb emple\'{o} la balanza para medir la fuerza electrost\'{a}tica entre dos cargas. Encontr\'{o} que la fuerza electrost\'{a}tica entre dos cargas puntuales es directamente proporcional al producto de las magnitudes de las cargas el\'{e}ctricas e inversamente al cuadrado de la distancia entre las cargas. Este descubrimiento se denomin\'{o} Ley de Coulomb.

\begin{figure}[hbtp]
\centering
\includegraphics[width=5cm]{../../../../../../Pictures/B-Coulomb1.jpg}
\caption{Balanza de torsi\'{o}n.}
\end{figure}
La balanza de torsi\'{o}n consiste en dos bolas de metal sujetas por los dos extremos de una barra suspendida por un cable, filamento o chapa delgada. Para medir la fuerza electrost\'{a}tica se puede poner una tercera bola cargada a una cierta distancia. Las dos bolas cargadas se repelen/atraen unas a otras, causando una torsi\'{o}n de un cierto \'{a}ngulo. De esta forma se puede saber cuanta fuerza, en newtons, es requerida para torsionar la balanza un cierto \'{a}ngulo. La balanza de torsi\'{o}n se emple\'{o} para definir inicialmente la unidad de carga electrost\'{a}tica y hoy en d\'{i}a se define como la carga que pasa por la secci\'{o}n de un cable cuando hay una corriente de un amperio durante un segundo de tiempo, la f\'{o}rmula para hacer esto es: $1 C = 1 As$. Un Culomb representa una carga aproximada de $6.241506X10^{18}$ e, siendo e la cantidad de carga que posee un electr\'{o}n.
Una balanza de torsi\'{o}n se emple\'{o} en el experimento de Cavendish realizado en 1798 para medir la densidad de la Tierra con la mayor precisi\'{o}n posible. Las balanzas de torsi\'{o}n se siguen empleando hoy en d\'{i}a en algunos experimentos de f\'{i}sica.\\
\\
\textbf{M\'{o}dulo de Torsi\'{o}n.}
\begin{figure}[hbtp]
\centering
\includegraphics[width=3.7cm]{../../../../../../Pictures/ddddddddddd.jpg}
\caption{Objeto sometido a un esfuerzo de corte. Se aplican fuerzas tangentes a superficies opuestas del objeto. Por
claridad, se exagera la deformaci\'{o}n x.}
\end{figure}
\\
Esfuerzo y deformaci\'{o}n por corte. La Figura 2 muestra un cuerpo deformado por un esfuerzo de corte. En la figura, fuerzas de igual magnitud pero direcci\'{o}n opuesta act\'{u}an de forma tangente a las superficies de extremos opuestos del objeto. Definimos el esfuerzo de corte como la fuerza $F\Vert$ que act\'{u}a tangente a la superficie, dividida entre el \'{a}rea A sobre la que act\'{u}a:
\[Esfuerzo \quad de \quad corte = \frac{F\Vert}{A}\longrightarrow (1)\]
Al igual que los otros dos tipos de esfuerzo, el esfuerzo de corte es una fuerza por unidad de \'{a}rea. La Figura 2
muestra que una cara del objeto sometido a esfuerzo de corte se desplaza una distancia x relativa a la cara opuesta. Definimos la \textbf{deformaci\'{o}n por corte} como el cociente del desplazamiento x entre la dimensi\'{o}n transversal h:
\[ Deformaci\'{o}n \quad por \quad corte= \frac{x}{h} \longrightarrow (2)\]
En situaciones reales, x casi siempre es mucho menor que h. Como todas las deformaciones, la deformaci\'{o}n por corte es un n\'{u}mero adimensional: un cociente de dos longitudes. Si las fuerzas son lo suficientemente peque\~{n}as como para que se obedezca la ley de Hooke, la deformaci\'{o}n por corte es proporcional al esfuerzo de corte. El m\'{o}dulo de elasticidad correspondiente(cociente del esfuerzo de corte entre la deformaci\'{o}n por corte) se denomina m\'{o}dulo de corte y se denota con S:
\[ S = \frac{ Esfuerzo \quad de \quad corte}{Deformaci\'{o}n \quad por \quad corte} = \frac{F/A}{x/h} = \frac{Fh}{Ax} \longrightarrow (3) \]
Para un material dado, S suele ser de un tercio a un medio del valor del m´odulo de Young Y para el esfuerzo de tensi\'{o}n. Tenga en cuenta que los conceptos de esfuerzo de corte, deformaci\'{o}n por corte y m\'{o}dulo de corte \'{u}nicamente se aplican a materiales s\'{o}lidos. La raz\'{o}n es que las fuerzas de corte deben deformar el bloque s\'{o}lido, el cual tiende a regresar a su forma original si se eliminan las fuerzas de corte. En cambio, los gases y l\'{i}quidos no tienen forma definida.\\
\\
\textbf{Mult\'{i}metro.} 
Un mult\'{i}metro, tambi\'{e}n denominado pol\'{i}metro, o tester, es un instrumento el\'{e}ctrico port\'{a}til para medir directamente magnitudes el\'{e}ctricas activas, como corrientes y potenciales (tensiones), o pasivas, como resistencias, capacidades y otras. Las medidas pueden realizarse para corriente continua o alterna y en varios m\'{a}rgenes de medida cada una. Los hay anal\'{o}gicos y posteriormente se han introducido los digitales cuya funci\'{o}n es la misma, con alguna variante a\~{n}adida.\\
\\
\textbf{Amplificador lineal.}
Amplificador electr\'{o}nico puede significar tanto un tipo de circuito electr\'{o}nico o etapa de este cuya funci\'{o}n es incrementar la intensidad de corriente, la tensi\'{o}n o la potencia de la se\~{n}al que se le aplica a su entrada; obteni\'{e}ndose la se\~{n}al aumentada a la salida. Para amplificar la potencia es necesario obtener la energ\'{i}a de una fuente de alimentaci\'{o}n externa. Es un dispositivo electr\'{o}nico que al acoplarlo con un mult\'{i}metro, es posible medir la carga el\'{e}ctrica de cuerpos cargados, tambi\'{e}n se puede medir la diferencia de potencial y la intensidad de corriente, en circuitos de corriente directa.\\
 \\
\textbf{CALIBRACI\'{O}N DE L BALANZA DE TORSI\'{O}N.}
\\
\textbf{M\'{e}todo est\'{a}tico}. Colocarla en posici\'{o}n de equilibrio y alinearla con el tetigo. 
\begin{figure}[hbtp]
\caption{Posici\'{o}n de equilibrio }
\centering
\includegraphics[width=6cm]{../../../../../../Pictures/EEHEEHEFE.jpg}
\end{figure}
\\
Al colocar la pesa pierde el equilibrio.
\begin{figure}[hbtp]
\centering
\includegraphics[width=5.3cm]{../../../../../../Pictures/sadadada.jpg}
\caption{Pierde la posici\'{o}n de equilibrio.}  
\end{figure}
\\
Se pierde el equilibrio por la torca que se produce, cuya funci\'{o}n es:
\[\tau=mgr \longrightarrow (4)\]
Se establece una nueva posici\'{o}n de equilibrio girando el cabezal un \'{a}nulo $\theta$.

\begin{figure}[hbtp]
\centering
\includegraphics[width=5.5cm]{../../../../../../Pictures/tres.jpg}
\caption{La nueva posici\'{o}n de equilibrio. }
\end{figure}
Se requiere $\tau_{t}=mgr$. Donde $\tau_{t}=K'\theta$ si denotamos por K' a la constante de toris\'{o}n de un alambre, entonces, se debe cumplir que: $K'\theta=mgr$. De donde $K'= \frac{mgr}{\theta}$. Como la balnaza incluye dos alambres entonces la constante de la balanza es K=2K', de donde: 
\[ K = \frac{2mgr}{\theta} \longrightarrow (5)\]
\textbf{M\'{e}todo Din\'{a}mico.} Sabemos que:
\[ \tau =I\alpha =I\frac { { \partial  }^{ 2 }\theta  }{ { \partial { t }^{ 2 } } } =-k\theta \longrightarrow (6)\]
La ecuaci\'{o}n de movimiento es:
\[\frac { { \partial  }^{ 2 }\theta  }{ { \partial { t }^{ 2 } } } +\frac { k }{ I } \theta =0 \longrightarrow (7) \]
Una ecuaci\'{o}n diferencial cuya soluci\'{o}n es: 
\[\theta ={ \theta  }_{ m }cos(\omega t+\delta ) \longrightarrow (8)\]
Donde: 
\[\omega =\sqrt { \frac { k }{ I }  } =\frac { 2\pi  }{ T } \longrightarrow (9) \]
Por lo que:
\[T=2\pi \sqrt { \frac { I }{ K }  } \longrightarrow (10)\]
Donde I es el momento de inercia de la barilla y K la constante de torsi\'{o}n de la balanza. 
\[I=\frac { M{ L }^{ 2 } }{ 12 } \longrightarrow (11) \]
\pagebreak

%----------------------------------------------------------------------------------------
%	SECTION 4
%----------------------------------------------------------------------------------------

\section{Instrumentaci\'{o}n.}\\
1- Balanza de torsi\'{o}ncon alambres de un material en espec\'{i}fico y una barilla uniforme.\\
2- Mult\'{i}metro.\\
3- Amplificador lineal. \\
4- Vernier. \\
5. Balanza para pesar.\\
6. Cron\'{o}metro.\\
7. Pesas relaivamente peque\~{n}as.\\

%----------------------------------------------------------------------------------------
%	SECTION 5
%----------------------------------------------------------------------------------------


\section{Datos y Resultados.}
Comenzamos la pr\'{a}ctica con ell material proporcionado por el equipo de laboratorio, procedimos a medir algunas de nuestras constantes como lo son; la masa de una pesita $m= 5\times { 10 }^{ -4 }\quad kg$, la masa de la barilla $M=57.8\times { 10 }^{ -3 }\quad kg$ el brazo de palanca sobre la barilla peque\~{n}a $r=5\times { 10 }^{ -2 }\quad m$ la longitud $L=2.4\times { 10 }^{ -1 }\quad m$de la barilla grande, por supuesto la gravedad que se tomar\'{a} como $g=9.81\quad \frac { m }{ { s }^{ 2 } }$, adem\'{a}s con estos datos sacamos el momento de inercia I utilizando tambi\'{e}n el radio $R=3.05\times { 10 }^{ -3 }\quad m$ de la barra.
\\
\\
\textbf{M\'{e}todo est\'{a}tico.}
\\
\\
\textbf{Tabla de entrada.}
\\
\begin{figure 6}
\centering
\begin{tabular}{|c|c|c|}
\hline 
n& $\theta_{g}$ (grad) & $\theta_{r}$ (rad)   \\ 
\hline 
1& 137.5 & 2.4  \\ 
\hline 
\end{tabular}
\end{figure 6} 
De la ecuaci\'{o}n (5), sustituyendo se sigue que:
\[ K = \frac {  2\left( 5\times { 10 }^{ -4 }\quad \left[ kg \right]  \right) \left( 9.81\left[ { m }/{ { s }^{ 2 } } \right]  \right) (5\times { 10 }^{ -2 }\quad \left[ m \right] ) }{ 2.4 } =2.04\times { 10 }^{ -4 }\left[ Nm \right] \longrightarrow (12)\]
\\
\\
\textbf{M\'{e}todo Din\'{a}mico.}
Sabemos que el momento de inercia de una barra cilindrica est\'{a} dado por la ecaci\'{o}n (11). Luego:
\[I=\frac { (57.8\times { 10 }^{ -3 }\quad \left[ kg \right] ){ (2.4\times { 10 }^{ -2 } \quad \left[ m \right] ) }^{ 2 } }{ 12 }= 1.156\times { 10 }^{ -4 }\left[ kg{ m }^{ 2 } \right]   \longrightarrow (13) \]
La constante K de torsi\'{o}n est\'{a} dado por la ecuaci\'{o}n (10), nuestro par\'{a}metro ser\'{a} de 10 oscilaciones, donde el tiempo total fue de $60.035(s)$, as\'{i} pues:
\[K=2\pi \frac { I }{ T^{ 2 } }= 2\pi \frac{(1.156\times { 10 }^{ -4 }\left[ kg{ m }^{ 2 } \right])}{(60.035\left[ s \right])^{2}}= 2.01524\times { 10 }^{ -4 }\left[ Nm \right] \longrightarrow (14) \] 
\\
\\
\textbf{Error Porcentual.}
En este caso, como el material proporcionado no nos dijeron de qu\'{e} estaba hecho, nuestro error porcentual que calcularemos ser\'{a} entre el resultado del M\'{e}todo est\'{a}tico y el M\'{e}todo Din\'{a}mico de las ecuaciones (12) y (14). 
\[ERROR \quad PORCENTUAL=1.214 \%  \] 
\\
\pagebreak
%----------------------------------------------------------------------------------------
%	SECTION 6
%----------------------------------------------------------------------------------------

\section{Discusi\'{o}n.}
Podemos decir que nuestra medici\'{o}n fue exitosa ya que el error porcentual que obtubimos por dos m\'{e}todos distintos fue muy peque\~{n}o adem\'{a}s podemos tomar cualquiera de los dos valores (pero fijo) para realizar nuestros experimentos con este material. Cabe resaltar que, el material ha ido perdiendo sus propiedades con el tiempo, y se comporta de manera distinta a como lo predicar\'{i}a el fabricante ya que el mismo est\'{a} ya muy gastado.    

%----------------------------------------------------------------------------------------
%	SECTION 7
%----------------------------------------------------------------------------------------

\section{Conclusiones.}
En esta pr\'{a}ctica aprendimos a utilizar e identificar bien los instrumentos de medici\'{o}n y las partes que lo componen, adem\'{a}s de que intuitivamente conocimos el principio b\'{a}sico de un amplificador lineal y un mult\'{i}metro, a calibrarlos y sus unidades. En la balanza de torsi\'{o}n aprendimos que al manejar con dispositivos de medici\'{o} de \'{e}sta \'{i}ndole las vibraciones debidas al movimiento de la mesa afectan en gran medida al resutado, por lo tanto la balanza de torsi\'{o}n debe estar en una mesa fija con arena como amortiguadora de vibraciones. En cuanto a los resultados del alambre referente a su m\'{o}dulo de torsi\'{o}n fueron obtenidos de manera exitosa, las variaciones en el resultado de ambos m\'{e}todos eran m\'{i}nimas y por lo tanto se pod\'{i}a tener certeza de la magnitud.

%----------------------------------------------------------------------------------------
%	SECTION 9
%----------------------------------------------------------------------------------------
 \section{Bibliograf\'{i}a.}
1.-https://es.wikipedia.org/wiki/Amplificador_ electrC3B3nico.\\
2.-https://es.wikipedia.org/wiki/Balanza_de_torsiC3B3n.\\
3.-https://es.wikipedia.org/wiki/MultC3ADmetro.\\
4.-https://sites.google.com/site/labenriquesalgadoruiz/home/politecnico-1/fisica-iii.\\
5.-Resnick/Halliday/Krane. Fundamentos de F\'{i}sica. Volumen 2. Edici\'{o}n 6, extendida. CESA\\

%----------------------------------------------------------------------------------------
%	SECTION 10
%----------------------------------------------------------------------------------------

\end{document}