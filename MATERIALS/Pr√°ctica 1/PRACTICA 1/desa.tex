\documentclass[11pt,a4paper]{article}
\usepackage[latin2]{inputenc}
\usepackage[spanish]{babel}
\usepackage{amsmath}
\usepackage{amsfonts}
\usepackage{amssymb}
\usepackage{makeidx}
\usepackage{graphicx}
\usepackage[left=2cm,right=2cm,top=2cm,bottom=2cm]{geometry}
\begin{document}
\section{Desarrollo experimental.}\\

\textbf{Materiales:}\\
Soporte con material \'{o}ptico reflexivo (espejo y rayo de luz).\\
Material a estudiar (lat\'{o}n o cobre).\\
Hoja de papel milim\'{e}trico.\\
Regla y l\'{a}piz.\\
Fuente de alimentaci\'{o}n.\\
Pesas de distintas masas.\\
Metro para medir.\\
Nivel de agua.\\

\textbf{Procedimiento:}\\
Comenzamos la pr\'{a}ctica con el maerial proporcionado por el equipo de laboratorio, procedimos a medir algunas de nuestras const\'{a}ntes como lo son; la distancia del soporte medida desde el espejo hasta la hoja de papel milim\'{e}trico (X), la elongaci\'{o}n inicial del material ya sea lat\'{o}n o cobre, la distancia del soporte para en espejo hasta el hilo, y porcedimos a medir rl \'{a}rea transversal del material por medio de un tornillo microm\'{e}trico.

\begin{figure}[hbtp]
\caption{Arreglo general. }
\centering
\includegraphics[width=4cm]{imágenes/sssssssssss.jpg}
\end{figure}




\end{document}