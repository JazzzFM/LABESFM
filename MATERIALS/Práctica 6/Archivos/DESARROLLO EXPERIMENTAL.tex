\documentclass[10pt,a4paper]{article}
\usepackage[latin1]{inputenc}
\usepackage[spanish]{babel}
\usepackage[utf8]{inputenc}
\usepackage{amsmath}
\usepackage{amsfonts}
\usepackage{amssymb}
\usepackage{graphicx}
\usepackage[left=2cm,right=2cm,top=2cm,bottom=2cm]{geometry}
\begin{document}
\section*{Desarrollo Experimental.}\\
\textbf{Lista de Materiales:} \\
Dispositivo generador de ondas planas con proyector de sombras.\\
Agua limpia. \\
C\'{a}mara fotogr\'{a}fica.\\
Distintas formas para actuar de barrera. \\

\section*{Procedimiento.}
1.-Identificar bien los instrumentos. \\
2.-Llenar apenas el tanque de ondas.\\
3. Identificar el oscilador y variar la frecuencia a una estable. \\
4.-Una vez que se estabilizaron las ondas planas plantear como se realizar\'{a}n las observaciones.\\
5.-Identificar cada uno de los fen\'{o}menos.\\
\section*{Resultados.}
\\
\\
\begin{figure 5}
\centering
\includegraphics[width=9cm]{../../../../../../Pictures/IMG_20170321_133058096.jpg} 
\\
\caption{Ondas incidentes, reflejada y estacionaria.}
\end{figure 5}
\\
\\
\begin{figure 5}
\centering
\includegraphics[width=9cm]{../../../../../../Pictures/p111.jpg} 
\\
\end{figure 5}
\\
\\
\medskip
\begin{figure 5}
\centering
\includegraphics[width=7cm]{../../../../../../Pictures/sssssss.jpg} 
\includegraphics[width=7cm]{../../../../../../Pictures/p11.jpg} 
\\
\end{figure 5}
\\
\\
\medskip
\begin{figure 5}
\centering
\includegraphics[width=9cm]{../../../../../../Pictures/phf1.jpg} 
\\
\caption{Principio de Huygens-Frensel 1.}
\end{figure 5}
\\
\\
\medskip
\begin{figure 5}
\centering
\includegraphics[width=7cm]{../../../../../../Pictures/phf2.jpg} 
\includegraphics[width=7cm]{../../../../../../Pictures/phf22.jpg}
\\
\caption{Principio de Huygens-Frensel 2 a y b.}
\end{figure 5}
\\
\medskip
\begin{figure 5}
\centering
\includegraphics[width=7cm]{../../../../../../Pictures/phf3.jpg} 
\includegraphics[width=7cm]{../../../../../../Pictures/phf32.jpg} 

\\
\caption{Principio de Huygens-Frensel 3 a y b.}
\end{figure 5}
\\
\\
\medskip
\begin{figure 5}
\centering
\includegraphics[width=9cm]{../../../../../../Pictures/IMG_20170321_134413995.jpg}
\\
\caption{Difracci\'{o}n de ondas.}
\end{figure 5}
\\
\\
\medskip
\begin{figure 5}
\centering
\includegraphics[width=7cm]{../../../../../../Pictures/IMG_20170321_133643792.jpg} 
\includegraphics[width=7cm]{../../../../../../Pictures/IMG_20170321_133601495.jpg} 
\\ 
\caption{obstaculos peque\~{n}os.}
\end{figure 5}
\\
\\
\medskip
\begin{figure 5}
\centering
\includegraphics[width=7cm]{../../../../../../Pictures/IMG_20170321_134224387.jpg} 
\includegraphics[width=7cm]{../../../../../../Pictures/IMG_20170321_133851465.jpg} 
\\
\caption{obstaculos peque\~{n}os.}
\end{figure 5}
\\
\\
\medskip
\begin{figure 5}
\centering
\includegraphics[width=7cm]{../../../../../../Pictures/efd.jpg} 
\\
\caption{Efecto Doppler.}
\end{figure 5}
\\




 \section*{Discusiones.}\\
Al llevar a cabo estas observaciones notamos que el instrumento no pod\'{i}a hacer algunas oscilaciones y por ende usamos los instrumentos a disposici\'{o}n adem\'{a}s el instrumento pod\'{i} variar sus oscilaciones y as\'{i} pudimos obserar mejor el efecto sobre estas, resaltamos que en esta pr\'{a}ctica s\'{o}lo se analizaron aspector cualitativos de las ondas planas y por ende, verificar la geometr\'{i}a propuesta a parir de la teor\'{i}a de \'{e}sta. 



\end{document}
