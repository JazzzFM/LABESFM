\documentclass[11pt,a4paper]{article}
\usepackage[latin1]{inputenc}
\usepackage[spanish]{babel}
\usepackage{amsmath}
\usepackage{amsfonts}
\usepackage{amssymb}
\usepackage{makeidx}
\usepackage{graphicx}
\usepackage{lmodern}
\usepackage[left=2cm,right=2cm,top=2cm,bottom=2cm]{geometry}
\author{Flores Rodguez Jaziel David }
\title{Práctica 1}
\begin{document}

\tableofcontents
%----------------------------------------------------------------------------------------
%	SECTION 1
%----------------------------------------------------------------------------------------
\\
\medskip
\medskip
\section{Resumen.}
\
Se maneja de manera experimental el concepto de capacitor, adem\'{a}s se hace un incapi\'{e} en el principio de funcionamiento de ciertos sistemas el\'{e}ctricos como el puente de impedancias. Se observa la relaci\'{o}n entre el voltaje y la capacitancia y finalmente se encuentran expresiones generales para diversos materiales en unas placas paralelas y as\'{i} variando distintas propiedades enconrar una relaci\'{o}n caracter\'{i}stica del material.\\
\\
\medskip
Palabras clave: voltaje, diferencia de potencial o voltaje, puente de impedancias.

%----------------------------------------------------------------------------------------
%	SECTION 2
%----------------------------------------------------------------------------------------

\section{Objectivo.}
a) Comprender el funcionamiento del puente de impedancias, su uso y manejo en la medici\'{o}n de resistencias, capacitancias e inductancias.
\\
\\
b)Estudiar la capacitancia en funci\'{o}n del voltaje aplicado a las placas del capacitor.
\\
\\
c) Estudiar la capacitancia de los capacitores de placas planas y paralelas en funci\'{o}n del \'{a}rea de las placas (A) y la distancia de separaci\'{o}n de las placas (d).
\\
\\
d)Con ayuda del capacitoe de placas planas y paralelas, obtener la constante diel\'{e}ctrica de distintos matriales.
\pagebreak


%----------------------------------------------------------------------------------------
%	SECTION 3
%----------------------------------------------------------------------------------------
\section{Metodolog\'{i}a.}
Un capacitor es un dispositivo el\'{e}ctrico, formado por dos conductores aislados entre si, que al aplicarles una diferencia de potencial V, se reacomoda su carga el\'{e}ctrica, uno queda con carga +Q y el otro con carga -Q de esta forma se dice que el capacitor esta cargado.
\\
\begin{figure}[hbtp]
\centering
\includegraphics[width=9cm]{../../../../../../Pictures/cacacaca.jpg} 
\caption{(A)Un capacitor de placas paralelas, hecho apartir de dos platos de \'{a}rea A separados por una distancia d. La carga que yace sobre la supercifie del plato tienen la misma magnitud q pero de signos opuestos. (b) As\'{i}como las lineas de campo muestra el campo el\'{e}ctrico debido a los platos cargados es uniforme en la regi\'{o}n central entre los platos. El campo no es uniforme en las esquinas de las placas como indica el borde de las lineas campo all\'{i}.}
\end{figure}
\\
\\
Un primer paso en nuestra discusi\'{o}n de los capaciores es determinar qu\'{e} tanta carga puede ser alamacenada. Este tanto es llamasa capacitancia. Un capacitor cargado puede activar un circuito el\'{e}ctrico. Se define la capacitancia de un capacitor como. 
\\
\[C=\frac{Q}{V} \quad (1)\]
\\
La unidad de capacitancia es:
\\
\[C= \frac{Coulomb}{Volt}= Farad \quad (2)\]
\\
Para un capacitor de placas planas y paralelas se tiene:
\\ 
\[C=\epsilon \frac{A}{d} \quad (2)\]
\\
Donde $\epsilon=({\epsilon}_{0})({\epsilon}_{T})$ y ${\epsilon}_{T}$ es la consta\'{a}nte diel\'{e}ctrica. Adem\'{a}s ${\epsilon}_{0} = 8.85\times 10^{-12} \frac{F}{m}$.
\\
\\
\textbf{Puente de impedancias.}
Es un dispositivo que se emplea en la medici\'{o}n de resistencias, capacitancias e inductancias. El principio de funcionamiento esta en relaci\'{o}n al diagrama de la siguiente figura. El puente de impedancias tiene un galvan\'{o}metro de alta sensibilidad como dispositivo indicador, cuatro elementos conectados como se indica en la figura y una fuente de voltaje de C.D. El galvan\'{o}metro sirve de indicador a cero y pone de manifiesto la condici\'{o}n de equilibrio. Este diagrama representa al tipo m\'{a}s com\'{u}n, llamado puente de Wheatstone. En principio, la medici\'{o}n se basa en la condici\'{o}n de equilibrio del puente, la cual se cumple cuando entre los puntos b y d no circula corriente y por lo tanto se debe cumplir que:
\\
\begin{figure}[hbtp]
\centering
\includegraphics[width=7cm]{../../../../../../Pictures/puente.jpg}
\caption{Puente de impedancias.}
\end{figure}
\\
\\
\[{I}_{1} {X}_{2} = {I}_{2} {X}_{1} \quad (3)\]
\[{I}_{1} {X}_{3} = {I}_{2} {X}_{4} \quad (4)\]
De (3)
\[{I}_{1} = \frac{{I}_{2} {X}_{1} }{{X}_{2}} \quad (5) \]
Sustituyendo en (4) 
\[\frac{{I}_{2} {X}_{1} {X}_{3} }{{X}_{2}}= {I}_{2}{X}_{4} \quad (6)\]
Por lo que: 
\[{X}_{3} {X}_{1} = {X}_{2} {X}_{4} \quad (7)\]
Siendo las X resistencias, capacitores o inductores (bobinas), por lo tanto conociendo tres elementos es posible calcular el cuarto elemento. Si se desconoce $ {X}_{4}$ se puede calibrar el puente para que una variaci\'{o}n de $ {X}_{4}$ corresponda al factor:
\[\frac{{X}_{4} {X}_{1}}{{X}_{2}} \quad (8)\]
Y se pueda medir directamente en la escala el valor de  ${X}_{4}$
%----------------------------------------------------------------------------------------
%	SECTION 4
%----------------------------------------------------------------------------------------

\section{Instrumentaci\'{o}n.}\\
1- Amplificador Lineal.\\
2- Mult\'{i}metro.\\
3- Capacitor de placas planas. \\
4- Capacitores y resistencias.
5. Fuente de alto voltaje.\\
6. Placas de distintos matriales de las mismas dimensiones.\\
7. Flex\'{o}metro.\\
\pagebreak

%----------------------------------------------------------------------------------------
%	SECTION 5
%----------------------------------------------------------------------------------------


\section{Datos y Resultados.}\\

\textbf{PARTE 1.}\\
En esta parte se da una introducci\'{o}n a los m\'{e}odos de medici\'{o}n de par\'{a}metros el\'{e}ctricos como lo es la capacitancia y la resistencia, as\'{i} como el voltaje y la corriente, aqu\'{i} se dio una breve explicaci\'{o}n del funcionamiento del instrumento de medici\'{o}n mult\'{i}metro basado en el puente de impedancias visto previamente. En este caso s\'{o}lo se cambia el par\'{a}metro de medici\'{o}n con el giro del dial en el s\'{i}mbolo deseado y la medida se muestra en la pantalla. Como se muestra en la figura: 
\\
\begin{figure}[hbtp]
\centering
\includegraphics[width=7cm]{../../../../../../Pictures/Multimetro-uso.jpg}
\caption{Medici\'{o}n de resistencia con el mult\'{i}metro.}
\end{figure}
\\
\textbf{PARTE 2.}
\\
Se conect\'{o} e equipo como se muestra en la figura. Posteriromente se separ\'{o} las placas del capacitor a 2mm  y con ayuda del tornillo microm\'{e}trico, luego se aplic\'{o} por medio de la fuente de alto voltaje de C.D. voltajes de 10 en 10 hasta 100V y medimos por medio del amplificador lineal la carga de una de las placas para cada voltaje aplicado.
\begin{figure}[hbtp]
\centering
\includegraphics[width=6cm]{../../../../../../Pictures/pepepepepepepepe.jpg} 
\includegraphics[width=6cm]{../../../../../../Pictures/IMG_20171003_125008153.jpg} 
\caption{Areglo experimental.}
\end{figure}
\\
Despu\'{e}s con los datos obtenidos se contruy\'{o} la siguiente tabla. 
\begin{figure}[hbtp]
\centering
\includegraphics[width=4.2cm]{../../../../../../Pictures/sssss.jpg}
\caption{Tabulaci\'{o}n de datos.}
\end{figure}
\\
Para despu\'{e}s obtener a siguiente gr\'{a}fico de dispersi\'{o}n.
\\
\begin{figure}[hbtp]
\centering
\includegraphics[width=12cm]{../../../../../../Pictures/dipssss.jpg}
\caption{Gr\'{a}fico de dispersi\'{o}n.}
\end{figure}
\\
\textbf{Ajuste de datos.} 
Ahora podemos hacer el respectivo ajuste por el m\'{e}todo de m\'{i}nimos cuadrados para encontrarun modelo lineal Y = ax + b tales tales que $\left( { x }_{ i },{ y }_{ i } \right) \rightarrow \left( V(  v  ),{Q (C))$  de los datos del experimento, cuya tablas de entrada y modelo es el siguiente:
\\
\\
\medskip
\textbf{Tabla de entrada.}
\\
\begin{figure 6}
\centering
\begin{tabular}{|c|c|c|c|c|}
\hline 
n&$\sum _{ i=1 }^{ n }{ { x }_{ i } } (V  )$ & $\sum _{ i=1 }^{ n }{ { y }_{ i } }(C)$ & $ \sum _{ i=1 }^{ n }{ { y }_{ i } } { x }_{ i }(V\cdot C )$ & $\sum _{ i=1 }^{ n }{ { x }_{ i }^{ 2 } }(V ^{ 2 })$ \\ 

\hline 
13&$6.65\times { 10 }^{ 2 } $& $1.46\times { 10 }^{ -7 }$ & $1.09\times { 10 }^{ -5 }$ & $5.0625\times { 10 }^{ 4 }$\\ 
\hline 
\end{tabular}
\end{figure 6} 
\\
\\
De donde:
\[a=\frac { n\sum _{ i=1 }^{ n }{ { x }_{ i }{ y }_{ i } } -\sum _{ i=1 }^{ n }{ { x }_{ i } } \sum _{ i=1 }^{ n }{ { y }_{ i } }  }{ n\sum _{ i=1 }^{ n }{ { x }_{ i }^{ 2 } } -{ \left( \sum _{ i=1 }^{ n }{ { x }_{ i } }  \right)  }^{ 2 } } \quad y\quad b=\frac { \sum _{ i=1 }^{ n }{ { x }_{ i }^{ 2 } } \sum _{ i=1 }^{ n }{ { y }_{ i } } -\sum _{ i=1 }^{ n }{ { x }_{ i }{ y }_{ i } } \sum _{ i=1 }^{ n }{ { x }_{ i } }  }{ n\sum _{ i=1 }^{ n }{ { x }_{ i }^{ 2 } } -{ \left( \sum _{ i=1 }^{ n }{ { x }_{ i } }  \right)  }^{ 2 } }.\]
Sustituyendo los valores queda:

\[{ a }=\frac { (7\times 1.09\times { 10 }^{ -5 }\left[ V{ C } \right] )-(6.65\times { 10 }^{ 2 }\times 1.46\times { 10 }^{ -7 }\left[ V{ C } \right] ) }{ (13\times 5.0625\times { 10 }^{ 4 }\left[ { V }^{ 2 } \right] )-{ \left( 6.65\times { 10 }^{ 2 }\left[ V \right]  \right)  }^{ 2 } } =2.04231\times {10}^{ -10 }\left[ { Farad } \right] .\]

\[ { b }=\frac { (5.0625\times { 10 }^{ 4 }\times 1.46{ \times 10 }^{ -7 }\left[ { C{ V }^{ 2 } } \right] )-(1,09\times { 10 }^{ -5 }\times 6.65{ \times 10 }^{ 2 }\left[ { C{ V }^{ 2 } } \right] ) }{ (13\times 5.0625\times { 10 }^{ 4 }\left[ { V }^{ 2 } \right] )-{ \left( 6,65\times { 10 }^{ 2 }\left[ {V  } \right]  \right)  }^{ 2 } } = 7\times { 10 }^{ -10 }\left[ { C }  \right] .\]
Finalmente queda el modelos propuesto:
\[ C(V)= 2.04231\times {10}^{ -10 }(V) + 7\times { 10 }^{ -10 } \quad \rightarrow (9) \]
De (9) podemos graficar sus correspondientes lineas de tendencia, a continuaci\'{o}n vamos a graficar su modelo y a tratarlo con m\'{a}s detalle.
\\ 
\\
 Luego de (1) podemos obtener la capacitancia de las placas paralelas: 
\[Capacitancia = 2.04231\times {10}^{ -10 } \left[ { Farad } \right] \]
\\
\begin{figure}[hbtp]
\centering
\includegraphics[width=12cm]{../../../../../../Pictures/exeeeeee.jpg}
\caption{Linea de tendencia y ecuaci\'{o}n de la misma. }
\end{figure}
\\
\\
\textbf{Error porcentual.}\\
Apartir de este resultado, con el mult\'{i}metro procedimos a medir directamente la capacitancia de las placas paralelas y fue de ${C}_{m} = 2.07\times {10}^{-10}$ y as\'{i} obtenemos el siguiente error porcentual: 
\[Error \quad Porcentual = 1.337 \% \]

\section*{Discusi\'{o}n.}
En esta parte del desarrolo experimental podemos decir que con las herramientas matem\'{a}ticas con las que contamos hasta ahora obtuvimos una expresi\'{o}n para la capacitancia bastante aceptable y con un error muy peque\~{n}o, podemos concluir que esta parte se realiz\'{o} de una manera adecuda ya que era muy sencilla las medici\'{o}nes y la expresi\'{o}n que encontramos entre la diferencia de potencial y la carga que era la pendiente de esta funci\'{o}n lineal relacionaba a todos los puntos de una manera muy aproximada con el m\'{e}todo de m\'{i}nimos cuadrados y as\'{i} encontramos la propiedad llamada capacitancia. 

\pagebreak

\textbf{PARTE 3.}
\\
Nuevamente con el capacitor de placas planas paralelas y con el mult\'{i}metromedimos la capacitancia para diferentes distintas entre las placas, iniciando en 2 mm y cambiando de 2 en 2 hasta 20 mm. Y as\'{i} obtuvimos la siguiente tabla de datos:\\
\begin{figure}[hbtp]
\centering
\includegraphics[width=5cm]{../../../../../../Pictures/OFOOFOFOFOF.jpg}
\caption{Tabla de datos.}
\end{figure}
\\
Y as\'{i} se construye la siguiente gr\'{a}fica de dispersi\'{o}n obtenida de la tabala de datos. 
\begin{figure}[hbtp]
\centering
\includegraphics[width=12cm]{../../../../../../Pictures/jaajajajajajajajaja.jpg}
\caption{Gr\'{a}fico de dispersi\'{o}n. }
\end{figure}
\\
\textbf{Ajuste de datos.} Se observa que los puntos experimentales siguen un patr\'{o}n alineal, entonces para ajustarlo a la mejor ecuaci\'{o}n ocuparemos un modelo exponencial ajustado por m\'{i}nimos cuadrados, esto es:

\[y=b{ x }^{ a }\]

Donde b y a son los par\'{a}metros buscados dados por:
\[a=\frac { n\sum _{ i=1 }^{ n }{ { x }_{ i }ln{ y }_{ i } } -\sum _{ i=1 }^{ n }{ ln{ y }_{ i } } \sum _{ i=1 }^{ n }{ { x }_{ i } }  }{ n\sum _{ i=1 }^{ n }{ { x }_{ i }^{ 2 } } (n)-\sum _{ i=1 }^{ n }{ { \left( { x }_{ i } \right)  }^{ 2 } }  } \]
\[ln(b)=\frac { \sum _{ i=1 }^{ n }{ { x }_{ i }ln({ y }_{ i })- } \sum _{ i=1 }^{ n }{ ln({ y }_{ i })\sum _{ i=1 }^{ n }{ { x }_{ i } }  }  }{ \sum _{ i=1 }^{ n }{ { x }_{ i }^{ 2 }(n)\quad -\sum _{ i=1 }^{ n }{ { \left( { x }_{ i } \right)  }^{ 2 } }  }  } . \]

\\
\textbf{Tabla de entrada 1.}
\\
\begin{figure 6}
\centering
\begin{tabular}{|c|c|c|c|c|c|}
\hline 
n&$\sum _{ i=1 }^{ n }{ { x }_{ i } } (m)$ & $\sum _{ i=1 }^{ n }{ { ln y }_{ i } }(F)$ & $ \sum _{ i=1 }^{ n }{ {ln y }_{ i } } { x }_{ i }(Fm)$ & $\sum _{ i=1 }^{ n }{ { x }_{ i }^{ 2 } }({ m }^{ 2 })$ \\ 
\hline 
10& $1.1\times{10}^{ -1 } $ & -234.943 & -281.985 & $1.54\times { 10 }^{ -3 }$  \\ 
\hline 
\end{tabular}
\end{figure 6} 
\\
Sustituyendo los valores queda:
\[ a=\frac { (10\times -281.95\left[ Fm \right] )-(-234.943\left[ F \right] \times 1.1 \times {10}^{ -1 }\left[ { m } \right] ) }{ (10 \times 1.54 \times {10}^{ -3 })\left[ { m }^{ 2 } \right] -{{((1.1\times {10}^{ -1 }})^{2})\left[ { m }^{ 2 }\right]}}  } =-8.03\times {10}^{-1}  \left[ { F }/m \right]  .\]

\[ ln(b)=\frac { \left( -281.985\right) \left[ Fm \right] -\left( (-234.943)(1.1\times {10}^{ -1 }) \right) \left[ Fm \right]  }{ (1.54\times {10}^{ -3 })\left[ m^{ 2 } \right] -((1.1\times{10}^{ -1 })^{2})\left[ m^{ 2 } \right]  } =-27.63102 \left[ { N }/m \right] .\]
Y entonces 
\[ b=1.0567\times { 10 }^{ -12 } \left[ { N }/m \right].\]
Finalmente queda el modelos propuesto:
\[ C(d)=1.0567\times { 10 }^{ -12 }d^{ -0.803 }  \quad \rightarrow(1) \]
\medskip
\\
\\
 Luego de (2) podemos obtener la permtitividad del medio en el vac\'{i}o usando suponiendo que est\'{a} en el vac\'{o}, adem\'{a}s de la expresi\'{o}n anterior deducimo que $b= ({\epsilon}_{0})(A)$ donde A es el \'{a}rea de las placas paralelas, tenemos que el di\'{a}metro es de $D= 256 \times {10}^{ -3 } (m)$, luego el \'{a}rea es de $A=5.147 \times {10}^{-2} \left[ { m }^{ 2 } \right]$: 
\[{\epsilon}_{0} = 2.031474\times {10}^{ -11 } \left[ { F/m } \right] \]
\\
\begin{figure}[hbtp]
\centering
\includegraphics[width=15.3cm]{../../../../../../Pictures/eoeoeoeoeoeoeooe.jpg} 
\caption{Linea de tendencia y ecuaci\'{o}n de la misma. }
\end{figure}
\\
\\
\textbf{Error porcentual.}\\
Apartir de este resultado, con el mult\'{i}metro procedimos a medir directamente la capacitancia de las placas paralelas y fue de ${\epsilon}_{0} = 8.85\times {10}^{-12} (F/m) $ y as\'{i} obtenemos el siguiente error porcentual: 
\[Error \quad Porcentual = 56..437 \% \]

\section*{Discusi\'{o}n.}
En esta parte tuvimos problemas y se ve reflejado en nuestro error porcentual, por las condiciones del laboratorio no podemos saber el diele\'{e}ctrico del aire en ese momento, por lo que tuvimos que suponer nuestros c\'{a}lculos como si estuviesemos en el vaci\'{o} y as\'{i} obtener la permitividad del vac\'{i}o con un error muy grande como para volver usarlo, como veremos despu\'{e}s esto nos causar\'{a} problemas. 

\pagebreak

\textbf{PARTE 4.}
\\
Se puso entre las placas del capacitor de placas planas y paralelas, uno de los diel\'{e}ctricos que se nos proporcion\'{o}, luego juntamos las placas justamente al espesor del material y se mid\'{o} esta distancia de separaci\'{o}n en el vernier del capacitar y an\'{o}tela. Posteriormente se conect\'{o} el puente de impedancias y medimos la capacitancia Cd, para contruir una tabla de datos. Sacamos el diel\'{e}ctrico del capacitor y pusimos la distancia entre placas igual a la que se midi\'{o} cuando ten\'{i}a el diel\'{e}ctrico, medimos con el puente de impedancias la capacitancia Ca. Todo esto se repiti\'{o} para los materiales restantes y llenamos la siguiente tabla.
\begin{figure}[hbtp]
\caption{Tabla de datos con distintos materiales. }
\centering
\includegraphics[width=12cm]{../../../../../../Pictures/CUATTROROFK.jpg}
\end{figure}
\\
\textbf{Error Porcentual.}\\
Atrav\'{e}s de estos datos obtenidos y con los valores reales de los materiales sacados de la bibliograf\'{i}a podemos calcular el error porcentual de cada uno, es decir:\\
Error porcentual del vidrio$= 96.07 \%$, Error porcentual del serol\'{o}n$= 95.02\%$, Error porcentual del Asbesto$= 97.2\%$, Error porcentual de la madera$=94.3 \%$, Error porcentual del acr\'{i}lico$=81.3 \%$.

\section*{Discusi\'{o}n.}
Aqu\'{i} podemos ver que era necesar\'{i}o poner de referencia la capacitancia inicial en el vac\'{i}o ya que el cambio entre esta y la del aire era muy grande y as\'{i} las constantes diel\'{e}ctricas resultaron muy distintas a las esperadas o las reales proporcionadas en la bibliograf\'{i}a, sin embargo sabemos que esto fue un error de la parte anterior por ignorar la permitivada en el vac\'{i}o, tambi\'{e}n podemos decir que factores externos como el rosamiento con otros materiales sobre los diel\'{e}ctrico afectaba directamente a la medidia del la capacitancia en el mult\'{i}metro. 

%----------------------------------------------------------------------------------------
%	SECTION 6
%---------------------------------------------------------------------------------------

\section{Conclusiones.}
En esta pr\'{a}ctica aprendimos a usar e puente de ipedancias en su forma de mult\'{i}metro, tambi\'{e}n aprendimos a medir las capacitancias y la resistencia de diversos materiales. Calculamos la onstante die\'{e}ctrica del serol\'{o}n vidrio, acr\'{i}lico, asbesto, madera os cuales fueron compaados con los datos de la bibliograf\'{i}a. Por otra lado se observ\'{o} que la carga es directamete proporcinal al producto de la capacitancia con el voltaje. Se encontr\'{o} que la capacitancia aumenta por el factor de la constante diel\'{e}ctrica cuando todo el espacio donde existe est\'{a} lleno completamente de un diel\'{e}ctrico. Vimos tambi\'{e}n que la constante diel\'{e}ctrica es una propiedad fudamental del materia diel\'{e}ctrico y es independiente del tama\*{n}o o la forma del conductor. 
%----------------------------------------------------------------------------------------
%	SECTION 9
%----------------------------------------------------------------------------------------
 \section{Bibliograf\'{i}a.}
1.-http://guasa.ya.com/elektron/electropedia.html\\
2.-https://es.wikipedia.org/wiki/MultC3ADmetro.\\
3.-https://sites.google.com/site/labenriquesalgadoruiz/home/politecnico-1/fisica-iii .\\
4.-Resnick/Halliday/Krane. Fundamentos de F\'{i}sica. Volumen 2. Edici\'{o}n 6, extendida. CESA\\
5.- https://es.wikipedia.org/wiki/Campo_electri78co.\\

%----------------------------------------------------------------------------------------
%	SECTION 10
%----------------------------------------------------------------------------------------

\end{document}