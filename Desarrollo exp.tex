\documentclass[10pt,a4paper]{article}
\usepackage[latin1]{inputenc}
\usepackage[spanish]{babel}
\usepackage{amsmath}
\usepackage{amsfonts}
\usepackage{amssymb}
\usepackage{graphicx}
\usepackage[left=2cm,right=2cm,top=2cm,bottom=2cm]{geometry}
\begin{document}
\section*{2.  Desarrollo Experimental.}\\

 \\
\textbf{Lista de Materiales:} \\
Soporte con material \'{o}ptico reflexivo (espejo y rayo de luz).\\
Material a estudiar (lat\'{o}n o cobre). \\
Hoja de papel milim\'{e}trico. \\
Regla y l\'{a}piz. \\
Fuente de alimentaci\'{o}n.\\
Pesas de distintas masas. \\
Metro para medir. \\
Nivel de agua. \\
Medidor de \'{a}ngulos de inclinaci\'{o}n \\
\\
\\
\section*{Arreglo experimenal. Parte 1: M\'{e}todo \'{O}ptico.} \\

\begin{figure}[hbtp]
\centering
\includegraphics[width=10cm]{../../../../../../Pictures/dddddd.jpg}   
\caption{Arreglo general}
\end{figure}

\begin{figure}[hbtp]
\centering
\includegraphics[width=11cm]{../../../../../../Pictures/xxxxxxxxxxxxxxxxxx.jpg} 
\caption{a)Horizontalidad del rayo inicial.  b)Marca de origen. }
\end{figure}

\begin{figure}[hbtp]
\centering
\includegraphics[width=7cm]{../../../../../../Pictures/16651396_1391761460896440_996823809_o.jpg}  
\caption{Agrupamiento de pesas en el arreglo.}
\end{figure}

\begin{figure}[hbtp]
\centering
\includegraphics[width=17cm]{../../../../../../Pictures/ssssssss.jpg}  
\caption{Material para medir a H correspondiente.}
\end{figure}
\pagebreak 

\textbf{Procedimiento.}\\
\\
1.-Medir las const\'{a}ntes X, lo, x, y, calcular A.\\
2.-Asegurar la horizontalidad del rayo reflejado inicial y marcar el origen en el papel. (Figura 1)\\
3.-Colocar un peso y marcar en el papella nueva posici\'{o}n del \'{i}ndice. \\
4. Repetir el paso 3, de 12 a 25 veces.\\
5.-Medir en cada caso la H corespondiente, y calcular $\beta$, $\alpha$ y $\Delta l$ respectiva.\\
6.- Llenar la tabla.\\
7.- Graficar $\varepsilon$ VS (DU)t y ajustar para calcular la Y de la muestra.\\

\textbf{Resultados.}\\
\\
Comenzamos la pr\'{a}ctica con ell material proporcionado por el equipo de laboratorio, procedimos a medir algunas de nuestras constantes como lo son; la distancia del soporte medida desde el espejo hasta la hoja de papel milim\'{e}trico (X), la elongaci\'{o}n inicial del material ya sea lat\'{o}n o cobre, la distancia del soporte para en espejo hasta el hilo, y procedimos a medir el \'{a}rea transversal del material por medio de un tornillo microm\'{e}trico. \\
\medskip
\\
\medskip
\caption{Tabla 1.} 
\\ 
\begin{tabular}{|c|c|c|c|c|}
\hline 
Di\'{a}metro(m) & X (m) & A(${ m }^{ 2 }$) & lo(m)& x (m)\\ 
\hline 
${ 3.54\times 10 }^{ -3 }$ & 2.32 & ${ 3.937\times 10 }^{ -3 }$ & 1.185 &${ 8.5\times 10 }^{ -2 }$ \\
\hline 
\end{tabular}\\

\medskip
Llenamos la Tabla 2 proporcionada para despu\'{e}s graficar el esfuerzo vs deformaci\'{o}n unitaria y ajustar para calcular el Y de cada muestra.\\
\\
\textbf{Tabla 2.}\\
\\
\begin{figure 2}
\centering
\includegraphics[width=14cm]{../../../../../../Pictures/momomom1.jpg} 
\\
\end{figure 2}
\\
De la cual extragimos los datos de la siguiente tabla para poder hacer la gr\'{a}fica $\varepsilon$ VS (DU).
\\
\textbf{Tabla 3.}\\
\\
\begin{figure 3}
\centering
\includegraphics[width=4cm]{../../../../../../Pictures/momomomeo2.jpg}  
\\
\end{figure 3}
\\
\section*{Gr\'{a}fica de dispersi\'{o}n}
\\
\\
\begin{figure 5}
\centering
\includegraphics[width=10cm]{../../../../../../Pictures/MOMOMO3.jpg} 
\\
\caption{Gr\'{a}fico de dispersi\'o}n para el m\'{e}todo \'{o}ptico.}
\end{figure 5}
\\
Cabe reslatar que , como se muestra en la figura, algunos puntos de nuestra tendencia tieen un salto muy grande y as\'{i} generean 2 conjuntos en los cuales cada uno parece conformar una linea reacta, primero analizaremos el comportamiento del conjunto ttal de la recta y luego por separado en un anexo.  
\section*{Ajuste de datos.}\\
Por el Apr\'{e}ndice 1 podemos hacer el respectivo ajuste por el m\'{e}todo de m\'{i}nimos cuadrados para encontrar un modelo lineal $Y=ax +b$ tales que $\left( { x }_{ i },{ y }_{ i } \right) \rightarrow \left( Du(Ad),{ \sigma  }_{ t }(Pa) \right) $ para cada uno de los datos de cada exprimento se tendr\'{a} su tabla de entrada y los siguientes coeficientes a y b, de donde:
\[a=\frac { n\sum _{ i=1 }^{ n }{ { x }_{ i }{ y }_{ i } } -\sum _{ i=1 }^{ n }{ { x }_{ i } } \sum _{ i=1 }^{ n }{ { y }_{ i } }  }{ n\sum _{ i=1 }^{ n }{ { x }_{ i }^{ 2 } } -{ \left( \sum _{ i=1 }^{ n }{ { x }_{ i } }  \right)  }^{ 2 } } \quad y\quad b=\frac { \sum _{ i=1 }^{ n }{ { x }_{ i }^{ 2 } } \sum _{ i=1 }^{ n }{ { y }_{ i } } -\sum _{ i=1 }^{ n }{ { x }_{ i }{ y }_{ i } } \sum _{ i=1 }^{ n }{ { x }_{ i } }  }{ n\sum _{ i=1 }^{ n }{ { x }_{ i }^{ 2 } } -{ \left( \sum _{ i=1 }^{ n }{ { x }_{ i } }  \right)  }^{ 2 } }.\]
\medskip
\\
\\Tabla de entrada 1.
\\
\\
\begin{figure 6}
\centering
\begin{tabular}{|c|c|c|c|c|}
\hline 
 n & $\sum _{ i=1 }^{ n }{ { x }_{ i } } $(Ad) & $\sum _{ i=1 }^{ n }{ { y }_{ i } } $ (Pa) & $ \sum _{ i=1 }^{ n }{ { y }_{ i } } { x }_{ i } \left( Pa \right) \left( Ad \right)$ & $\sum _{ i=1 }^{ n }{ { x }_{ i }^{ 2 } }{ \left( Ad \right)  }^{ 2 }$ \\ 

\hline 
13 & $1.541\times { 10 }^{ -2 }$& $1.837\times { 10 }^{ 9 }$ & $4.193\times { 10 }^{ 8 }$ & $2.449\times { 10 }^{ -5 }$ \\ 
\hline 
\end{tabular}
\end{figure 6} 
\\
\medskip
Sustituyendo los valores queda:
\[a=\frac { (13\times 4.193\times { 10 }^{ 8 }\left[ Pa \right] \left[ Ad \right] )-(1.541\times { 10 }^{ -2 }\times 1.837\times { 10 }^{ 9 }\left[ Pa \right] \left[ Ad \right] ) }{ (13\times 2.449\times { 10 }^{ -5 }\left[ { Ad }^{ 2 } \right] )-{ \left( 1.541\times { 10 }^{ -2 }\left[ { Ad } \right]  \right)  }^{ 2 } } ={ 6.67\times 10 }^{ 10 } \frac { \left[ Pa \right]  }{ \left[ Ad \right]  } .\]

\[b=\frac { (2.449\times { 10 }^{ -5 }\times 1.837\times { 10 }^{ 9 }\left[ { Ad }^{ 2 } \right] \left[ Pa \right] )-(4.193\times { 10 }^{ 8 }\times 1.541\times { 10 }^{ -2 }\left[ Pa \right] \left[ { Ad }^{ 2 } \right] ) }{ (13\times 2.449\times { 10 }^{ -5 }\left[ { Ad }^{ 2 } \right] )-{ \left( 1.541\times { 10 }^{ -2 }\left[ Ad \right]  \right)  }^{ 2 } } ={ -7.88\times 10 }^{ 10 }\left[ Pa \right] .\]
Finalmente queda el modelos propuesto:
\[{ Y }_{ 1 }= -788\times { 10 }^{ 10 } + 6.67\times { 10 }^{ 10 }x\quad \rightarrow(1) \]

\\
Como la pendiente de la recta a tangente a la curva misma nos representa el m\'{o}dulo de Young , en el cas por su puesto para la gr\'{a}fica $\varepsilon$t VS (DU)t, es decir:
\[m=b=tan(\theta)= \frac { \varepsilon t }{ (Du)t } =\quad Y\quad (M\'{o}dulo\quad de\quad Young).\]
Y as\'{i}, podemos decir, por definici\'{o}n, que el m\'{o}duo de Young del material es: $Y=6.67\times 10 }^{ 10 }\left[ Pa \right] .$
\pagebreak
\section*{Error Porcentual.}\\
Los valores verdaderos (Tabla 1) de los m\'{o}dulos de Young del cobre es ${ Y }_{ 1 }=11\times { 10 }^{ 10 }Pa$. Entonces, de nuestras mediciones y c\'{a}lculos podemos obtener el error porcentual:

\[{ E }rror \quad porcentual \quad { Y }_{ 1 }=\frac { Error\quad verdadero }{ Valor\quad Verdadero } =\frac { Valor\quad verdadero - Valor\quad aproximado }{ Valor\quad verdadero } \times 100= 36 \%.\]
\\
\section*{Ajuste por excel.}\\
Notemos que mientras por medio de c\'{a}alculos pudimos enconrar un modoelo, el programa Excel pudo enontrar otro, es cual se muestra a continuaci\'{o}n.

\begin{figure}[hbtp]
 \centering
\includegraphics[width=11cm]{../../../../../../Pictures/KKKKKKKKKKK.jpg} 
 \caption{Modelo por medio de excel. }  
\end{figure}
 Cuya ecuaci\'{o}n es:
 \[{ Y }_{ ex}= 2\times { 10 }^{ 7 } + 1\times { 10 }^{ 11 }x\quad \rightarrow(1.1) \]
 \pagebreak
\section*{Anexo.}
En esta secci\'{o}n analizaremos los comporamientos por separado de los conjuntos de puntos los cuales parcen tener un comportamiento lineal distinto, con un salto marcado. Los conjuntos est\'{a}n marcados por una linea roja y una linea azul los cuales llamaremos A y B respectivamente. As\'{i} tendremos dos tablas de datos de los bloques A y B que vamos a ananalizar. 
\\
\medskip
\begin{figure 9}
\centering
\includegraphics[width=4cm]{../../../../../../Pictures/ELASTICIDAD23.jpg} 
\includegraphics[width=4.5cm]{../../../../../../Pictures/ELASTICIDAD24.jpg} 
\\
\caption{Tablas de las cuales se analizar\'{a} los bloques de datos.}
\end{figure 9}
\\
\section*{Gr\'{a}fica de dispersi\'{o}n}
\\
\\
\begin{figure 5}
\centering
\includegraphics[width=10cm]{../../../../../../Pictures/MOMOMO31.jpg} 
\\
\caption{Gr\'{a}fico de dispersi\'o}n para el m\'{e}todo \'{o}ptico.}
\end{figure 5}
\\
\section*{Ajuste de datos.}\\
Por el Apr\'{e}ndice 1 podemos hacer el respectivo ajuste por el m\'{e}todo de m\'{i}nimos cuadrados para encontrar un modelo lineal ${Y}_{a}={a}_{1}x +{b}_{1}$ y ${Y}_{b}={a}_{2}x + {b}_{2}$ tales que $\left( { x }_{ i },{ y }_{ i } \right) \rightarrow \left( Du(Ad),{ \sigma  }_{ t }(Pa) \right) $ y $\left( { x }_{ j },{ y }_{ j } \right) \rightarrow \left( Du(Ad),{ \sigma  }_{ t }(Pa) \right)$ para cada uno de los datos de cada exprimento se tendr\'{a} su tabla de entrada y los siguientes coeficientes a y b, de donde sus tablas de entrada correspondientes son :
\medskip
\\
\\Tabla de entrada 1.
\\
\\
\begin{figure 6}
\centering
\begin{tabular}{|c|c|c|c|c|}
\hline 
 n & $\sum _{ i=1 }^{ n }{ { x }_{ i } } $(Ad) & $\sum _{ i=1 }^{ n }{ { y }_{ i } } $ (Pa) & $ \sum _{ i=1 }^{ n }{ { y }_{ i } } { x }_{ i } \left( Pa \right) \left( Ad \right)$ & $\sum _{ i=1 }^{ n }{ { x }_{ i }^{ 2 } }{ \left( Ad \right)  }^{ 2 }$ \\ 
\hline 
7 & $4.28\times { 10 }^{ -3 }$& $5.76\times { 10 }^{ 9 }$ & $450409.14$ & $5.8171\times { 10 }^{ -5 }$ \\ 
\hline 
\end{tabular}
\end{figure 6} 
\\
\medskip
De donde los coeficientes ser\'{a}n de la forma:
\[a=\frac { n\sum _{ i=1 }^{ n }{ { x }_{ i }{ y }_{ i } } -\sum _{ i=1 }^{ n }{ { x }_{ i } } \sum _{ i=1 }^{ n }{ { y }_{ i } }  }{ n\sum _{ i=1 }^{ n }{ { x }_{ i }^{ 2 } } -{ \left( \sum _{ i=1 }^{ n }{ { x }_{ i } }  \right)  }^{ 2 } } \quad y\quad b=\frac { \sum _{ i=1 }^{ n }{ { x }_{ i }^{ 2 } } \sum _{ i=1 }^{ n }{ { y }_{ i } } -\sum _{ i=1 }^{ n }{ { x }_{ i }{ y }_{ i } } \sum _{ i=1 }^{ n }{ { x }_{ i } }  }{ n\sum _{ i=1 }^{ n }{ { x }_{ i }^{ 2 } } -{ \left( \sum _{ i=1 }^{ n }{ { x }_{ i } }  \right)  }^{ 2 } }.\]

Sustituyendo los valores queda:
\[a_{ 1 }=\frac { (7\times 450409.14\left[ Pa \right] \left[ Ad \right] )-(4.28{ 10 }^{ -3 }\times 5.76{ 10 }^{ 8 }\left[ Pa \right] \left[ Ad \right] ) }{ (7\times 5.8171\times { 10 }^{ 8 }\left[ { Ad }^{ 2 } \right] )-{ \left( 4.28\times { 10 }^{ 3 }\left[ { Ad } \right]  \right)  }^{ 2 } } =1.69\times { 10 }^{ -12 }\frac { \left[ Pa \right]  }{ \left[ Ad \right]  }.\]

\[b_{1}=\frac { (5.8171\times { 10 }^{ 16 }\times 5.76\times { 10 }^{ 8 }\left[ { Ad }^{ 2 } \right] \left[ Pa \right] )-(450409.14\times 4.28\times { 10 }^{ -3 }\left[ Pa \right] \left[ { Ad }^{ 2 } \right] ) }{ (7\times 5.8171\times { 10 }^{ 8 }\left[ { Ad }^{ 2 } \right] )-{ \left( 4.28\times { 10 }^{ 3 }\left[ { Ad } \right]  \right)  }^{ 2 } } ={ 8.23\times 10 }^{ 7 }\left[ Pa \right] .\]
Finalmente queda el modelos propuesto:
\[{ Y }_{ 1 }= { 8.23\times 10 }^{ 7 } + 1.69\times { 10 }^{ -12 }x\quad \rightarrow(1.1) \]
\\
Como la pendiente de la recta a tangente a la curva misma nos representa el m\'{o}dulo de Young , en el cas por su puesto para la gr\'{a}fica $\varepsilon$t VS (DU)t, es decir:
\[m=b=tan(\theta)= \frac { \varepsilon t }{ (Du)t } =\quad Y\quad (M\'{o}dulo\quad de\quad Young).\]
Y as\'{i}, podemos decir, por definici\'{o}n, que el m\'{o}duo de Young del material es: $Y_{1}=1.69\times { 10 }^{ -12 }\left[ Pa \right] .$
\\
\medskip
\\
\\Tabla de entrada 2.
\\
\\
\begin{figure 6}
\centering
\begin{tabular}{|c|c|c|c|c|}
\hline 
 n & $\sum _{ j=1 }^{ n }{ { x }_{ j } } $(Ad) & $\sum _{ j=1 }^{ n }{ { y }_{ j } } $ (Pa) & $ \sum _{ j=1 }^{ n }{ { y }_{ j } } { x }_{ j } \left( Pa \right) \left( Ad \right)$ & $\sum _{ j=1 }^{ n }{ { x }_{ j }^{ 2 } }{ \left( Ad \right)  }^{ 2 }$ \\ 
\hline 
6 & $1.12\times { 10 }^{ -2 }$& $1.29\times { 10 }^{ 9 }$ & $2378050$ & $2.09843\times { 10 }^{ -5 }$ \\ 
\hline 
\end{tabular}
\end{figure 6} 
\\
\medskip
De donde los coeficientes ser\'{a}n de la forma:
\[a=\frac { n\sum _{ j=1 }^{ n }{ { x }_{ j }{ y }_{ j } } -\sum _{ j=1 }^{ n }{ { x }_{ j } } \sum _{ j=1 }^{ n }{ { y }_{ j } }  }{ n\sum _{ j=1 }^{ n }{ { x }_{ j }^{ 2 } } -{ \left( \sum _{ j=1 }^{ n }{ { x }_{ j } }  \right)  }^{ 2 } } \quad y\quad b=\frac { \sum _{ j=1 }^{ n }{ { x }_{ j }^{ 2 } } \sum _{ j=1 }^{ n }{ { y }_{ j } } -\sum _{ j=1 }^{ n }{ { x }_{ j }{ y }_{ j } } \sum _{ j=1 }^{ n }{ { x }_{ j } }  }{ n\sum _{ j=1 }^{ n }{ { x }_{ j }^{ 2 } } -{ \left( \sum _{ j=1 }^{ n }{ { x }_{ j } }  \right)  }^{ 2 } }.\]

Sustituyendo los valores queda:
\[{ a }_{ 2 }=\frac { (6\times 2378050\left[ Pa \right] \left[ Ad \right] )-(1.12\times { 10 }^{ -2 }\times 1.26\times { 10 }^{ 9 }\left[ Pa \right] \left[ Ad \right] ) }{ (6\times 2.09843\times { 10 }^{ -5 }\left[ { Ad }^{ 2 } \right] )-{ \left( 1.12\times { 10 }^{ -2 }\left[ { Ad } \right]  \right)  }^{ 2 } } ={ 1.31\times 10 }^{ 11 }\frac { \left[ Pa \right]  }{ \left[ Ad \right]  }.\]

\[{ b }_{ 2 }=\frac { (2.09843{ \times 10 }^{ -5 }\times 1.26\times { \times 10 }^{ 9 }\left[ Pa \right] \left[ { Ad }^{ 2 } \right] )-(2378050\times 1.12\times { 10 }^{ -2 }\left[ Pa \right] \left[ { Ad }^{ 2 } \right] ) }{ (6\times 2.09843\times { 10 }^{ -5 }\left[ { Ad }^{ 2 } \right] )-{ \left( 1.12\times { 10 }^{ -2 }\left[ { Ad } \right]  \right)  }^{ 2 } } ={ -3.41\times 10 }^{ 7 }\left[ Pa \right].\]
Finalmente queda el modelos propuesto:
\[{ Y }_{ 2 }= -3.41\times 10 }^{ 7 } + { 1.31\times 10 }^{ 11 }x\quad \rightarrow(1) \]
\\
Como la pendiente de la recta a tangente a la curva misma nos representa el m\'{o}dulo de Young , en el cas por su puesto para la gr\'{a}fica $\varepsilon$t VS (DU)t, es decir:
\[m=b=tan(\theta)= \frac { \varepsilon t }{ (Du)t } =\quad Y\quad (M\'{o}dulo\quad de\quad Young).\]
Y as\'{i}, podemos decir, por definici\'{o}n, que el m\'{o}duo de Young del material es: $Y_{2}={ 1.31\times 10 }^{ 11 }\left[ Pa \right] .$ Para esto, tenemos dos valores de m\'{o}dulo de Young, con bastante diferencia, as\'{i} que optaamos por tomar su promedio, m\'{a}s expl\'{i}citamente:
\[{ Y }_{ prom }=\frac { { Y }_{ 1 }+{ Y }_{ 2 } }{ 2 } =9.57\times { 10 }^{ 10 }\left[ Pa \right].\]
\\
\medskip
\section*{Error Porcentual.}\\
Los valores verdaderos (Taba 1) de los m\'{o}dulos de Young del cobre es ${ Y }_{ 1 }=11\times { 10 }^{ 10 }Pa$. Entonces, de nuestras mediciones y c\'{a}lculos podemos obtener el error porcentual:

\[{ E }rror \quad porcentual \quad { Y }_{ 1 }=\frac { Error\quad verdadero }{ Valor\quad Verdadero } =\frac { Valor\quad verdadero - Valor\quad aproximado }{ Valor\quad verdadero } \times 100= 13.56 \%.\]
\\
\section*{Ajuste por excel.}\\
Notemos que mientras por medio de c\'{a}alculos pudimos enconrar un modoelo, el programa Excel pudo enontrar otro, es cual se muestra a continuaci\'{o}n.
\\
\begin{figure}[hbtp]
 \centering
 \includegraphics[width=8cm]{../../../../../../Pictures/EXELASTI.jpg} 
 \includegraphics[width=8cm]{../../../../../../Pictures/EXELASTI2.jpg} 
 \caption{Modelo por medio de excel. }
\end{figure}
 \\
 Cuyas ecuaciones son:
 \[{ Y }_{ ex1 }=1\times { 10 }^{ 11 }x+2\times { 10 }^{ 7 }\quad y\quad { Y }_{ ex2 }=1\times { 10 }^{ 11 }x-3\times { 10 }^{ 7 }\]
 As\'{i}:
\[{ Y }_{ prom ex }=\frac { { Y }_{ ex1 }+{ Y }_{ ex2 } }{ 2 } =8.44\times { 10 }^{ 10 }\left[ Pa \right]\]
 \section*{Discusiones.}\\
Al llevar acabo este experimento pudimos notar que hubo ciertas fallas, como por ejemplo: perdimos la continuaci\'{o}n de las pesas, es decir, quitabamos unas para poner m\'{a}s pesadas y as\'{i} sin darnos cuenta perdimos nuestro r\'{e}gimen el\'{a}stico y se comenzaba a deformar nuestro alambre. Los modelos que propusimos tieen un amplio rango de error, adem\'{a}s uno de de un orden mayor, pero as\'{i} falla por un r\'{e}gimen aceptable a comparaci\'{o} de el valor verdadero.
\pagebreak

\section*{Parte 2: M\'{e}todo de Nivel.} \\

\begin{figure}[hbtp]
\centering
\includegraphics[width=6cm]{../../../../../../Pictures/general.jpg}   
\caption{Arreglo general.}
\end{figure}

\begin{figure}[hbtp]
\centering
\includegraphics[width=5cm]{../../../../../../Pictures/jhhhfajsfhf.jpg}   
\caption{Vista lateral del arreglo}
\end{figure}

\pagebreak 

\textbf{Procedimiento.}\\
\\
1.-Medir las const\'{a}ntes lo, x.\\
2.-Colocar una masa en el alambre y registrar el alcance del tornillo, nivelando de nuevo barra por medio de la burbuja.\\
3. Repetir el paso anterior, de 12 a 25 veces.\\
4.- Llenar la tabla.\\
5.- Graficar $\varepsilon$ VS (DU)t y ajustar para calcular la Y de la muestra.\\

\textbf{Resultados.}\\
\\
Comenzamos la pr\'{a}ctica con ell material proporcionado por el equipo de laboratorio, procedimos a medir algunas de nuestras constantes como lo son; la elongaci\'{o}n inicial del material ll\'{a}mese lat\'{o}n, procedimos a medir el \'{a}rea transversal del material por medio de un tornillo microm\'{e}trico y tambi\'{e}n la medida inicial del instrumento con medida angular. \\
\medskip
\\
\medskip
\medskip
\caption{Tabla 1.} 
\\
\begin{tabular}{|c|c|c|c|c|}
\hline 
Di\'{a}metro(m) & A(${ m }^{ 2 }$) & lo(m)& $\varphi$ (m)\\ 
\hline 
${ 1.34\times 10 }^{ -3 }$ & ${ 3.84\times 10 }^{ -5 }$ & 1.438 & ${ 3.5\times 10 }^{ -4 }$ \\
\hline 
\end{tabular}\\


\medskip

Llenamos la Tabla 2 proporcionada para despu\'{e}s graficar el esfuerzo vs deformaci\'{o}n unitaria y ajustar para calcular el Y de cada muestra.\\
\textbf{Tabla 2.}\\
\\
\begin{figure 2}
\centering
\includegraphics[width=12cm]{../../../../../../Pictures/popopop.jpg} 
\\
\caption{De la cual extragimos los datos de la siguiente tabla para poder hacer la gr\'{a}fica $\varepsilon$ VS (DU)t.}
\end{figure 2}

\begin{figure 3}
\centering
\caption{\textbf{Tabla 3.}}
\\
\includegraphics[width=6cm]{../../../../../../Pictures/fgqhwregnetw.jpg} 
\\
\caption{Datos extra\'{i}dos.}
\end{figure 3}


\\
\section*{Gr\'{a}fica de dispersi\'{o}n}
\\
\\
\begin{figure 5}
\centering
\includegraphics[width=9cm]{../../../../../../Pictures/rrrrrrrrrrrrr.jpg} 
\\ 
\caption{Gr\'{a}fico de dispersi\'o}n para el m\'{e}todo de nivel.}
\end{figure 5}
\\
\\

Claramente se ve que en su gr\'{a}fico de dispersi\'{o}n se observa que los dos bloques marcados, de acuerdo a nuestro marco te\'{o}rico cumple la Ley de Hooke s\'{o}lo en la parte el\'{a}stica y hasta cierto punto, marcado en la figura de arriba, tenemos un comportamiento pl\'{a}sico. As\'{i} que justificamos la manera en que abordaremos el problema , es decir, discriminaremos los \'{u}ltimos 5 puntos.  
\section*{Ajuste de datos.}\\
Por el Apr\'{e}ndice 1 podemos hacer el respectivo ajuste por el m\'{e}todo de m\'{i}nimos cuadrados para encontrar un modelo lineal $Y=ax +b$ tales que $\left( { x }_{ i },{ y }_{ i } \right) \rightarrow \left( Du(Ad),{ \sigma  }_{ t }(Pa) \right) $  para cada uno de los datos de cada exprimento se tendr\'{a} su tabla de entrada y los siguientes coeficientes a y b, de donde:
\\
\medskip
\[a=\frac { n\sum _{ i=1 }^{ n }{ { x }_{ i }{ y }_{ i } } -\sum _{ i=1 }^{ n }{ { x }_{ i } } \sum _{ i=1 }^{ n }{ { y }_{ i } }  }{ n\sum _{ i=1 }^{ n }{ { x }_{ i }^{ 2 } } -{ \left( \sum _{ i=1 }^{ n }{ { x }_{ i } }  \right)  }^{ 2 } } \quad y\quad b=\frac { \sum _{ i=1 }^{ n }{ { x }_{ i }^{ 2 } } \sum _{ i=1 }^{ n }{ { y }_{ i } } -\sum _{ i=1 }^{ n }{ { x }_{ i }{ y }_{ i } } \sum _{ i=1 }^{ n }{ { x }_{ i } }  }{ n\sum _{ i=1 }^{ n }{ { x }_{ i }^{ 2 } } -{ \left( \sum _{ i=1 }^{ n }{ { x }_{ i } }  \right)  }^{ 2 } }.\]
\medskip
\\
\caption{Tabla de entrada 2.}
\\
\begin{figure 6}
\\
\medskip  
\medskip 
\centering
\begin{tabular}{|c|c|c|c|c|}
\hline 
 n & $\sum _{ i=1 }^{ n }{ { x }_{ i } } $(Ad) & $\sum _{ i=1 }^{ n }{ { y }_{ i } } $ (Pa) & $ \sum _{ i=1 }^{ n }{ { y }_{ i } } { x }_{ i }$ (Ad)(Pa) & $\sum _{ i=1 }^{ n }{ { x }_{ i }^{ 2 } }\left( Ad \right)  }^{ 2 }$ \\ 
\hline 
6 & $5.67\times { 10 }^{ -3 }$& $1.92\times { 10 }^{ 6 } $ & 1997.162 & $5.85996\times { 10 }^{ -6 }$\\ 
\hline 
\end{tabular} 
\end{figure 6}
\\

\end{figure 6} 
\\
\medskip
Sustituyendo los valores queda:

\[a=\frac { (6\times 1997.162\left[ Ad \right] \left[ Pa \right] )-(5.67\times { 10 }^{ -3 }\times 1.92\times { 10 }^{ 6 }\left[ Ad \right] \left[ Pa \right] ) }{ (6\times 5.85996\times { 10 }^{ -6 }\left[ { Ad }^{ 2 } \right] )-{ \left( 5.67\times { 10 }^{ -3 }\left[ Ad \right]  \right)  }^{ 2 } } ={ 3.748\times 10 }^{ 8 }\frac { \left[ Pa \right]  }{ \left[ Ad \right]  } .\]

\[\quad b=\frac { (5.85996\times { 10 }^{ -6 }\times 1.92\times { 10 }^{ -6 }\left[ { Ad }^{ 2 } \right] \left[ Pa \right] )-(1997.162\times 5.67\times { 10 }^{ -3 }\left[ { Ad }^{ 2 } \right] \left[ Pa \right] ) }{ (6\times 5.85996\times { 10 }^{ -6 }\left[ { Ad }^{ 2 } \right] )-{ \left( 5.67\times { 10 }^{ -3 }\left[ Ad \right]  \right)  }^{ 2 } } ={ -3.39\times 10 }^{ -4 }\left[ Pa \right].\]
Finalmente queda el modelos propuesto:
\[{ Y }_{ 1 }= { -3.39\times 10 }^{ -4 } + { 3.748\times 10 }^{ 8 }x\quad \rightarrow(2) \]

Como la pendiente de la recta a tangente a la curva misma nos representa el m\'{o}dulo de Young , en el cas por su puesto para la gr\'{a}fica $\varepsilon$t VS (DU)t, es decir:
\[m=b=tan(\theta)= \frac { \varepsilon t }{ (Du)t } =\quad Y\quad (M\'{o}dulo\quad de\quad Young).\]
Y as\'{i}, podemos decir, por definici\'{o}n, que el m\'{o}duo de Young del material es: $Y={ 3.748\times 10 }^{ 8 }(Pa).$
\\
\section*{Error Porcentual.}\\
Los valores verdaderos de los m\'{o}dulos de Young del lat\'{o}n es ${ Y }_{ 1 }=5.48\times { 10 }^{ 8 }Pa$. Entonces, de nuestras mediciones y c\'{a}lculos podemos obtener el error porcentual:

\[{ E }rror-porcentual-{ Y }_{ 1 }=\frac { Error\quad verdadero }{ Valor\quad Verdadero } =\frac { Valor\quad verdadero - Valor\quad aproximado }{ Valor\quad verdadero } \times 100=23.42 \% \]
\\
\section*{Ajuste por excel.}\\
Notemos que mientras por medio de c\'{a}alculos pudimos enconrar un modelo, el programa Excel pudo enontrar otro, es cual se muestra a continuaci\'{o}n.

\begin{figure}[hbtp]
 \centering
 \includegraphics[width=9cm]{../../../../../../Pictures/fefeffef.jpg} 
 \caption{Modelo por medio de excel. }
\end{figure} 
 Cuya ecuaci\'{o}n es:
 \[{ Y }_{ ex}= 4\times { 10 }^{ 8 }x - 33908. \quad \rightarrow(2.1) \]
 
 \section*{Discusiones.}\\
 Al llevar acabo este experimento pudimos notar que hubo ciertas fallas, como por ejemplo: perdimos la continuaci\'{o}n de las pesas, es decir, quitabamos unas para poner m\'{a}s pesadas y as\'{i} sin darnos cuenta perdimos nuestro r\'{e}gimen el\'{a}stico y se comenzaba a deformar nuestro alambre. Los modelos que propusimos tienen un amplio rango de error, adem\'{a}s uno de de un orden mayor, pero as\'{i} falla por un r\'{e}gimen aceptable a comparaci\'{o} de el valor verdadero. Desde la linea vertical hasta el final del conjunsto de puntos discretos en la Figura anterior, el esfuerzo y la deformaci\'{o}n ya no son proporcionales, y no se obedece la ley de Hooke, la deformaci\'{o}n sigue aumentando, el material sufri\'{o} una deformaci\'{o}n irreversible y adquiri\'{o} un ajuste permanente. El comportamiento del material entre la regi\'{o}n se denomina flujo pl\'{a}stico o deformaci\'{o}n pl\'{a}stica. Una deformaci\'{o}n pl\'{a}stica es irreversible; si se elimina el esfuerzo, el material no vuelve a su estado original. 
 

\end{document}