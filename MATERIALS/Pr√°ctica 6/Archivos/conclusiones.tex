 \documentclass[10pt,a4paper]{article}
\usepackage[latin2]{inputenc}
\usepackage[spanish]{babel}
\usepackage{amsmath}
\usepackage{amsfonts}
\usepackage{amssymb}
\usepackage{makeidx}
\usepackage{graphicx}
\usepackage[left=2cm,right=2cm,top=2cm,bottom=2cm]{geometry}

\begin{document}

\section*{Conclusiones:}\\
En general, al realizar la pr\'{a}ctica pudimos observar claramente los fen\'{o}menos de propagaci\'{o}n de ondas, adem\'{a}s de los fen\'{o}menos de reflexi\'{o}n y refracci\'{o}n, los principios de Houygens-Fresnel, interferencia y dfracci\'{o}n de ondas, y la propagaci\'{o}n de estas sometidas en disintos obst\'{a}culos peque\~{n}os, e identificamos bien cada caso y cada fen\'{o}meno de manera cualitativa. 

\section*{Referencias:}\\
\\
\medskip
\\
\\1.- Bit\'{a}cora de laboratorio de Flores Rodr\'{i}guez Jaziel David.
\\
2.- Manual de pr\'{a}cticas auxilar. Autor: Fco. Havez Varela y las notas del profesor Salvador Tirado Guerra.
\\
3.- Fisica Universitaria - Sears - Zemansky - 12ava Edici\'{o}n - Cap\'{i}tulo l5 -2009.\\
4.- http://azufre.quimica.upmadrid.es/DetEst/Tema2-AAP.pdf\\
\end{document}