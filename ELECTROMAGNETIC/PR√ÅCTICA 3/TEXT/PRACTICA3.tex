\documentclass[11pt,a4paper]{article}
\usepackage[latin1]{inputenc}
\usepackage[spanish]{babel}
\usepackage{amsmath}
\usepackage{amsfonts}
\usepackage{amssymb}
\usepackage{makeidx}
\usepackage{graphicx}
\usepackage{lmodern}
\usepackage[left=2cm,right=2cm,top=2cm,bottom=2cm]{geometry}
\author{Flores Rodguez Jaziel David }
\title{Práctica 1}
\begin{document}

\tableofcontents
%----------------------------------------------------------------------------------------
%	SECTION 1
%----------------------------------------------------------------------------------------
\medskip
\section{Resumen.}
\
Comprender el concepto de campo el\'{e}ctrico de manera factual, en el cual se pueda identificar sus partes como los son su intesidad (magnitud) y sentido (su orientaci\'{o}n) adem\'{a}s de su naturaleza el\'{e}ctrica. Asimismo definir las lineas de campo el\'{e}ctrico y la diferencia de potencial, para que as\'{i} se comprenda el fen\'{o}meno de las superficies equipotenciales, es decir explicar de manera objetiva y descriptiva en qu\'{e} consisten \'{e}stos fen\'{o}menos. 
\\
\\
\medskip
Palabras clave: Campo e\'{e}ctrico, diferencia de potencial, superficie equipotencial.

%----------------------------------------------------------------------------------------
%	SECTION 2
%----------------------------------------------------------------------------------------

\section{Objectivo.}
a) Examinar algunos patrones de campo el\'{e}ctrico.
\\
\\
b)Entender que son y como se obtienen la superficies equipotenciales en un campo el\'{e}ctrico.
\\
\\
c)Estudiar el campo el\'{e}ctrico producido en un capacitor de placas planas y paralelas.
\pagebreak


%----------------------------------------------------------------------------------------
%	SECTION 3
%----------------------------------------------------------------------------------------
\section{Metodolog\'{i}a.}

El \textbf{campo el\'{e}ctrico} (regi\'{o}n del espacio en la que interact\'{u}a la fuerza el\'{e}ctrica) es un campo f\'{i}sico que se representa, mediante un modelo que describe la interacci\'{o}n entre cuerpos y sistemas con propiedades de naturaleza el\'{e}ctrica. ​Se describe como un campo vectorial en el cual una carga el\'{e}ctrica puntual de valor q sufre los efectos de una fuerza el\'{e}ctrica F dada por la siguiente ecuaci\'{o}n:
\[F=qE\]
Los campos el\'{e}ctricos pueden tener su origen tanto en cargas el\'{e}ctricas como en campos magn\'{e}ticos variables. Las primeras descripciones de los fen\'{o}menos el\'{e}ctricos, como la ley de Coulomb, solo ten\'{i}an en cuenta las cargas el\'{e}ctricas, pero las investigaciones de Michael Faraday y los estudios posteriores de James Clerk Maxwell permitieron establecer las leyes completas en las que tambi\'{e}n se tiene en cuenta la variaci\'{o}n del campo magn\'{e}tico.
\\
Esta definici\'{o}n general indica que el campo no es directamente medible, sino que lo que es observable es su efecto sobre alguna carga colocada en su seno. La idea de campo el\'{e}ctrico fue propuesta por Faraday al demostrar el principio de inducci\'{o}n electromagn\'{e}tica en el a\~{n}o 1833.
\\
\\
\textbf{Lineas Fuerza}:\\
Faraday introdujo el concepto de l\'{i}neas de fuerza, que nos sirven para visualizar la intensidad y la direcci\'{o}n de un campo el\'{e}ctrico. Las l\'{i}neas de fuerza indican la direcci\'{o}n que una carga de prueba positiva experimenta en un campo el\'{e}ctrico. Estas l\'{i}neas se representan con flechas que indican la direcci\'{o}n del campo el\'{e}ctrico.

\\
\begin{figure}[hbtp]
\centering
\includegraphics[width=7cm]{../../../../../../Pictures/uuuuusususs.jpg}
\end{figure}
\\

Si se desea dibujar las l\'{i}neas de campo el\'{e}ctrico en una regi\'{o}n pr\'{o}xima a una carga puntual positiva Q, se pone una carga de prueba positiva en un punto A y se observa que tipo de fuerza experimenta. La carga de prueba positiva ser\'{a} repelida por la carga Q del mismo signo y ser\'{a} radial movi\'{e}ndose de Q hacia A, por lo tanto las l\'{i}neas de campo deben salir de +Q en forma radial. De manera similar se puede encontrar las l\'{i}neas de fuerza el\'{e}ctrica cerca de una carga negativa Q. En este caso la carga de prueba colocada en A es atra\'{i}da por la carga negativa Q, por tanto la fuerza y las l\'{i}neas de campo el\'{e}ctrico est\'{a}n dirigidas radialmente hacia adentro. De donde se puede concluir que: Las l\'{i}neas de campo emergen de las cargas positivas y entran en las cargas negativas.
\\
\\
\textbf{Diferenci de potencial}:\\
Diferencia de potencial. Si se requiere hacer trabajo para llevar una carga de un punto a otro dentro de un campo el\'{e}ctrico se dice que hay una diferencia de potencial. Este trabajo es independiente de la trayectoria recorrida entre los dos puntos. La diferencia de potencial entre dos puntos de un campo el\'{e}ctrico, se define como la raz\'{o}n del trabajo hecho para mover una carga peque\~{n}a positiva entre los dos puntos considerados, esto es:
\[V=\frac { W }{ q } =\frac { Fd }{ q } =Ed\]
Donde V es la diferencia de potencial en volts, si W esta medido en Joule y q en Coulomb.\\
\\
\\
\\
\textbf{Superficies equipotenciales}:\\
Es posible encontrar varios puntos que tengan el mismo potencial de un campo el\'{e}ctrico, que si se unen, es lo que se conoce como l\'{i}nea o superficie equipotencial como se muestra en la figura:
\\
\begin{figure}[hbtp]
\centering
\includegraphics[width=7cm]{../../../../../../Pictures/pppppp.jpg}
\end{figure}

\\

%----------------------------------------------------------------------------------------
%	SECTION 4
%----------------------------------------------------------------------------------------

\section{Instrumentaci\'{o}n.}\\
1- Amplificador Lineal.\\
2- Mult\'{i}metro.\\
3- Balanza de torsi\'{o}n. \\
4- Capacitor de placas planas. \\
5. Fuente de alto voltaje.\\
6. L\'{a}mpara de iluminaci\'{o}n.\\
7. Tripie.\\
8. M\'{a}quina elecrost\'{a}tica.\\
9. Electroscopio de Wulf.\\
10. Proyector de acetatos.\\
11. Patrones de campo el\'{e}ctrico.\\

%----------------------------------------------------------------------------------------
%	SECTION 5
%----------------------------------------------------------------------------------------


\section{Datos y Resultados.}\\

\textbf{(a)Lineas de Fuerza:}\\
Ponga uno de los patrones para visualizar las l\'{i}neas de campo el\'{e}ctrico, sobre el proyector de acetatos y disperse semillas de pasto entre los electrodos, cargue cada uno de los electrodos con carga de igual o distinto signo por medio de la fuente de alto voltaje, observe en la pantalla las formas de las l\'{i}neas de campo generadas y explique los resultados. Cambie los diferentes patrones proporcionados y repita lo anterior.

\begin{figure}[hbtp]
\centering 
\includegraphics[width=8cm]{../../../../../../Pictures/IMG_20171003_123352436.jpg} 
\caption{Aqu\'{i} se observa que el la caracter\'{i}stica que se le indujo a la superficie afect\'{o} a las inmediaciones del medio, en este caso se ve afectado con las semillas apuntando hacia una sola direci\'{o}n uniforme, o bien la semilla actuaba como las lineas de fuerza en el campo el\'{e}ctrico.}
\end{figure}


\begin{figure}[hbtp]
\centering
\includegraphics[width=8cm]{../../../../../../Pictures/IMG_20171003_123651482.jpg} 
\caption{Se observa ahora que al electrizar una forma aribiraria como esta las semillas que actuan como lineas de fuerza y apuntan hacia donde hay puntas.
}
\end{figure}

\begin{figure}[hbtp]
\centering
\includegraphics[width=8cm]{../../../../../../Pictures/IMG_20171003_124356462.jpg} 
\caption{Se observa que en esta forma circular las semillas act\'{u}an como las lineas de fuerza, esta vez actuando radialmente hacia adentro la direcci\'{o}n del campo.}
\end{figure}


\pagebreak

\section*{(b)Campo el\'{e}ctrico uniforme y constante:}\\
\textbf{Procedimiento 1.}
1) Con ayuda de la balanza de torsi\'{o}n calcular la fuerza F ejercida sobre una carga de prueba q producida por un campo el\'{e}ctrico generado por un capacitor de placas planas y paralelas al cual se le aplica un voltaje.\\
2) Arme el equipo como se indica y separe las placas del capacitor d = 5 cm. Y marque la posici\'{o}n de equilibrio del haz luminoso sobre la pantalla.\\
3) Cargue la paleta de prueba que se encuentra en el centro del capacitor, pormedio del portador de carga proporcionado, aplique el voltaje al capacitor de entre 300 a 1500 volt el que usted decida, para producir el campo el\'{e}ctrico, mida la desviaci\'{o}n x del haz luminoso y calcula la fuerza F obtenga los siguientes datos y llene su tabla.\\
4) Construya una gr\'{a}fica fuerza - carga, proponga la funci\'{o}n F= F(q) de la curva que mejor represente los puntos experimentales y ajuste dicha funci\'{o}n por el m\'{e}todo de m\'{i}nimos cuadrados, trace la curva ajustada en la gr\'{a}fica Fq.
\\
\begin{figure}[hbtp]
\centering
\includegraphics[width=7cm]{../../../../../../Pictures/IMG_20171003_125008153.jpg} 
\caption{Areglo experimental.}
\end{figure}
\\
Y luego obtuvimos los siguientes datos:
\begin{figure}[hbtp]
\centering
\includegraphics[width=6cm]{../../../../../../Pictures/HHHHHHHHHHHHHHHH.jpg} 
\caption{Tabla1. Carga y distancia var\'{i}a.}
\end{figure}
\\
Y as\'{i} obtenemos la siguiente gr\'{a}fica de dispersi\'{o}n se obtine como sigue:
\begin{figure}[hbtp]
\centering
\includegraphics[width=14cm]{../../../../../../Pictures/OOOODADADA.jpg} 
\caption{Gr\'{a}fico de dispersi\'{o}n de la Fuerza-Carga.}
\end{figure}
\\
\textbf{Ajuste de datos.} 
Por el Apr\'{e}ndice 1 podemos hacer el respectivo ajuste por el m\'{e}todo de m\'{i}nimos cuadrados para encontrarun modelo lineal Y = ax + b tales tales que $\left( { x }_{ i },{ y }_{ i } \right) \rightarrow \left( Q(  C  ),{F (N))$  de los datos del experimento, cuya tablas de entrada y modelo es el siguiente:
\\
\textbf{Tabla de entrada.}
\\
\begin{figure 6}
\centering
\begin{tabular}{|c|c|c|c|c|}
\hline 
n&$\sum _{ i=1 }^{ n }{ { x }_{ i } } (C  )$ & $\sum _{ i=1 }^{ n }{ { y }_{ i } }(N)$ & $ \sum _{ i=1 }^{ n }{ { y }_{ i } } { x }_{ i }(N \cdot C )$ & $\sum _{ i=1 }^{ n }{ { x }_{ i }^{ 2 } }(C ^{ 2 })$ \\ 

\hline 
7&$ 1.19\times { 10 }^{ -7 } $& $7.07\times { 10 }^{ -3 }$ & $1.33\times { 10 }^{ -10 }$ & $2.28\times { 10 }^{ -15 }$\\ 
\hline 
\end{tabular}
\end{figure 6} 
\\
\\
De donde:
\[a=\frac { n\sum _{ i=1 }^{ n }{ { x }_{ i }{ y }_{ i } } -\sum _{ i=1 }^{ n }{ { x }_{ i } } \sum _{ i=1 }^{ n }{ { y }_{ i } }  }{ n\sum _{ i=1 }^{ n }{ { x }_{ i }^{ 2 } } -{ \left( \sum _{ i=1 }^{ n }{ { x }_{ i } }  \right)  }^{ 2 } } \quad y\quad b=\frac { \sum _{ i=1 }^{ n }{ { x }_{ i }^{ 2 } } \sum _{ i=1 }^{ n }{ { y }_{ i } } -\sum _{ i=1 }^{ n }{ { x }_{ i }{ y }_{ i } } \sum _{ i=1 }^{ n }{ { x }_{ i } }  }{ n\sum _{ i=1 }^{ n }{ { x }_{ i }^{ 2 } } -{ \left( \sum _{ i=1 }^{ n }{ { x }_{ i } }  \right)  }^{ 2 } }.\]
Sustituyendo los valores queda:

\[{ a }=\frac { (7\times 1,33\times { 10 }^{ -10 }\left[ N{ C } \right] )-(1,19\times { 10 }^{ -7 }\times 7.07\times { 10 }^{ -3 }\left[ N{ C } \right] ) }{ (7\times 2.28\times { 10 }^{ -15 }\left[ { C }^{ 2 } \right] )-{ \left( 1.19\times { 10 }^{ -7 }\left[ C \right]  \right)  }^{ 2 } } =48016\left[ { N }/{ { C } } \right] .\]

\[ { b }=\frac { (2,28\times { 10 }^{ -15 }\times 7.07{ \times 10 }^{ -3 }\left[ { N{ C }^{ 2 } } \right] )-(1,33\times { 10 }^{ -10 }\times 1,19{ \times 10 }^{ -7 }\left[ { N{ C }^{ 2 } } \right] ) }{ (9\times 5.84\times { 10 }^{ -2 }\left[ { C }^{ 4 } \right] )-{ \left( 2.61\times { 10 }^{ -15 }\left[ { C }^{ 2 } \right]  \right)  }^{ 2 } } = 2\times { 10 }^{ -4 }\left[ \frac { N }{ { C }^{ 2 } }  \right] .\]
Finalmente queda el modelos propuesto:
\[ F(Q)= 48016(Q) + 2\times { 10 }^{ -4 } \quad \rightarrow(2) \]
De (2)  podemos graficar sus correspondientes lineas de tendencia, a continuaci\'{o}n vamos a graficar su modelo y a tratarlo con m\'{a}s detalle
\section*{Ajuste por excel.}\\
Notemos que mientras por medio de c\'{a}alculos pudimos enconrar un modoelo, el programa Excel pudo enontrar otro, es cual se muestra a continuaci\'{o}n.
\\
\begin{figure}[hbtp]
 \centering
\includegraphics[width=14cm]{../../../../../../Pictures/UYSSDACSDHGsd.jpg} 
\end{figure}
\\

%----------------------------------------------------------------------------------------
%	SECTION 6
%----------------------------------------------------------------------------------------

\section{Discusi\'{o}n.}
En este procedimiento enontramos una funci\'{o}n de la fuerza que depende de la carga, aqu\'{i} usamos los resultados e las pr\'{a}cticas anteriores, es decir usamos la constante de torsi\'{o}n k, y usamos la ley de Coulomb b\'{s}sicamente. Para m\'{i} esta parte es ide\'{e}ntica ala aley de Coulomb, s\'{o}lo que con otro enfoque, es decir encontramos una expresi\'{o}n de la fuerza en cada punto del espacio. 

%----------------------------------------------------------------------------------------
%	SECTION 7e
%----------------------------------------------------------------------------------------

\section{Conclusiones.}
En esta pr\'{a}ctica se comprendi\'{o} el fen\'{o}meno de los campos y como \'{e}stos interact\'{u}an por el espacio y le dotamos de una estructura matem\'{a}tica, es decir a estos campos llamados vectoriales se les asigna una interpretaci\'{o}n matem\'{a}tica y f\'{i}isica para plantear nuevas formas de entender procesos f\'{i}sicos como los son interacci\'{o}n de cargas. 
%----------------------------------------------------------------------------------------
%	SECTION 9
%----------------------------------------------------------------------------------------
 \section{Bibliograf\'{i}a.}
1.- https://es.wikipedia.org/wiki/Ley_de_Coulomb.\\
2.-https://es.wikipedia.org/wiki/Balanza_de_torsiC3B3n.\\
3.-https://es.wikipedia.org/wiki/MultC3ADmetro.\\
4.-https://sites.google.com/site/labenriquesalgadoruiz/home/politecnico-1/fisica-iii.\\
5.-Resnick/Halliday/Krane. Fundamentos de F\'{i}sica. Volumen 2. Edici\'{o}n 6, extendida. CESA\\
6.- https://es.wikipedia.org/wiki/Campo_electri78co.\\

%----------------------------------------------------------------------------------------
%	SECTION 10
%----------------------------------------------------------------------------------------

\end{document}