\documentclass[11pt,a4paper]{article}
\usepackage[latin2]{inputenc}
\usepackage[spanish]{babel}
\usepackage{amsmath}
\usepackage{amsfonts}
\usepackage{amssymb}
\usepackage{makeidx}
\usepackage{graphicx}
\usepackage[left=2cm,right=2cm,top=2cm,bottom=2cm]{geometry}

\begin{document}
\section*{Elasticidad y plasticidad}
La ley de Hooke (la proporcionalidad del esfuerzo y la deformaci\'{o}n el\'{a}stica) tiene un intervalo de validez limitado. En las secciones anteriores usamos frases como “si las fuerzas son tan peque\~{n}as que se obedece la ley de Hooke”. Cu\'{a}les son exactamente las limitaciones de la ley de Hooke? Sabemos que: si tiramos de cualquier cosa, la aplastamos o la torcemos lo suficiente, se doblar\'{a} o romper\'{a}. Podemos ser m\'{a}s precisos que eso?

Examinemos de nuevo el esfuerzo y la deformaci\'{o}n por tensi\'{o}n. Supongamos que graficamos el esfuerzo en funci\'{o}n de la deformaci\'{o}n. Si se obedece la ley de Hooke, la gr\'{a}fica ser\'{a} una recta con pendiente igual al m\'{o}dulo de Young. La Figura 1 muestra una gr\'{a}fica esfuerzo deformaci\'{o}n t\'{i}pica de un metal como cobre o hierro blando. La deformaci\'{o}n se muestra como porcentaje de alargamiento; la escala horizontal no es uniforme despu\'{e}s de la primera porci\'{o}n de la curva, hasta una deformaci\'{o}n menor que el 1 porcieto. La primera porci\'{o}n es una l\'{i}nea recta, que indica un comportamiento de ley de Hooke con el esfuerzo directamente proporcional a la deformaci\'{o}n. Esta porci\'{o}n rectil\'{i}nea termina en el punto a; el esfuerzo en este punto se denomina l\'{i}mite proporcional.
\\
\begin{figure}[hbtp]
\centering
\includegraphics[width=6cm]{../../../Pictures/esa1.jpg}
\caption{ Diagrama de esfuerzo-deformaci\'{o}n t\'{i}pico para un metal d\'{u}ctil sometido a tensi\'{o}n.}
\end{figure}
\\
Desde a hasta b, el esfuerzo y la deformaci\'{o}n ya no son proporcionales, y no se obedece la ley de Hooke. Si la carga se retira gradualmente, partiendo de cualquier punto entre O y b, la curva se sigue a la inversa hasta que el material recupera su longitud original. La deformaci\'{o}n es reversible, y las fuerzas son conservativas; la energ\'{i}a introducida en el material para causar la deformaci\'{o}n se recupera cuando se elimina el esfuerzo. En la regi\'{o}n Ob decimos que el material tiene comportamiento el\'{a}stico. El punto b, donde termina esta regi\'{o}n, es el punto de relajamiento; el esfuerzo en este punto se denomina l\'{i}mite el\'{a}stico. Si aumentamos el esfuerzo m\'{a}s all\'{a} del punto b, la deformaci\'{o}n sigue aumentando; pero si retiramos la carga en un punto m\'{a}s all\'{a} de b, digamos c, el material no recupera su longitud original, sino que sigue la l\'{i}nea roja de la Figura 1. La longitud con cero esfuerzo ahora es mayor que la original; el material sufri\'{o} una deformaci\'{o}n irreversible y adquiri\'{o} un ajuste permanente. Un aumento de la carga m\'{a}s all\'{a} de c produce un aumento grande en la deformaci\'{o}n con un incremento relativamente peque\~{n}o del esfuerzo, hasta llegar a un punto d en el que se presenta la fractura. El comportamiento del material entre b y d se denomina flujo pl\'{a}stico o deformaci\'{o}n pl\'{a}stica. Una deformaci\'{o}n pl\'{a}stica es irreversible; si se elimina el esfuerzo, el material no vuelve a su estado original.En algunos materiales, se presenta una deformaci\'{o}n pl\'{a}stica considerable entre el l\'{i}mite el\'{a}stico y el punto de fractura, como aquel cuyas propiedades se grafican en la Figura 1. Decimos que tales materiales son d\'{u}ctiles. En cambio, si la fractura se presenta poco despu\'{e}s de rebasarse el l\'{i}mite el\'{a}stico, decimos que el material es quebradizo. Un alambre de hierro blando que puede sufrir un estiramiento permanente considerable sin romperse es d\'{u}ctil; una cuerda de acero de piano que se rompe poco despu\'{e}s de alcanzar su l\'{i}mite el\'{a}stico es quebradiza. Algo muy curioso puede ocurrir cuando un objeto se estira y luego se deja relajar.\\

Un ejemplo se presenta en la figura 2, que es una curva de esfuerzo-deformaci\'{o}n de hule vulcanizado estirado a m\'{a}s de siete veces su longitud original. El esfuerzo no es proporcional a la deformaci\'{o}n pero el comportamiento es el\'{a}stico porque, al retirarse la carga, el material recupera su longitud original. Sin embargo, el material sigue curvas diferentes cuando aumenta y cuando disminuye el esfuerzo. Esto se denomina hist\'{e}resis el\'{a}stica. El trabajo efectuado por el material cuando recupera su forma original es menor que el requerido para deformarlo; hay fuerzas no conservativas asociadas con la fricci\'{o}n interna. El hule con hist\'{e}resis el\'{a}stica grande es muy \'{u}til para absorbervibraciones, como en los soportes de motores y bujes amortiguadores para  autom\'{o}vil. El esfuerzo requerido para causar la fractura de un material se denomina esfuerzo de rotura, resistencia l\'{i}mite o (para el esfuerzo de tensi\'{o}n) resistencia a la tensi\'{o}n. Dos materiales, digamos dos tipos de acero, pueden tener constantes el\'{a}sticas muy similares pero esfuerzos de rotura muy distintos. 
\\
\begin{figure}[hbtp]
\centering
\includegraphics[width=6cm]{../../../Pictures/esa2.jpg}
\caption{ Diagrama esfuerzo-deformaci\'{o}n t\'{i}pico para hule vulcanizado. Las curvas son diferentes para un aumento y unadisminuci\'{o}n del esfuerzo; este fen\'{o}meno se denomina hist\'{e}resis el\'{a}stica}
\end{figure}
\\
La tabla 3 presenta valores t\'{i}picos de esfuerzo de rotura para varios materiales en tensi\'{o}n. El factor de conversi\'{o}n $6.9\times { 10 }^{ 8 }Pa=100,00\quad psi$ ayuda a poner estos n\'{u}meros en perspectiva. Por ejemplo, si el esfuerzo de rotura de cierto acero es de $6.9\times { 10 }^{ 8 }Pa$, una barra con secci\'{o}n transversal de $1\quad { in }^{ 2 }$ tendr\'{a} una resistencia a la rotura de 100,000 lb.\\
\\
\begin{figure}[hbtp]
\caption{Esfuerzos de rotura
aproximados.}
\centering
\includegraphics[width=6cm]{../../../Pictures/esa3.jpg}
\end{figure}
\\

\section*{Apr\'{e}ndice 1: M\'{e}todo de los m\'{i}nimos cuadrados.}
Cuando se hace una medici\'{o}n experimental se pueden obtener os datos de un determinado fen\'{o}meno y estos al hacer una dispersi\'{o}n podemos obtener su acomodo en cierta curva por ejemplo una linea reacta. Se busca una  ecuaci\'{o}n de la forma; y= ax+b.\\

Para evaluar a y b. Sean $\left\{ ({ x }_{ i },{ y }_{ i }) \right\}$ ;
\[\varphi =\quad \gamma -{ y }_{ i }=\quad { ax }_{ i }+b-{ y }_{ i }\]
Se eleva a cuadrado para considerar los valores negativosy se toma la media aritm\'{e}tica:
\[\frac { 1 }{ 2 } \sum _{ i=1 }^{ n }{ { { \varphi  }^{ 2 } }_{ i } } =\frac { 1 }{ n } \sum _{ i=1 }^{ n }{ { { ({ ax }_{ i }+b-{ y }_{ i } })^{ 2 } } } \]
Utilizando las derivadas parciaes y igualando a cero para minimizar los distancias tenemos: 
\[\frac { \partial  }{ \partial a } \frac { 1 }{ n } \sum _{ i=1 }^{ n }{ { { \varphi  }^{ 2 } }_{ i } } =0\quad \quad y\quad \frac { \partial  }{ \partial b } \frac { 1 }{ n } \sum _{ i=1 }^{ n }{ { { \varphi  }^{ 2 } }_{ i } } =0\] 
De donde;
\[\frac { \partial  }{ \partial a } \frac { 1 }{ n } \sum _{ i=1 }^{ n }{ { ( }{ ax }_{ i }+b-{ y }_{ i })^{ 2 } } =\frac { 1 }{ n } \frac { \partial  }{ \partial a } \sum _{ i=1 }^{ n }{ { ( }{ ax }_{ i }+b-{ y }_{ i })^{ 2 } }= \frac { 1 }{ n } \sum _{ i=1 }^{ n }{ \frac { \partial  }{ \partial a }  } { ({ ax }_{ i }+b-{ y }_{ i }) }^{ 2 } ;\]
\[ \Rightarrow \frac { 1 }{ n }  \sum _{ i=1 }^{ n }{ { 2( }{ ax }_{ i }+b-{ y }_{ i }){ x }_{ i } } \]
Es decir, que la condici\'{o}n se cumple;
\[\frac { 2 }{ n } \sum _{ i=1 }^{ n }{ \frac { \partial  }{ \partial a }  } { ({ { ax }_{ i }^{ 2 } }+b{ x }_{ i }-{ y }_{ i }{ x }_{ i }) }=0=\frac { 2 }{ n } \left\{ a\sum _{ i=1 }^{ n }{ { x }_{ i }^{ 2 } }  \right +b\sum _{ i=1 }^{ n }{ { x }_{ i } } { -\left \sum _{ i=1 }^{ n }{ { y }_{ i }{ x }_{ i } }  \right\}  }\]
Por lo tanto;
\[\ a\sum _{ i=1 }^{ n }{ { x }_{ i }^{ 2 } } + b\sum _{ i=1 }^{ n }{ { x }_{ i } } -\sum _{ i=1 }^{ n }{ { y }_{ i }{ x }_{ i } } = 0;
\Rightarrow a\sum _{ i=1 }^{ n }{ { x }_{ i }^{ 2 } } + b\sum _{ i=1 }^{ n }{ { x }_{ i } } =\sum _{ i=1 }^{ n }{ { y }_{ i }{ x }_{ i } }. \]
Lo cual implica que;
\[\frac { \partial  }{ \partial a } \left\{ \frac { 1 }{ n } \sum _{ i=1 }^{ n }{ ({ a{ x }_{ i }+b-{ y }_{ i } })^{ 2 } }  \right\} =\frac { 1 }{ n } \frac { \partial  }{ \partial b } \sum _{ i=1 }^{ n }{ { (a{ x }_{ i }+b-{ y }_{ i }) }^{ 2 }= } \frac { 1 }{ n } \sum _{ i=1 }^{ n }{ { \frac { \partial  }{ \partial b }  }{ (a{ x }_{ i }+b-{ y }_{ i }) }^{ 2 } } =\frac { 1 }{ 2 } \sum _{ i=1 }^{ n }{ 2(a{ x }_{ i }+b-{ y }_{ i })\quad ; \]\\
\[ \Longrightarrow \sum _{ i=1 }^{ n }{ { x }_{ i } } =\sum _{ i=1 }^{ n }{ { y }_{ i } } .\]
Luego; 
\[a=\frac { 1 }{ \Delta  } \left| \begin{matrix} \sum _{ i=1 }^{ n }{ { x }_{ i }{ y }_{ i } }  & \sum _{ i=1 }^{ n }{ { x }_{ i } }  \\ \sum _{ i=1 }^{ n }{ x_{ i }^{ 2 } }  & n \end{matrix} \right| \quad y\quad b=\frac { 1 }{ \Delta  } \left| \begin{matrix} \sum _{ i=1 }^{ n }{ { x }_{ i }^{ 2 } }  & \sum _{ i=1 }^{ n }{ { x }_{ i }{ y }_{ i } }  \\ \sum _{ i=1 }^{ n }{ x_{ i } }  & \sum _{ i=1 }^{ n }{ { y }_{ i } }  \end{matrix} \right| \]
Finalmente tenemos que; 
\[a=\frac { n\sum _{ i=1 }^{ n }{ { x }_{ i }{ y }_{ i } } -\sum _{ i=1 }^{ n }{ { x }_{ i } } \sum _{ i=1 }^{ n }{ { y }_{ i } }  }{ n\sum _{ i=1 }^{ n }{ { x }_{ i }^{ 2 } } -{ \left( \sum _{ i=1 }^{ n }{ { x }_{ i } }  \right)  }^{ 2 } } \quad y\quad b=\frac { \sum _{ i=1 }^{ n }{ { x }_{ i }^{ 2 } } \sum _{ i=1 }^{ n }{ { y }_{ i } } -\sum _{ i=1 }^{ n }{ { x }_{ i }{ y }_{ i } } \sum _{ i=1 }^{ n }{ { x }_{ i } }  }{ n\sum _{ i=1 }^{ n }{ { x }_{ i }^{ 2 } } -{ \left( \sum _{ i=1 }^{ n }{ { x }_{ i } }  \right)  }^{ 2 } }.\]
Las cuales satistfacen la ecuaci\'{o}n lineal que aproxima a los puntos discretos. 
\begin{equation}
Y= ax+b
\end{equation}

\end{document}