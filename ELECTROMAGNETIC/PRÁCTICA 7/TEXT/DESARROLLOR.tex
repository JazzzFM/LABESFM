\documentclass[11pt,a4paper]{article}
\usepackage[latin1]{inputenc}
\usepackage[spanish]{babel}
\usepackage{amsmath}
\usepackage{amsfonts}
\usepackage{amssymb}
\usepackage{makeidx}
\usepackage{graphicx}
\usepackage{lmodern}
\usepackage[left=2cm,right=2cm,top=2cm,bottom=2cm]{geometry}
\author{Flores Rodguez Jaziel David }
\title{Pr\'{a}ctica 1}
\begin{document}

\tableofcontents
%----------------------------------------------------------------------------------------
%	SECTION 1
%----------------------------------------------------------------------------------------
\\
\medskip
\medskip
\section{Resumen.}
\\
En esta pr\'{a}ctica, teniendo como antecedente la pr\'{a}ctica de conductores donde medimos la resistencia y la resistividad y aqu\'{i} los conectamos a una fuente de voltaje para formar una teor\'{i}a que empalme con lo esperado para este tipo de materiales, encontramos una expresi\'{o}n para las resistencias en diferentes configuraciones como los son serie y paralelo, y a\'{i} deducir para cualquier configuraci\'{o}n.
\\
\medskip
Palabras clave: voltaje, resistencia, corriente.

%----------------------------------------------------------------------------------------
%	SECTION 2
%----------------------------------------------------------------------------------------

\section{Objetivos.}\\
Estudiar la relaci\'{o}n entre el voltaje, la corriente y la resistencia en circuitos que contengan resistencias en serie, en paralelo y en serie-paralelo
\\
\pagebreak


%----------------------------------------------------------------------------------------
%	SECTION 3
%----------------------------------------------------------------------------------------
\section{Metodolog\'{i}a.}
\textbf{Resistencias en serie}. Se dice que las resistencias est\'{a}n en serie cuando se conectan como se indica en la Figura l. As\'{i} la corriente fluye sin cambios de una resistencia a otra.
\begin{figure}[hbtp]
\centering
\includegraphics[width=9cm]{../../../../../../Pictures/practca8.jpg}
\end{figure}
\\
En un circuito que contenga resistencias en serie se aplican las siguientes condiciones.\\
1) La corriente en todas las partes del circuito es la misma
\[{I}_{t} ={I}_{1}={I}_{2}={I}_{3} \quad (1) \]
2) El voltaje aplicado al circuito es igual a la suma de voltajes de cada resistencia
\[ {V}_{t} = {V}_{1} + {V}_{2} + {V}_{3}  \quad (2) \]
3) La resistencia total del circuito en serie es igual a la suma de las resistencias individuales.
\[{R}_{t} = {R}_{1}+ {R}_{2} + {R}_{3} \quad (3)\]
Resistencias en paralelo.- Cuando las resistencias se conectan como se indica en la Figura 2, se dice que est\'{a}n en paralelo. Para resistencias en paralelo se aplican las siguientes condiciones.
\begin{figure}[hbtp]
\caption{j}
\centering
\includegraphics[width=8cm]{../../../../../../Pictures/practica8888.jpg}
\end{figure}
\\
4) La corriente total es igual a la suma de las corrientes individuales.
\[{I}_{t} = {I}_{1} + {I}_{2} + {I}_{3} \quad (4)\]
5) El voltaje total del circuito es igual al voltaje de cualquiera de sus elementos.
\[ {V}_{t} = {V}_{1} = {V}_{2} = {V}_{3} \quad (5) \]
6) El rec\'{i}proco de la resistencia total es igual a la suma de los recíprocos de cada resistencia.
\[\frac{1}{{R}_{t}} = \frac{1}{{R}_{1}}+ \frac{1}{{R}_{2}} + \frac{1}{{R}_{3}} \quad (6) \]
\[{I}_{t} = {I}_{1} + {I}_{2} + {I}_{3} \quad (7) \]
\[\frac{V}{{R}_{t}} = \frac{V}{{R}_{1}}+ \frac{V}{{R}_{2}} + \frac{V}{{R}_{3}} \quad (8)\]
\[{R}_{t}= \frac{1}{\frac{1}{{R}_{1}} + \frac{1}{{R}_{2}} + \frac{1}{{R}_{3}}} \quad (9) \]
\[{R}_{t} = \frac{{R}_{1}{R}_{2} {R}_{3}}{{R}_{1}{R}_{2} + {R}_{1}{R}_{3} + {R}_{2} {R}_{3}} \quad (10) \]
Los seis casos anteriores se estudiar\'{a}n en esta pr\'{a}ctica. Es importante notar que la conexi\'{o}n de resistencias adicionales en serie aumenta la resistencia total, mientras que conectando adicionalmente resistencias en paralelo disminuye la resistencia total.
%----------------------------------------------------------------------------------------
%	SECTION 4
%----------------------------------------------------------------------------------------

\section{Instrumentaci\'{o}n.}\\
1).- Fuente regulada 400 V, 150 mA.\\
2).- Fuente regulada 40 V, 10 A.\\
3).- Tablero para conexiones con resistencias.\\
4).- Resistencias de diferentes valores.\\
5).- Volt\'{i}metro (V.O.M.)\\
6).- Amper\'{i}metro ( V.O.M.)\\
7).- Cables de conexi\'{o}n.\\
\pagebreak

%----------------------------------------------------------------------------------------
%	SECTION 5
%----------------------------------------------------------------------------------------


\section{Datos y Resultados.}\\
\textbf{PARTE 1: Resistencias en Serie.}
\\
\begin{figure}[hbtp]
\centering
\includegraphics[width=8cm]{../../../../../../Pictures/PITER.jpg}
\end{figure}
\\
Primero armamos un circuito en serie midiendo primeramente las resistencias ${R}_{1}$, ${R}_{2}$, ${R}_{3}$ y ${R}_{4}$ con el c\'{o}digo de colores de la misma resistencia y despu\'{e}s con el puente de impedancias, como se indica en la Figura 3, despu\'{e}s aplicamos un voltaje de 5 volts y medimos la corriente en los puntos indicados por los amper\'{i}metros ${A}_{1}$, ${A}_{2}$ , ${A}_{3}$, ${A}_{4}$. 
\begin{figure}[hbtp]
\centering
\includegraphics[width=8cm]{../../../../../../Pictures/sASjsNADKJBadjbAJBLjabvj.jpg}
\end{figure}
\\
Con el volt\'{i}metro medimos a trav\'{e}s de cada una de las resistencias y de la diferencia de potencial entre todas las resistencias como se indica en la Figura 4. Determinando la diferencia porcentual entre la lectura ${V}_{4}$ y la suma de ${V}_{1}$, ${V}_{2}$, ${V}_{3}$, mostramos a partir de los datos que el voltaje es directamente proporcional a la resistencia cuando la corriente es constante. As\'{i} pues, tenemos que el valor real de las resistencias es:
\\
\\
${R}_{1}$= 1 K$\Omega$         ${R}_{2}$=828 $\Omega$      ${R}_{3}$=480 $\Omega$       ${R}_{4}$=804 $\Omega$ 
\\
\\
Valor de la resistencia medida con el Puente de impedancias:
\\
\\
${R}_{1}$= 1 k$\Omega$        ${R}_{2}$=820 $\Omega$      ${R}_{3}$=470 $\Omega$       ${R}_{4}$=804 $\Omega$
\\
\\
Con el amper\'{i}metro medimos las siguientes corrientes:
\\
\\
       ${I}_{1}$= 0.0022 A       ${I}_{2}$= 0.0022 A     ${I}_{3}$= 0.0022 A    ${I}_{4}$=0.0022 A
\\
\\
Con el volt\'{i}metro medimos las siguientes diferencias de potencial:
\\
\\
      ${V}_{AB}$=2.17 v    ${V}_{BC}$=1.78 v    ${V}_{CD}$= 1.03 v     ${V}_{AD}$ = 5 v
       
       \[{V}_{AB} + {V}_{BC} + {V}_{CD} =2.17 + 1.78 + 1.03=4.98 V\]

Comparado con ${V}_{AD}$= 5V hay una diferencia de : -${V}_{AB}$ - ${V}_{BC}$ - ${V}_{CD}$ + ${V}_{AD}$ = 5 V - 4.98 V =0.02 V. Existe una diferencia porcentual del 2$\%$.  Como ${I}_{1}={I}_{2}={I}_{3}$ entonces:  V = IR.\\ 
Por lo tanto el voltaje es directamente proporcional a la resistencia cuando la corriente es constante.
\section*{Discusi\'{o}n.}
Podemos concluir que la conexi\'{o}n de resistencias adicionales en serie aumenta la resistencia total, mientras que conectando adicionalmente resistencias en paralelo disminuye la resistencia total.Se aprendi\'{o} tambi\'{e}n que cuando tenemos conexiones de resistencias en serie el voltaje se conserva y la corriente es la misma para todos los puntos, al igual que la resistencia se conserva.
\\
\\
\textbf{PARTE 2: Resistencias en paralelo.}
Ahora armamos un circuito en paralelo como en la figura 5, conectamos el conjunto de resistencias a la fuente por medio del amper\'{i}metro ${A}_{4}$, medimos la corriente indicada por el amper\'{i}metro ${A}_{4}$, despu\'{e}s abrimos la conexi\'{o}n de la resistencia ${R}_{1}$ e insertamos el amper\'{i}metro ${A}_{1}$ en serie con la resistencia anotamos esta corriente ${I}_{1}$, similarmente medimos ${I}_{2}$ e ${I}_{3}$. Comparamos ${I}_{4}$ con la suma de ${I}_{1}$, ${I}_{2}$ e ${I}_{3}$. De esto demostramos que la corriente para cada resistencia varia inversamente proporcional a la resistencia. Medimos el voltaje entre A y B y despu\'{e}s el voltaje entre cada resistencia. Quitamos las resistencias ${R}_{1}$ y ${R}_{2}$ la resistencia restante ${R}_{3}$ la sustituimos por una cuyo valor permita que circule corriente igual a la medida inicialmente por ${A}_{4}$, anote cual es el valor de esta resistencia y la comparamos con el valor de Req, obtuvimos el porcentaje de error y de sus conclusiones.
\begin{figure}[hbtp]
\centering
\includegraphics[width=8cm]{../../../../../../Pictures/JAJAJAJAJAJAJAJAJAJAJAJ.jpg}
\end{figure}
\\
Medimos la corriente indicada por el amper\'{i}metro ${A}_{4}$ = 0.0215 A, despu\'{e}s abrimos la conexi\'{o}n de la resistencia ${R}_{1}$ e insertamos el amper\'{i}metro A1 en serie y anotamos la corriente ${I}_{1}$ que registra el amper\'{i}metro ${A}_{1}$ y as\'{i} ${I}_{2}$ e ${I}_{3}$:
\\
\\
${I}_{1}$=0.005 A   ${I}_{2}$= 0.006  A    ${I}_{3}$=0.0105  A   ${I}_{4}$ = 0.0215 A  y  ${I}_{1} + {I}_{2} + {I}_{3}= 0.0215$ A, por lo que la diferencia es cero, lo que significa que no tiene error.Ahora mostraremos que la corriente para cada resistencia varia inversamente proporcional a la resistencia. Como:
\\
IR=${I}_{1} {R}_{1}$=${I}_{2} {R}_{2}$=${I}_{3} {R}_{3}$
\\
Por lo tanto
\[ \frac{{I}_{1}}{{I}_{2}} = \frac{{R}_{2}}{{R}_{1}}, \quad \frac{{I}_{2}}{{I}_{3}}= \frac{{R}_{3}}{{R}_{2}} \]
   
${V}_{AB}$ = 4.99 V ${V}_{BC}$ = 4.99 V  ${V}_{CD}$ = 4.99 V  ${V}_{AD}$ = 4.99 V.
\\
No se pueden comparar los valores de la diferencia de potencial puesto que son el mismo valor para cada punto del circuito.
\section*{Discusi\'{o}n.}
Tambi\'{e}n se aprendi\'{o} que las resistencias conectadas en paralelo la corriente se conserva y la deferencia de potencial es el mismo en todos los puntos, pero aparte de eso hay un hecho interesante este es que la suma de los inversos de las resistencias da el inverso de la resistencia equivalente.
\\
\\
\textbf{PARTE 3: Resistencias en combinaci\'{o}n serie-paralelo.} Conectamos las tres resistencias como nos indic\'{o} el profesor, Figura 6.(Resistencias en combinaci\'{o}n serie-paralelo). Medimos el voltaje entre AB, BC y AC y con los valores obtenidos calculamos la  corriente que circula en cada resistencia y finalmente medimos cada una de las corriente y las comparamos con el valor calculado.
\begin{figure}[hbtp]
 \centering
 \includegraphics[width=8cm]{../../../../../../Pictures/SSAJEJEJEJASJAJJAJAJAJJAJAJA.jpg}
 \end{figure}
\\
Medimos el voltaje entre AB = 3.87 V, BC = 1.12 V y AC = 4.99 V.Calculemos la corriente que circula en cada resistencia.
\\
${V}_{AB}$=${I}_{AB} {R}_{1}$ entonces ${I}_{AB}= \frac{{V}_{AB}}{{R}_{1}}= \frac{3.87 V}{1 k \Omega} = 0.00387 A$
\\
${V}_{BC} = {I}_{BC}{R}_{2}$ entonces ${I}_{BC} = \frac{{V}_{BC}}{{R}_{2}} = \frac{1.12 V}{820 \Omega}= 0.00137$ A.
\\
${I}_{BC} = \frac{V_{BC}}{{R}_{3}} = \frac{1.12 V}{480\Omega} = 0.00233$ A
 \\
Medimos cada una de las corrientes:
\\ 
${I}_{A}$= 0.0048 A ${I}_{B}$= 0.0048 A	${I}_{C}$= 0.0023 A	${I}_{D}$= 0.0014 A	${I}_{E}$= 0.0011 A.
\section*{Discusi\'{o}n.}
Conocimos tambi\'{e}n ahora otra forma de conexi\'{o}n, en esta forma conectamos resistencias en serie y paralelo. Las cuales ayudan a  aumentar o disminuir las resistencias del circuito dependiendo de nuestras necesidades.
\\
\\
\section{Preguntas.}
\\
1.- Conecte un amper\'{i}metro a un circuito para medir la corriente. Afectar\'{a} este el valor de la corriente medida? Es necesario para un amper\'{i}metro que tenga baja o alta resistencia?\\
No mucho, puesto que el amper\'{i}metro tiene una resistencia muy baja para no afectar la corriente. Es necesario que el amper\'{i}metro tenga muy baja resistencia por que por la ecuaci\'{o}n I = V/R, al momento de que la resistencia sea alta alterar\'{i}a la medici\'{o}n de la corriente.
\\
2.- Ponga un volt\'{i}metro dentro de un circuito para medir el voltaje a trav\'{e}s de una resistencia. Cambia el voltaje a trav\'{e}s de la resistencia? Explique claramente. Es deseable para un volt\'{i}metro que tenga la resistencia grande o peque\~{n}a?\\
Si la corriente es constante el voltaje no cambiar\'{i}a puesto que V = IR. En este caso es deseable que la resistencia sea grande para poder medir voltajes altos dado que V = IR, entonces si R es grande V tambi\'{e}n lo ser\'{a} cuando la i sea constante.
\\
3.- Una serie de \'{a}rbol de navidad se hace frecuentemente de focos de miniatura conectados en serie, para un conjunto de ocho l\'{a}mparas y alimentaci\'{o}n de 120 V, Cu\'{a}l es el voltaje a trav\'{e}s de cada l\'{a}mpara? Si se quita una l\'{a}mpara, Qu\'{e} ocurre?.
\\
El voltaje de cada l\'{a}mpara es  de 15 V. Si se quita una l\'{a}mpara las dem\'{a}s no funcionan puesto que no se cierra el circuito.
\\
4.- Una pieza de alambre de cobre se corta en 10 partes iguales, estas partes son conectadas en paralelo. Cu\'{a}l ser\'{a} la resistencia combinada en paralelo comparada con la resistencia original de todo el alambre?
\\
Notemos que la conexi\'{o}n de resistencias en paralelo desminuye la resistencia total comparada con la resistencia original de todo el alambre. 

%----------------------------------------------------------------------------------------
%	SECTION 6
%---------------------------------------------------------------------------------------


\section{Conclusiones.}
Se comprendi\'r{o} la manera de manejar las conexiones entre resistencias de acuerdo a nuestras necesidades, se compar\'{o}el resultado te\'{o}rico con el pr\'{a}ctico y \'{e}sta vez extra\~{n}amente nuestros resultados fueron safisfactrios, se puede decir que la Ley de Ohm es muy exacta en cuanto a stos para\'{a}metros. Cabe se\~{n}alar que hubo algunos inconvenientes al tratar de interpretar los diagramas de conexi\'{o} pero gracias a la ayuda del profesor y del t\'{e}cnico que nos ayud\'{o} a conetar la fuente se pudo realizar correctamente la medici\'{o}n con el mult\'{i}metro. 
%----------------------------------------------------------------------------------------
%	SECTION 9
%----------------------------------------------------------------------------------------
 \section{Bibliograf\'{i}a.}
1.-https://es.wikipedia.org/wiki/Conductor_elC3A9ctrico\\
2.-https://unicrom.com/codigo-de-colores-de-las-resistencias/.\\
3.-https://sites.google.com/site/labenriquesalgadoruiz/home/politecnico-1/fisica-iii .\\
4.-Resnick/Halliday/Krane. Fundamentos de F\'{i}sica. Volumen 2. Edici\'{o}n 6, extendida. CESA\\

%----------------------------------------------------------------------------------------
%	SECTION 10
%----------------------------------------------------------------------------------------

\end{document}