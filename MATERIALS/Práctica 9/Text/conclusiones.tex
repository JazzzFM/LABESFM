 \documentclass[10pt,a4paper]{article}
\usepackage[latin2]{inputenc}
\usepackage[spanish]{babel}
\usepackage{amsmath}
\usepackage{amsfonts}
\usepackage{amssymb}
\usepackage{makeidx}
\usepackage{graphicx}
\usepackage[left=2cm,right=2cm,top=2cm,bottom=2cm]{geometry}

\begin{document}

\section*{ Conclusiones:}\\
Se puede concluir que: la longitud de onda disminuye si la frecuencia aumenta. Si hay una mayor tensi\'{o}n sobre la cuerda, la longitud de onda aumentara, ya que estas tienen un comportamiento directamente proporcional. Los nodos son puntos de la cuerda donde no se trasmite energ\'{i}a en estos, en cambio en los antinodos son los puntos donde la amplitud es m\'{a}xima. La velocidad de propagaci\'{o}n de una onda depende de la tensi\'{o}n que hay en la cuerda por tanto a un aumento de tensi\'{o}n en una misma cuerda, su velocidad ser\'{a} mayor. Al aumentar la frecuencia la longitud de onda (lambda) disminuye porque ante el aumento de la frecuencia empiezan a parecer una mayor cantidad de nodos y antinodos (arm\'{o}nicos), haciendo que lambda disminuya.

\section*{8. Referencias:}\\
\\
\medskip
\\
\\1.- Bit\'{a}cora de laboratorio de Flores Rodr\'{i}guez Jaziel David.
\\
2.- Manual de pr\'{a}cticas auxilar. Autor: Fco. Havez Varela y las notas del profesor Salvador Tirado Guerra.
\\
3.- Fisica Universitaria - Sears - Zemansky - 12ava Edici\'{o}n - Cap\'{i}tulo l7 -2009.\\
4.- http://es.wikipedia.org/wiki/Onda_estacionaria .\\ 
5.- http://www.uclm.es/profesorado/ajbarbero/CursoAB2007/OndasEstacionarias06.pdf .\\
6.- http://web.educastur.princast.es/proyectos/fisquiweb/Videos/OndasEstacionarias/index.htm.
\\
\end{document}