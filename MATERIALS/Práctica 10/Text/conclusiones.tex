 \documentclass[10pt,a4paper]{article}
\usepackage[latin2]{inputenc}
\usepackage[spanish]{babel}
\usepackage{amsmath}
\usepackage{amsfonts}
\usepackage{amssymb}
\usepackage{makeidx}
\usepackage{graphicx}
\usepackage[left=2cm,right=2cm,top=2cm,bottom=2cm]{geometry}

\begin{document}

\section*{ Conclusiones:}\\
La relaci\'{o}n entre la frecuencia y el sonido nos dice que a medida que aumenta la presi\'{o}n en el interior, aumenta la amplitud de la onda estacionaria, adem\'{aa}s que se cumple la ecuaci\'{o}n para la velocidad del sonido y este tiende a ser lineal.

\section*{8. Referencias:}\\
\\
\\
\\1.- Bit\'{a}cora de laboratorio de Flores Rodr\'{i}guez Jaziel David.
\\
2.- Manual de pr\'{a}cticas auxilar. Autor: Fco. Havez Varela y las notas del profesor Salvador Tirado Guerra.
\\
3.- Fisica Universitaria - Sears - Zemansky - 12ava Edici\'{o}n - Cap\'{i}tulo l7 -2009.\\
4.- https://www.ucm.es/data/cont/docs/76-2013-07-11-09_Rubens_tube.pdf.\\ 
5.- http://hyperphysics.phy-astr.gsu.edu/hbasees/Sound/reson.html.\\
\\
\end{document}